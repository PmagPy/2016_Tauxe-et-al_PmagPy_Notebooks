
% Default to the notebook output style

    


% Inherit from the specified cell style.




    
\documentclass{article}

    
    
    \usepackage{graphicx} % Used to insert images
    \usepackage{adjustbox} % Used to constrain images to a maximum size 
    \usepackage{color} % Allow colors to be defined
    \usepackage{enumerate} % Needed for markdown enumerations to work
    \usepackage{geometry} % Used to adjust the document margins
    \usepackage{amsmath} % Equations
    \usepackage{amssymb} % Equations
    \usepackage{eurosym} % defines \euro
    \usepackage[mathletters]{ucs} % Extended unicode (utf-8) support
    \usepackage[utf8x]{inputenc} % Allow utf-8 characters in the tex document
    \usepackage{fancyvrb} % verbatim replacement that allows latex
    \usepackage{grffile} % extends the file name processing of package graphics 
                         % to support a larger range 
    % The hyperref package gives us a pdf with properly built
    % internal navigation ('pdf bookmarks' for the table of contents,
    % internal cross-reference links, web links for URLs, etc.)
    \usepackage{hyperref}
    \usepackage{longtable} % longtable support required by pandoc >1.10
    \usepackage{booktabs}  % table support for pandoc > 1.12.2
    \usepackage{ulem} % ulem is needed to support strikethroughs (\sout)
    

    
    
    \definecolor{orange}{cmyk}{0,0.4,0.8,0.2}
    \definecolor{darkorange}{rgb}{.71,0.21,0.01}
    \definecolor{darkgreen}{rgb}{.12,.54,.11}
    \definecolor{myteal}{rgb}{.26, .44, .56}
    \definecolor{gray}{gray}{0.45}
    \definecolor{lightgray}{gray}{.95}
    \definecolor{mediumgray}{gray}{.8}
    \definecolor{inputbackground}{rgb}{.95, .95, .85}
    \definecolor{outputbackground}{rgb}{.95, .95, .95}
    \definecolor{traceback}{rgb}{1, .95, .95}
    % ansi colors
    \definecolor{red}{rgb}{.6,0,0}
    \definecolor{green}{rgb}{0,.65,0}
    \definecolor{brown}{rgb}{0.6,0.6,0}
    \definecolor{blue}{rgb}{0,.145,.698}
    \definecolor{purple}{rgb}{.698,.145,.698}
    \definecolor{cyan}{rgb}{0,.698,.698}
    \definecolor{lightgray}{gray}{0.5}
    
    % bright ansi colors
    \definecolor{darkgray}{gray}{0.25}
    \definecolor{lightred}{rgb}{1.0,0.39,0.28}
    \definecolor{lightgreen}{rgb}{0.48,0.99,0.0}
    \definecolor{lightblue}{rgb}{0.53,0.81,0.92}
    \definecolor{lightpurple}{rgb}{0.87,0.63,0.87}
    \definecolor{lightcyan}{rgb}{0.5,1.0,0.83}
    
    % commands and environments needed by pandoc snippets
    % extracted from the output of `pandoc -s`
    \providecommand{\tightlist}{%
      \setlength{\itemsep}{0pt}\setlength{\parskip}{0pt}}
    \DefineVerbatimEnvironment{Highlighting}{Verbatim}{commandchars=\\\{\}}
    % Add ',fontsize=\small' for more characters per line
    \newenvironment{Shaded}{}{}
    \newcommand{\KeywordTok}[1]{\textcolor[rgb]{0.00,0.44,0.13}{\textbf{{#1}}}}
    \newcommand{\DataTypeTok}[1]{\textcolor[rgb]{0.56,0.13,0.00}{{#1}}}
    \newcommand{\DecValTok}[1]{\textcolor[rgb]{0.25,0.63,0.44}{{#1}}}
    \newcommand{\BaseNTok}[1]{\textcolor[rgb]{0.25,0.63,0.44}{{#1}}}
    \newcommand{\FloatTok}[1]{\textcolor[rgb]{0.25,0.63,0.44}{{#1}}}
    \newcommand{\CharTok}[1]{\textcolor[rgb]{0.25,0.44,0.63}{{#1}}}
    \newcommand{\StringTok}[1]{\textcolor[rgb]{0.25,0.44,0.63}{{#1}}}
    \newcommand{\CommentTok}[1]{\textcolor[rgb]{0.38,0.63,0.69}{\textit{{#1}}}}
    \newcommand{\OtherTok}[1]{\textcolor[rgb]{0.00,0.44,0.13}{{#1}}}
    \newcommand{\AlertTok}[1]{\textcolor[rgb]{1.00,0.00,0.00}{\textbf{{#1}}}}
    \newcommand{\FunctionTok}[1]{\textcolor[rgb]{0.02,0.16,0.49}{{#1}}}
    \newcommand{\RegionMarkerTok}[1]{{#1}}
    \newcommand{\ErrorTok}[1]{\textcolor[rgb]{1.00,0.00,0.00}{\textbf{{#1}}}}
    \newcommand{\NormalTok}[1]{{#1}}
    
    % Additional commands for more recent versions of Pandoc
    \newcommand{\ConstantTok}[1]{\textcolor[rgb]{0.53,0.00,0.00}{{#1}}}
    \newcommand{\SpecialCharTok}[1]{\textcolor[rgb]{0.25,0.44,0.63}{{#1}}}
    \newcommand{\VerbatimStringTok}[1]{\textcolor[rgb]{0.25,0.44,0.63}{{#1}}}
    \newcommand{\SpecialStringTok}[1]{\textcolor[rgb]{0.73,0.40,0.53}{{#1}}}
    \newcommand{\ImportTok}[1]{{#1}}
    \newcommand{\DocumentationTok}[1]{\textcolor[rgb]{0.73,0.13,0.13}{\textit{{#1}}}}
    \newcommand{\AnnotationTok}[1]{\textcolor[rgb]{0.38,0.63,0.69}{\textbf{\textit{{#1}}}}}
    \newcommand{\CommentVarTok}[1]{\textcolor[rgb]{0.38,0.63,0.69}{\textbf{\textit{{#1}}}}}
    \newcommand{\VariableTok}[1]{\textcolor[rgb]{0.10,0.09,0.49}{{#1}}}
    \newcommand{\ControlFlowTok}[1]{\textcolor[rgb]{0.00,0.44,0.13}{\textbf{{#1}}}}
    \newcommand{\OperatorTok}[1]{\textcolor[rgb]{0.40,0.40,0.40}{{#1}}}
    \newcommand{\BuiltInTok}[1]{{#1}}
    \newcommand{\ExtensionTok}[1]{{#1}}
    \newcommand{\PreprocessorTok}[1]{\textcolor[rgb]{0.74,0.48,0.00}{{#1}}}
    \newcommand{\AttributeTok}[1]{\textcolor[rgb]{0.49,0.56,0.16}{{#1}}}
    \newcommand{\InformationTok}[1]{\textcolor[rgb]{0.38,0.63,0.69}{\textbf{\textit{{#1}}}}}
    \newcommand{\WarningTok}[1]{\textcolor[rgb]{0.38,0.63,0.69}{\textbf{\textit{{#1}}}}}
    
    
    % Define a nice break command that doesn't care if a line doesn't already
    % exist.
    \def\br{\hspace*{\fill} \\* }
    % Math Jax compatability definitions
    \def\gt{>}
    \def\lt{<}
    % Document parameters
    \title{Example\_PmagPy\_Notebook}
    
    
    

    % Pygments definitions
    
\makeatletter
\def\PY@reset{\let\PY@it=\relax \let\PY@bf=\relax%
    \let\PY@ul=\relax \let\PY@tc=\relax%
    \let\PY@bc=\relax \let\PY@ff=\relax}
\def\PY@tok#1{\csname PY@tok@#1\endcsname}
\def\PY@toks#1+{\ifx\relax#1\empty\else%
    \PY@tok{#1}\expandafter\PY@toks\fi}
\def\PY@do#1{\PY@bc{\PY@tc{\PY@ul{%
    \PY@it{\PY@bf{\PY@ff{#1}}}}}}}
\def\PY#1#2{\PY@reset\PY@toks#1+\relax+\PY@do{#2}}

\expandafter\def\csname PY@tok@gd\endcsname{\def\PY@tc##1{\textcolor[rgb]{0.63,0.00,0.00}{##1}}}
\expandafter\def\csname PY@tok@gu\endcsname{\let\PY@bf=\textbf\def\PY@tc##1{\textcolor[rgb]{0.50,0.00,0.50}{##1}}}
\expandafter\def\csname PY@tok@gt\endcsname{\def\PY@tc##1{\textcolor[rgb]{0.00,0.27,0.87}{##1}}}
\expandafter\def\csname PY@tok@gs\endcsname{\let\PY@bf=\textbf}
\expandafter\def\csname PY@tok@gr\endcsname{\def\PY@tc##1{\textcolor[rgb]{1.00,0.00,0.00}{##1}}}
\expandafter\def\csname PY@tok@cm\endcsname{\let\PY@it=\textit\def\PY@tc##1{\textcolor[rgb]{0.25,0.50,0.50}{##1}}}
\expandafter\def\csname PY@tok@vg\endcsname{\def\PY@tc##1{\textcolor[rgb]{0.10,0.09,0.49}{##1}}}
\expandafter\def\csname PY@tok@m\endcsname{\def\PY@tc##1{\textcolor[rgb]{0.40,0.40,0.40}{##1}}}
\expandafter\def\csname PY@tok@mh\endcsname{\def\PY@tc##1{\textcolor[rgb]{0.40,0.40,0.40}{##1}}}
\expandafter\def\csname PY@tok@go\endcsname{\def\PY@tc##1{\textcolor[rgb]{0.53,0.53,0.53}{##1}}}
\expandafter\def\csname PY@tok@ge\endcsname{\let\PY@it=\textit}
\expandafter\def\csname PY@tok@vc\endcsname{\def\PY@tc##1{\textcolor[rgb]{0.10,0.09,0.49}{##1}}}
\expandafter\def\csname PY@tok@il\endcsname{\def\PY@tc##1{\textcolor[rgb]{0.40,0.40,0.40}{##1}}}
\expandafter\def\csname PY@tok@cs\endcsname{\let\PY@it=\textit\def\PY@tc##1{\textcolor[rgb]{0.25,0.50,0.50}{##1}}}
\expandafter\def\csname PY@tok@cp\endcsname{\def\PY@tc##1{\textcolor[rgb]{0.74,0.48,0.00}{##1}}}
\expandafter\def\csname PY@tok@gi\endcsname{\def\PY@tc##1{\textcolor[rgb]{0.00,0.63,0.00}{##1}}}
\expandafter\def\csname PY@tok@gh\endcsname{\let\PY@bf=\textbf\def\PY@tc##1{\textcolor[rgb]{0.00,0.00,0.50}{##1}}}
\expandafter\def\csname PY@tok@ni\endcsname{\let\PY@bf=\textbf\def\PY@tc##1{\textcolor[rgb]{0.60,0.60,0.60}{##1}}}
\expandafter\def\csname PY@tok@nl\endcsname{\def\PY@tc##1{\textcolor[rgb]{0.63,0.63,0.00}{##1}}}
\expandafter\def\csname PY@tok@nn\endcsname{\let\PY@bf=\textbf\def\PY@tc##1{\textcolor[rgb]{0.00,0.00,1.00}{##1}}}
\expandafter\def\csname PY@tok@no\endcsname{\def\PY@tc##1{\textcolor[rgb]{0.53,0.00,0.00}{##1}}}
\expandafter\def\csname PY@tok@na\endcsname{\def\PY@tc##1{\textcolor[rgb]{0.49,0.56,0.16}{##1}}}
\expandafter\def\csname PY@tok@nb\endcsname{\def\PY@tc##1{\textcolor[rgb]{0.00,0.50,0.00}{##1}}}
\expandafter\def\csname PY@tok@nc\endcsname{\let\PY@bf=\textbf\def\PY@tc##1{\textcolor[rgb]{0.00,0.00,1.00}{##1}}}
\expandafter\def\csname PY@tok@nd\endcsname{\def\PY@tc##1{\textcolor[rgb]{0.67,0.13,1.00}{##1}}}
\expandafter\def\csname PY@tok@ne\endcsname{\let\PY@bf=\textbf\def\PY@tc##1{\textcolor[rgb]{0.82,0.25,0.23}{##1}}}
\expandafter\def\csname PY@tok@nf\endcsname{\def\PY@tc##1{\textcolor[rgb]{0.00,0.00,1.00}{##1}}}
\expandafter\def\csname PY@tok@si\endcsname{\let\PY@bf=\textbf\def\PY@tc##1{\textcolor[rgb]{0.73,0.40,0.53}{##1}}}
\expandafter\def\csname PY@tok@s2\endcsname{\def\PY@tc##1{\textcolor[rgb]{0.73,0.13,0.13}{##1}}}
\expandafter\def\csname PY@tok@vi\endcsname{\def\PY@tc##1{\textcolor[rgb]{0.10,0.09,0.49}{##1}}}
\expandafter\def\csname PY@tok@nt\endcsname{\let\PY@bf=\textbf\def\PY@tc##1{\textcolor[rgb]{0.00,0.50,0.00}{##1}}}
\expandafter\def\csname PY@tok@nv\endcsname{\def\PY@tc##1{\textcolor[rgb]{0.10,0.09,0.49}{##1}}}
\expandafter\def\csname PY@tok@s1\endcsname{\def\PY@tc##1{\textcolor[rgb]{0.73,0.13,0.13}{##1}}}
\expandafter\def\csname PY@tok@kd\endcsname{\let\PY@bf=\textbf\def\PY@tc##1{\textcolor[rgb]{0.00,0.50,0.00}{##1}}}
\expandafter\def\csname PY@tok@sh\endcsname{\def\PY@tc##1{\textcolor[rgb]{0.73,0.13,0.13}{##1}}}
\expandafter\def\csname PY@tok@sc\endcsname{\def\PY@tc##1{\textcolor[rgb]{0.73,0.13,0.13}{##1}}}
\expandafter\def\csname PY@tok@sx\endcsname{\def\PY@tc##1{\textcolor[rgb]{0.00,0.50,0.00}{##1}}}
\expandafter\def\csname PY@tok@bp\endcsname{\def\PY@tc##1{\textcolor[rgb]{0.00,0.50,0.00}{##1}}}
\expandafter\def\csname PY@tok@c1\endcsname{\let\PY@it=\textit\def\PY@tc##1{\textcolor[rgb]{0.25,0.50,0.50}{##1}}}
\expandafter\def\csname PY@tok@kc\endcsname{\let\PY@bf=\textbf\def\PY@tc##1{\textcolor[rgb]{0.00,0.50,0.00}{##1}}}
\expandafter\def\csname PY@tok@c\endcsname{\let\PY@it=\textit\def\PY@tc##1{\textcolor[rgb]{0.25,0.50,0.50}{##1}}}
\expandafter\def\csname PY@tok@mf\endcsname{\def\PY@tc##1{\textcolor[rgb]{0.40,0.40,0.40}{##1}}}
\expandafter\def\csname PY@tok@err\endcsname{\def\PY@bc##1{\setlength{\fboxsep}{0pt}\fcolorbox[rgb]{1.00,0.00,0.00}{1,1,1}{\strut ##1}}}
\expandafter\def\csname PY@tok@mb\endcsname{\def\PY@tc##1{\textcolor[rgb]{0.40,0.40,0.40}{##1}}}
\expandafter\def\csname PY@tok@ss\endcsname{\def\PY@tc##1{\textcolor[rgb]{0.10,0.09,0.49}{##1}}}
\expandafter\def\csname PY@tok@sr\endcsname{\def\PY@tc##1{\textcolor[rgb]{0.73,0.40,0.53}{##1}}}
\expandafter\def\csname PY@tok@mo\endcsname{\def\PY@tc##1{\textcolor[rgb]{0.40,0.40,0.40}{##1}}}
\expandafter\def\csname PY@tok@kn\endcsname{\let\PY@bf=\textbf\def\PY@tc##1{\textcolor[rgb]{0.00,0.50,0.00}{##1}}}
\expandafter\def\csname PY@tok@mi\endcsname{\def\PY@tc##1{\textcolor[rgb]{0.40,0.40,0.40}{##1}}}
\expandafter\def\csname PY@tok@gp\endcsname{\let\PY@bf=\textbf\def\PY@tc##1{\textcolor[rgb]{0.00,0.00,0.50}{##1}}}
\expandafter\def\csname PY@tok@o\endcsname{\def\PY@tc##1{\textcolor[rgb]{0.40,0.40,0.40}{##1}}}
\expandafter\def\csname PY@tok@kr\endcsname{\let\PY@bf=\textbf\def\PY@tc##1{\textcolor[rgb]{0.00,0.50,0.00}{##1}}}
\expandafter\def\csname PY@tok@s\endcsname{\def\PY@tc##1{\textcolor[rgb]{0.73,0.13,0.13}{##1}}}
\expandafter\def\csname PY@tok@kp\endcsname{\def\PY@tc##1{\textcolor[rgb]{0.00,0.50,0.00}{##1}}}
\expandafter\def\csname PY@tok@w\endcsname{\def\PY@tc##1{\textcolor[rgb]{0.73,0.73,0.73}{##1}}}
\expandafter\def\csname PY@tok@kt\endcsname{\def\PY@tc##1{\textcolor[rgb]{0.69,0.00,0.25}{##1}}}
\expandafter\def\csname PY@tok@ow\endcsname{\let\PY@bf=\textbf\def\PY@tc##1{\textcolor[rgb]{0.67,0.13,1.00}{##1}}}
\expandafter\def\csname PY@tok@sb\endcsname{\def\PY@tc##1{\textcolor[rgb]{0.73,0.13,0.13}{##1}}}
\expandafter\def\csname PY@tok@k\endcsname{\let\PY@bf=\textbf\def\PY@tc##1{\textcolor[rgb]{0.00,0.50,0.00}{##1}}}
\expandafter\def\csname PY@tok@se\endcsname{\let\PY@bf=\textbf\def\PY@tc##1{\textcolor[rgb]{0.73,0.40,0.13}{##1}}}
\expandafter\def\csname PY@tok@sd\endcsname{\let\PY@it=\textit\def\PY@tc##1{\textcolor[rgb]{0.73,0.13,0.13}{##1}}}

\def\PYZbs{\char`\\}
\def\PYZus{\char`\_}
\def\PYZob{\char`\{}
\def\PYZcb{\char`\}}
\def\PYZca{\char`\^}
\def\PYZam{\char`\&}
\def\PYZlt{\char`\<}
\def\PYZgt{\char`\>}
\def\PYZsh{\char`\#}
\def\PYZpc{\char`\%}
\def\PYZdl{\char`\$}
\def\PYZhy{\char`\-}
\def\PYZsq{\char`\'}
\def\PYZdq{\char`\"}
\def\PYZti{\char`\~}
% for compatibility with earlier versions
\def\PYZat{@}
\def\PYZlb{[}
\def\PYZrb{]}
\makeatother


    % Exact colors from NB
    \definecolor{incolor}{rgb}{0.0, 0.0, 0.5}
    \definecolor{outcolor}{rgb}{0.545, 0.0, 0.0}



    
    % Prevent overflowing lines due to hard-to-break entities
    \sloppy 
    % Setup hyperref package
    \hypersetup{
      breaklinks=true,  % so long urls are correctly broken across lines
      colorlinks=true,
      urlcolor=blue,
      linkcolor=darkorange,
      citecolor=darkgreen,
      }
    % Slightly bigger margins than the latex defaults
    
    \geometry{verbose,tmargin=1in,bmargin=1in,lmargin=1in,rmargin=1in}
    
    \parindent = 0.0 in
    \parskip = 0.1 in

    \begin{document}
    
   {\textbf{\LARGE{Jupyter notebook illustrating the use of PmagPy for analysis of
paleomagnetic
data}\label{jupyter-notebook-illustrating-the-use-of-pmagpy-for-analysis-of-paleomagnetic-data}}}

This notebook accompanies a submitted manuscript entitled:

\textbf{PmagPy: Software package for paleomagnetic data analysis and a
bridge to the Magnetics Information Consortium (MagIC) Database}

\textit{L. Tauxe, R. Shaar, L. Jonestrask, N.L. Swanson-Hysell, K.
Gaastra, L. Fairchild, N. Jarboe, R. Minnett, A.A.P. Koppers, and C.G.
Constable}

The notebook can be viewed as html at the following link where the code and tables are better rendered than in this PDF: \url{http://pmagpy.github.io/Example_PmagPy_Notebook.html}

\section{Before you begin}\label{before-you-begin}

You may be viewing this notebook as rendered html webpage or PDF in which case
you can go ahead and simply have a look at it. However, if you wish to
execute the code within a downloaded version of this notebook, it is
necessary to have an installed distribution of Python and to have
downloaded the PmagPy software distribution. The instructions in the
PmagPy Cookbook can help get you started:

http://earthref.org/PmagPy/cookbook

The user would also benefit from perusing the `Survival computer skills'
and `Introduction to Python Programming' chapters of the cookbook if
they are new to programming in Python.

\section{Introduction}\label{introduction}

The analysis in this notebook uses data from two contributions within
the MagIC database:

Halls, H. (1974), A paleomagnetic reversal in the Osler Volcanic Group,
northern Lake Superior, Can. J. Earth Sci., 11, 1200--1207,
doi:10.1139/e74-113. Link to MagIC contribution:
http://earthref.org/MAGIC/doi/10.1139/e74-113

and

Swanson-Hysell, N. L., A. A. Vaughan, M. R. Mustain, and K. E. Asp
(2014), Confirmation of progressive plate motion during the Midcontinent
Rift's early magmatic stage from the Osler Volcanic Group, Ontario,
Canada, Geochem. Geophys. Geosyst., 15, 2039--2047,
doi:10.1002/2013GC005180. Link to MagIC contribution:
http://earthref.org/MAGIC/doi/10.1002/2013GC005180

To explore the use of Jupyter notebooks, follow each of these links and
click on the text file icon in the Data column. Move each of the files
from your download directory into a `Project Directory' on your hard
drive. In the example below, the project directory for the Halls data
set is \textbf{Halls1974} and for the Swanson-Hysell data set,
\textbf{Swanson-Hysell2014} within a subdirectory with the notebook code
in it. Fire up a command line prompt (cmd on PCs and `terminal' in the
Applications/Utilities folder on a Mac) and type
\texttt{jupyter\ notebook} or \texttt{ipython\ notebook} to launch a
notebook environment within your default web browser.

In this notebook, we will us PmagPy to:

\begin{itemize}
\tightlist
\item
  Unpack data downloaded from the MagIC database.
\item
  Plot directions and VGPs.
\item
  Calculate and plot Fisher means for directions and VGPs.
\item
  Conduct a bootstrap fold test on the data.
\item
  Conduct a common mean test between the data from the two
  contributions.
\end{itemize}

    \section{Import necessary function libraries for data
analysis}\label{import-necessary-function-libraries-for-data-analysis}

    The code block below imports necessary modules from PmagPy that provide
functions that will be used in the data analysis. Using
`sys.path.insert' allows you to point to the directory where you keep
PmagPy in order to import it. \textbf{You will need to change the path
(in the form of `/Users/YOUR\_NAME/PmagPy') to match where the PmagPy
folder is on your computer.} Note that text after the pound sign is a
comment and will be ignored. To execute the code, click on play button
in the menu bar, choose run under the `cell' menu at the top of the
notebook, or type shift+enter.

With the PmagPy folder in your PYTHONPATH, the function modules from
\textbf{PmagPy} can be imported: \textbf{pmag}, a module with
\textasciitilde{}160 (and growing) functions for analyzing paleomagnetic
and rock magnetic data. \textbf{ipmag}, a module with functions that
combine and extend \textbf{pmag} functions and generate output that
works well within the Jupyter notebook environment, and
\textbf{ipmagplotlib}, a module with functions that make various plots
used in paleomagnetic data visualization.

    \begin{Verbatim}[commandchars=\\\{\}]
{\color{incolor}In [{\color{incolor}1}]:} \PY{k+kn}{import} \PY{n+nn}{pmagpy.ipmag} \PY{k+kn}{as} \PY{n+nn}{ipmag}
        \PY{k+kn}{import} \PY{n+nn}{pmagpy.pmagplotlib} \PY{k+kn}{as} \PY{n+nn}{pmagplotlib}
        \PY{k+kn}{import} \PY{n+nn}{pmagpy.pmag} \PY{k+kn}{as} \PY{n+nn}{pmag}
\end{Verbatim}

    There are three other important Python libraries (which are bundled with
the Canopy and Anaconda installations of Python) that come in quite
handy and are used within this notebook: \textbf{numpy} for data
analysis using arrays, \textbf{pandas} for data manipulation within
dataframes and \textbf{matplotlib} for plotting. The call
\texttt{\%matplotlib\ inline} results in the plots being shown within
this notebook rather than coming up in external windows.

    \begin{Verbatim}[commandchars=\\\{\}]
{\color{incolor}In [{\color{incolor}2}]:} \PY{k+kn}{import} \PY{n+nn}{numpy} \PY{k+kn}{as} \PY{n+nn}{np}
        \PY{k+kn}{import} \PY{n+nn}{pandas} \PY{k+kn}{as} \PY{n+nn}{pd}
        \PY{k+kn}{import} \PY{n+nn}{matplotlib.pyplot} \PY{k+kn}{as} \PY{n+nn}{plt}
        \PY{o}{\PYZpc{}}\PY{k}{matplotlib} inline
        \PY{o}{\PYZpc{}}\PY{k}{config} InlineBackend.figure\PYZus{}formats = \PYZob{}\PYZsq{}svg\PYZsq{},\PYZcb{}
\end{Verbatim}

    \section{Background on the Osler Volcanic
Group}\label{background-on-the-osler-volcanic-group}

    The data being analyzed in this notebook come from lava flows of the
Osler Volcanic Group which is a sequence of Midcontinent Rift lava flows
exposed on Black Bay Peninsula and the Lake Superior Archipelago in
northern Lake Superior. Halls (1974; doi:10.1139/e74-113) conducted the
first paleomagnetic study of these flows and determined that they were
of dominantly reversed polarity with a paleomagnetic reversal very near
to the top of the exposed stratigraphy. This reversal is associated with
the deposition of a conglomerate unit and an angular unconformity. The
data presented in Halls (1974) from the reversed polarity lavas were
from flows high in the succession in close stratigraphic proximity to a
sequence of felsic flows at Agate Point---one of which was dated by
Davis and Green (1997; doi:10.1139/e17-039) with a resulting
\(^{207}\)Pb/\(^{206}\)Pb date on zircon of 1105.3 ± 2.1 Ma.
Swanson-Hysell et al. (2014; doi:10.1002/2013GC005180) conducted a
paleomagnetic study that spanned more of the Osler Group flows from
their base up to the upper portions of the exposed reversed polarity
flows. The analysis of these data determined that there was a
significant change in direction between the data from flows in the lower
third of the reversed polarity stratigraphy and those in the upper third
stratigraphy. This change was interpretted to be caused by a progression
along the apparent polar wander path associated with equatorward motion
of Laurentia.

    \section{Import data into the
notebook}\label{import-data-into-the-notebook}

\subsection{Unpack files downloaded from the MagIC
database}\label{unpack-files-downloaded-from-the-magic-database}

Data within the MagIC database can be downloaded as a single .txt file.
This file can be split into its constituent Earthref and pmag tables
either within the QuickMagic GUI, using a command line PmagPy program or
using the function \textbf{ipmag.download\_magic} within the notebook as
is done here. The code block below uses this function by giving the file
name, specifying where it is and telling it where to unpack the data set
into the magic tables. Once these tables are unpacked, we can use them
for subsequent data analysis. The \texttt{\%\%capture} line blocks
annoying messages printed by the \textbf{ipmag.download\_magic}
function.

    \begin{Verbatim}[commandchars=\\\{\}]
{\color{incolor}In [{\color{incolor}3}]:} \PY{o}{\PYZpc{}\PYZpc{}}\PY{k}{capture}
        ipmag.download\PYZus{}magic(\PYZsq{}magic\PYZus{}contribution\PYZus{}11087.txt\PYZsq{},
                             dir\PYZus{}path=\PYZsq{}./Example\PYZus{}Data/Halls1974\PYZsq{},
                             input\PYZus{}dir\PYZus{}path=\PYZsq{}./Example\PYZus{}Data/Halls1974\PYZsq{},
                             overwrite=True,print\PYZus{}progress=False)
        ipmag.download\PYZus{}magic(\PYZsq{}magic\PYZus{}contribution\PYZus{}11088.txt\PYZsq{},
                             dir\PYZus{}path=\PYZsq{}./Example\PYZus{}Data/Swanson\PYZhy{}Hysell2014\PYZsq{},
                             input\PYZus{}dir\PYZus{}path=\PYZsq{}./Example\PYZus{}Data/Swanson\PYZhy{}Hysell2014\PYZsq{},
                             overwrite=True,print\PYZus{}progress=False)
\end{Verbatim}

            \begin{Verbatim}[commandchars=\\\{\}]
{\color{outcolor}Out[{\color{outcolor}3}]:} True
\end{Verbatim}
        
    \subsection{Loading the unpacked data into Pandas
Dataframes}\label{loading-the-unpacked-data-into-pandas-dataframes}

With the results unpacked from MagIC, the data can now be analyzed
within the notebook. The data from the Halls1974/pmag\_sites.txt table
can be imported in order to look at the directions from each site. A
nice way to deal with data within Python is using the dataframe
structure of the pandas package. The code block below uses the
pd.read\_csv function to create a dataframe from the pmag\_sites.txt
file. All MagIC formatted files are tab delimited (sep=`\t') with two
header lines, one with the type of delimiter and MagIC table and one
with the MagIC database column names. To see the formats for MagIC
tables, go to http://earthref.org/MAGIC/metadata.htm. To skip the first
header line in a particular file, set skiprows=1. The
\texttt{data\_frame\_name.head(X)} function allows us to inspect the
first \texttt{X} rows of the dataframe (in this case 4).

    \begin{Verbatim}[commandchars=\\\{\}]
{\color{incolor}In [{\color{incolor}4}]:} \PY{n}{Halls1974\PYZus{}sites} \PY{o}{=} \PY{n}{pd}\PY{o}{.}\PY{n}{read\PYZus{}csv}\PY{p}{(}\PY{l+s}{\PYZsq{}}\PY{l+s}{./Example\PYZus{}Data/Halls1974/pmag\PYZus{}sites.txt}\PY{l+s}{\PYZsq{}}\PY{p}{,}
                                      \PY{n}{sep}\PY{o}{=}\PY{l+s}{\PYZsq{}}\PY{l+s+se}{\PYZbs{}t}\PY{l+s}{\PYZsq{}}\PY{p}{,}\PY{n}{skiprows}\PY{o}{=}\PY{l+m+mi}{1}\PY{p}{)}
        \PY{n}{Halls1974\PYZus{}sites}\PY{o}{.}\PY{n}{head}\PY{p}{(}\PY{l+m+mi}{4}\PY{p}{)}
\end{Verbatim}

            \begin{Verbatim}[commandchars=\\\{\}]
{\color{outcolor}Out[{\color{outcolor}4}]:}   er\_citation\_names                               er\_location\_name  \textbackslash{}
        0        This study  Osler Volcanics, Nipigon Strait, Upper Normal   
        1        This study  Osler Volcanics, Nipigon Strait, Upper Normal   
        2        This study  Osler Volcanics, Nipigon Strait, Upper Normal   
        3        This study  Osler Volcanics, Nipigon Strait, Upper Normal   
        
           er\_site\_name            magic\_method\_codes  site\_dec  site\_inc  site\_k  \textbackslash{}
        0             1  DE-K:FS-FD:FS-H:LP-DC2:SO-SM     289.8      43.6     517   
        1             1  DE-K:FS-FD:FS-H:LP-DC2:SO-SM     293.1      34.5     517   
        2             2  DE-K:FS-FD:FS-H:LP-DC2:SO-SM     285.7      42.0     243   
        3             2  DE-K:FS-FD:FS-H:LP-DC2:SO-SM     290.6      31.9     243   
        
           site\_n site\_polarity  site\_tilt\_correction  
        0       5             n                   100  
        1       5             n                     0  
        2       5             n                   100  
        3       5             n                     0  
\end{Verbatim}
        
    \subsection{Filtering by polarity and
tilt-correction}\label{filtering-by-polarity-and-tilt-correction}

Let's start our analysis on the reversed polarity sites that are below
an angular unconformity in the Halls (1974) data set. A nice thing about
dataframes is that there is built-in functionality to filter them based
on column values. The code block below creates a new dataframe of sites
that have the site\_polarity value of `r' in the pmag\_sites table. It
then creates one dataframe for tilt-corrected sites (value of 100 in
site\_tilt\_correction) and sites that have not been tilt-corrected
(value of 0 in site\_tilt\_correction).

    \begin{Verbatim}[commandchars=\\\{\}]
{\color{incolor}In [{\color{incolor}5}]:} \PY{n}{Halls1974\PYZus{}sites\PYZus{}r} \PY{o}{=} \PY{n}{Halls1974\PYZus{}sites}\PY{o}{.}\PY{n}{ix}\PY{p}{[}\PY{n}{Halls1974\PYZus{}sites}\PY{o}{.}\PY{n}{site\PYZus{}polarity}\PY{o}{==}\PY{l+s}{\PYZsq{}}\PY{l+s}{r}\PY{l+s}{\PYZsq{}}\PY{p}{]}
        \PY{n}{Halls1974\PYZus{}sites\PYZus{}r\PYZus{}tc} \PY{o}{=} \PY{n}{Halls1974\PYZus{}sites\PYZus{}r}\PY{o}{.}\PY{n}{ix}\PY{p}{[}\PY{n}{Halls1974\PYZus{}sites\PYZus{}r}\PY{o}{.}\PY{n}{site\PYZus{}tilt\PYZus{}correction}\PY{o}{==}\PY{l+m+mi}{100}\PY{p}{]}
        \PY{n}{Halls1974\PYZus{}sites\PYZus{}r\PYZus{}tc}\PY{o}{.}\PY{n}{reset\PYZus{}index}\PY{p}{(}\PY{n}{inplace}\PY{o}{=}\PY{n+nb+bp}{True}\PY{p}{)}
        \PY{n}{Halls1974\PYZus{}sites\PYZus{}r\PYZus{}is} \PY{o}{=} \PY{n}{Halls1974\PYZus{}sites\PYZus{}r}\PY{o}{.}\PY{n}{ix}\PY{p}{[}\PY{n}{Halls1974\PYZus{}sites\PYZus{}r}\PY{o}{.}\PY{n}{site\PYZus{}tilt\PYZus{}correction}\PY{o}{==}\PY{l+m+mi}{0}\PY{p}{]}
        \PY{n}{Halls1974\PYZus{}sites\PYZus{}r\PYZus{}is}\PY{o}{.}\PY{n}{reset\PYZus{}index}\PY{p}{(}\PY{n}{inplace}\PY{o}{=}\PY{n+nb+bp}{True}\PY{p}{)}
\end{Verbatim}

    \section{Data analysis and
visualization}\label{data-analysis-and-visualization}

The data can be analyzed and visualized using PmagPy functions. We will
first create a directory called `Example\_Notebook\_Output' from within
the notebook (using the initial ! before the command invokes a command
as if typed on the command line from within a notebook) so that we have
a place to put saved figures. You can do this outside of the notebook if
you prefer.

    \begin{Verbatim}[commandchars=\\\{\}]
{\color{incolor}In [{\color{incolor}6}]:} \PY{o}{!}mkdir \PY{l+s+s1}{\PYZsq{}Example\PYZus{}Notebook\PYZus{}Output\PYZsq{}}
\end{Verbatim}

    \begin{Verbatim}[commandchars=\\\{\}]
mkdir: Example\_Notebook\_Output: File exists
    \end{Verbatim}

    \subsection{Calculating Fisher means}\label{calculating-fisher-means}

Fisher means for the data can be calculated with the function
\textbf{ipmag.fisher\_mean}. This function takes in a list of
declination values, a list of inclination values and returns a
dictionary that gives the Fisher mean and associated statistical
parameters. This dictionary is printed out for the mean of the
tilt-corrected data within the first code block. The second code block
uses the \textbf{ipmag.print\_direction\_mean} to print out a formatted
version of these same results.

    \begin{Verbatim}[commandchars=\\\{\}]
{\color{incolor}In [{\color{incolor}7}]:} \PY{n}{Halls1974\PYZus{}r\PYZus{}is\PYZus{}mean} \PY{o}{=} \PY{n}{ipmag}\PY{o}{.}\PY{n}{fisher\PYZus{}mean}\PY{p}{(}\PY{n}{Halls1974\PYZus{}sites\PYZus{}r\PYZus{}is}\PY{o}{.}\PY{n}{site\PYZus{}dec}\PY{o}{.}\PY{n}{tolist}\PY{p}{(}\PY{p}{)}\PY{p}{,}
                                                \PY{n}{Halls1974\PYZus{}sites\PYZus{}r\PYZus{}is}\PY{o}{.}\PY{n}{site\PYZus{}inc}\PY{o}{.}\PY{n}{tolist}\PY{p}{(}\PY{p}{)}\PY{p}{)}
        \PY{n}{Halls1974\PYZus{}r\PYZus{}tc\PYZus{}mean} \PY{o}{=} \PY{n}{ipmag}\PY{o}{.}\PY{n}{fisher\PYZus{}mean}\PY{p}{(}\PY{n}{Halls1974\PYZus{}sites\PYZus{}r\PYZus{}tc}\PY{o}{.}\PY{n}{site\PYZus{}dec}\PY{o}{.}\PY{n}{tolist}\PY{p}{(}\PY{p}{)}\PY{p}{,}
                                                \PY{n}{Halls1974\PYZus{}sites\PYZus{}r\PYZus{}tc}\PY{o}{.}\PY{n}{site\PYZus{}inc}\PY{o}{.}\PY{n}{tolist}\PY{p}{(}\PY{p}{)}\PY{p}{)}
        \PY{k}{print} \PY{n}{Halls1974\PYZus{}r\PYZus{}tc\PYZus{}mean}
\end{Verbatim}

    \begin{Verbatim}[commandchars=\\\{\}]
\{'csd': 12.780983354077163, 'k': 40.164419949836713, 'n': 25, 'r': 24.402456203028084, 'alpha95': 4.6244793676899487, 'dec': 114.96744501761694, 'inc': -57.572956213862817\}
    \end{Verbatim}

    \begin{Verbatim}[commandchars=\\\{\}]
{\color{incolor}In [{\color{incolor}8}]:} \PY{k}{print} \PY{l+s}{\PYZsq{}}\PY{l+s}{The mean for the tilt\PYZhy{}corrected Halls (1974) Osler directions is:}\PY{l+s}{\PYZsq{}}
        \PY{n}{ipmag}\PY{o}{.}\PY{n}{print\PYZus{}direction\PYZus{}mean}\PY{p}{(}\PY{n}{Halls1974\PYZus{}r\PYZus{}tc\PYZus{}mean}\PY{p}{)}
\end{Verbatim}

    \begin{Verbatim}[commandchars=\\\{\}]
The mean for the tilt-corrected Halls (1974) Osler directions is:
Dec: 115.0  Inc: -57.6
Number of directions in mean (n): 25
Angular radius of 95\% confidence (a\_95): 4.6
Precision parameter (k) estimate: 40.2
    \end{Verbatim}

    \subsection{Plotting the Halls (1974)
results}\label{plotting-the-halls-1974-results}

The code block below creates a figure with an equal area stereonet and
uses the \textbf{ipmag.plot\_di} function to plot the data both in
tilt-corrected and \emph{in situ} coordinates.

    \begin{Verbatim}[commandchars=\\\{\}]
{\color{incolor}In [{\color{incolor}9}]:} \PY{n}{plt}\PY{o}{.}\PY{n}{figure}\PY{p}{(}\PY{n}{num}\PY{o}{=}\PY{l+m+mi}{1}\PY{p}{,}\PY{n}{figsize}\PY{o}{=}\PY{p}{(}\PY{l+m+mi}{5}\PY{p}{,}\PY{l+m+mi}{5}\PY{p}{)}\PY{p}{)}
        \PY{n}{ipmag}\PY{o}{.}\PY{n}{plot\PYZus{}net}\PY{p}{(}\PY{n}{fignum}\PY{o}{=}\PY{l+m+mi}{1}\PY{p}{)}
        \PY{n}{ipmag}\PY{o}{.}\PY{n}{plot\PYZus{}di}\PY{p}{(}\PY{n}{Halls1974\PYZus{}sites\PYZus{}r\PYZus{}is}\PY{o}{.}\PY{n}{site\PYZus{}dec}\PY{o}{.}\PY{n}{tolist}\PY{p}{(}\PY{p}{)}\PY{p}{,}
                      \PY{n}{Halls1974\PYZus{}sites\PYZus{}r\PYZus{}is}\PY{o}{.}\PY{n}{site\PYZus{}inc}\PY{o}{.}\PY{n}{tolist}\PY{p}{(}\PY{p}{)}\PY{p}{,}\PY{n}{color}\PY{o}{=}\PY{l+s}{\PYZsq{}}\PY{l+s}{r}\PY{l+s}{\PYZsq{}}\PY{p}{,}
                      \PY{n}{label}\PY{o}{=}\PY{l+s}{\PYZsq{}}\PY{l+s}{Halls (1974) site means (in situ)}\PY{l+s}{\PYZsq{}}\PY{p}{)}
        \PY{n}{ipmag}\PY{o}{.}\PY{n}{plot\PYZus{}di}\PY{p}{(}\PY{n}{Halls1974\PYZus{}sites\PYZus{}r\PYZus{}tc}\PY{o}{.}\PY{n}{site\PYZus{}dec}\PY{o}{.}\PY{n}{tolist}\PY{p}{(}\PY{p}{)}\PY{p}{,}
                      \PY{n}{Halls1974\PYZus{}sites\PYZus{}r\PYZus{}tc}\PY{o}{.}\PY{n}{site\PYZus{}inc}\PY{o}{.}\PY{n}{tolist}\PY{p}{(}\PY{p}{)}\PY{p}{,}\PY{n}{color}\PY{o}{=}\PY{l+s}{\PYZsq{}}\PY{l+s}{b}\PY{l+s}{\PYZsq{}}\PY{p}{,}
                     \PY{n}{label}\PY{o}{=}\PY{l+s}{\PYZsq{}}\PY{l+s}{Halls (1974) site means (tilt\PYZhy{}corrected)}\PY{l+s}{\PYZsq{}}\PY{p}{)}
        \PY{n}{plt}\PY{o}{.}\PY{n}{legend}\PY{p}{(}\PY{n}{loc}\PY{o}{=}\PY{l+m+mi}{9}\PY{p}{)}
        \PY{n}{plt}\PY{o}{.}\PY{n}{savefig}\PY{p}{(}\PY{l+s}{\PYZsq{}}\PY{l+s}{Example\PYZus{}Notebook\PYZus{}Output/Halls\PYZus{}1974\PYZus{}sites.svg}\PY{l+s}{\PYZsq{}}\PY{p}{)}
\end{Verbatim}

    \begin{center}
    \adjustimage{max size={0.9\linewidth}{0.9\paperheight}}{Example_PmagPy_Notebook_files/Example_PmagPy_Notebook_20_0.pdf}
    \end{center}
    { \hspace*{\fill} \\}
    
    A similar plot showing the Fisher means of the site means calculated
above and their associated \(\alpha_{95}\) confidence ellipses can be
generated using the \textbf{ipmag.plot\_di\_mean} function. Both of
these figures can be saved out of the notebook using the
\texttt{plt.savefig()} function. The saved figure file type can be .png,
.eps, .svg among others. Saved figures can be used as is for publication
or, if necessary, exported vector graphics (e.g. .eps and .svg files)
can be editted with software such as Adobe Illustrator or Inkscape.

    \begin{Verbatim}[commandchars=\\\{\}]
{\color{incolor}In [{\color{incolor}10}]:} \PY{n}{plt}\PY{o}{.}\PY{n}{figure}\PY{p}{(}\PY{n}{num}\PY{o}{=}\PY{l+m+mi}{1}\PY{p}{,}\PY{n}{figsize}\PY{o}{=}\PY{p}{(}\PY{l+m+mi}{5}\PY{p}{,}\PY{l+m+mi}{5}\PY{p}{)}\PY{p}{)}
         \PY{n}{ipmag}\PY{o}{.}\PY{n}{plot\PYZus{}net}\PY{p}{(}\PY{n}{fignum}\PY{o}{=}\PY{l+m+mi}{1}\PY{p}{)}
         \PY{n}{ipmag}\PY{o}{.}\PY{n}{plot\PYZus{}di\PYZus{}mean}\PY{p}{(}\PY{n}{Halls1974\PYZus{}r\PYZus{}tc\PYZus{}mean}\PY{p}{[}\PY{l+s}{\PYZsq{}}\PY{l+s}{dec}\PY{l+s}{\PYZsq{}}\PY{p}{]}\PY{p}{,}
                            \PY{n}{Halls1974\PYZus{}r\PYZus{}tc\PYZus{}mean}\PY{p}{[}\PY{l+s}{\PYZsq{}}\PY{l+s}{inc}\PY{l+s}{\PYZsq{}}\PY{p}{]}\PY{p}{,}
                            \PY{n}{Halls1974\PYZus{}r\PYZus{}tc\PYZus{}mean}\PY{p}{[}\PY{l+s}{\PYZsq{}}\PY{l+s}{alpha95}\PY{l+s}{\PYZsq{}}\PY{p}{]}\PY{p}{,}\PY{l+s}{\PYZsq{}}\PY{l+s}{b}\PY{l+s}{\PYZsq{}}\PY{p}{,}
                            \PY{n}{label}\PY{o}{=}\PY{l+s}{\PYZsq{}}\PY{l+s}{Halls (1974) tilt\PYZhy{}corrected mean}\PY{l+s}{\PYZsq{}}\PY{p}{)}
         \PY{n}{ipmag}\PY{o}{.}\PY{n}{plot\PYZus{}di\PYZus{}mean}\PY{p}{(}\PY{n}{Halls1974\PYZus{}r\PYZus{}is\PYZus{}mean}\PY{p}{[}\PY{l+s}{\PYZsq{}}\PY{l+s}{dec}\PY{l+s}{\PYZsq{}}\PY{p}{]}\PY{p}{,}
                            \PY{n}{Halls1974\PYZus{}r\PYZus{}is\PYZus{}mean}\PY{p}{[}\PY{l+s}{\PYZsq{}}\PY{l+s}{inc}\PY{l+s}{\PYZsq{}}\PY{p}{]}\PY{p}{,}
                            \PY{n}{Halls1974\PYZus{}r\PYZus{}is\PYZus{}mean}\PY{p}{[}\PY{l+s}{\PYZsq{}}\PY{l+s}{alpha95}\PY{l+s}{\PYZsq{}}\PY{p}{]}\PY{p}{,}\PY{l+s}{\PYZsq{}}\PY{l+s}{r}\PY{l+s}{\PYZsq{}}\PY{p}{,}
                            \PY{n}{label}\PY{o}{=}\PY{l+s}{\PYZsq{}}\PY{l+s}{Halls (1974) insitu mean}\PY{l+s}{\PYZsq{}}\PY{p}{)}
         \PY{n}{plt}\PY{o}{.}\PY{n}{legend}\PY{p}{(}\PY{n}{loc}\PY{o}{=}\PY{l+m+mi}{9}\PY{p}{)}
         \PY{n}{plt}\PY{o}{.}\PY{n}{savefig}\PY{p}{(}\PY{l+s}{\PYZsq{}}\PY{l+s}{Example\PYZus{}Notebook\PYZus{}Output/Halls\PYZus{}1974\PYZus{}means.svg}\PY{l+s}{\PYZsq{}}\PY{p}{)}
\end{Verbatim}

    \begin{center}
    \adjustimage{max size={0.9\linewidth}{0.9\paperheight}}{Example_PmagPy_Notebook_files/Example_PmagPy_Notebook_22_0.pdf}
    \end{center}
    { \hspace*{\fill} \\}
    
    The means that have been calculated are now dictionaries that can be
made into a new dataframe to present the results. A table like this can
be exported into a variety of formats (e.g.~LaTEX, html, csv) for
inclusion in a publication.

    \begin{Verbatim}[commandchars=\\\{\}]
{\color{incolor}In [{\color{incolor}11}]:} \PY{n}{Halls1974\PYZus{}r\PYZus{}is\PYZus{}mean} \PY{o}{=} \PY{n}{ipmag}\PY{o}{.}\PY{n}{fisher\PYZus{}mean}\PY{p}{(}\PY{n}{Halls1974\PYZus{}sites\PYZus{}r\PYZus{}is}\PY{o}{.}\PY{n}{site\PYZus{}dec}\PY{o}{.}\PY{n}{tolist}\PY{p}{(}\PY{p}{)}\PY{p}{,}
                                                 \PY{n}{Halls1974\PYZus{}sites\PYZus{}r\PYZus{}is}\PY{o}{.}\PY{n}{site\PYZus{}inc}\PY{o}{.}\PY{n}{tolist}\PY{p}{(}\PY{p}{)}\PY{p}{)}
         \PY{n}{Halls1974\PYZus{}r\PYZus{}tc\PYZus{}mean} \PY{o}{=} \PY{n}{ipmag}\PY{o}{.}\PY{n}{fisher\PYZus{}mean}\PY{p}{(}\PY{n}{Halls1974\PYZus{}sites\PYZus{}r\PYZus{}tc}\PY{o}{.}\PY{n}{site\PYZus{}dec}\PY{o}{.}\PY{n}{tolist}\PY{p}{(}\PY{p}{)}\PY{p}{,}
                                                 \PY{n}{Halls1974\PYZus{}sites\PYZus{}r\PYZus{}tc}\PY{o}{.}\PY{n}{site\PYZus{}inc}\PY{o}{.}\PY{n}{tolist}\PY{p}{(}\PY{p}{)}\PY{p}{)}
         \PY{n}{means} \PY{o}{=} \PY{n}{pd}\PY{o}{.}\PY{n}{DataFrame}\PY{p}{(}\PY{p}{[}\PY{n}{Halls1974\PYZus{}r\PYZus{}is\PYZus{}mean}\PY{p}{,}\PY{n}{Halls1974\PYZus{}r\PYZus{}tc\PYZus{}mean}\PY{p}{]}\PY{p}{,}
                              \PY{n}{index}\PY{o}{=}\PY{p}{[}\PY{l+s}{\PYZsq{}}\PY{l+s}{Halls 1974 Osler R (insitu)}\PY{l+s}{\PYZsq{}}\PY{p}{,}\PY{l+s}{\PYZsq{}}\PY{l+s}{Halls 1974 Osler R (tilt\PYZhy{}corrected)}\PY{l+s}{\PYZsq{}}\PY{p}{]}\PY{p}{)}
         \PY{n}{means}
\end{Verbatim}

            \begin{Verbatim}[commandchars=\\\{\}]
{\color{outcolor}Out[{\color{outcolor}11}]:}                                       alpha95        csd         dec  \textbackslash{}
         Halls 1974 Osler R (insitu)          7.429483  20.147915  119.491601   
         Halls 1974 Osler R (tilt-corrected)  4.624479  12.780983  114.967445   
         
                                                    inc          k   n          r  
         Halls 1974 Osler R (insitu)         -37.834483  16.162548  25  23.515086  
         Halls 1974 Osler R (tilt-corrected) -57.572956  40.164420  25  24.402456  
\end{Verbatim}
        
    Alternatively, one can export the MagIC data table pmag\_results.txt
into a tab delimited or latex file using the PmagPy program:
\textbf{pmag\_results\_extract.py} which can be run at the command line
as:

\begin{verbatim}
pmag_results_extract.py -f Halls1974/pmag_results.txt
\end{verbatim}

\begin{verbatim}
pmag_results_extract.py -f Halls1974/pmag_results.txt -tex
\end{verbatim}

or executed as shell command within the notebook by using the ! prefix
as is done in the cell block below.

    \begin{Verbatim}[commandchars=\\\{\}]
{\color{incolor}In [{\color{incolor}12}]:} \PY{o}{!}pmag\PYZus{}results\PYZus{}extract.py \PYZhy{}f Example\PYZus{}Data/Halls1974/pmag\PYZus{}results.txt
\end{Verbatim}

    \begin{Verbatim}[commandchars=\\\{\}]
data saved in:  /Users/Laurentia/0000\_Github/2016\_Tauxe-et-al\_PmagPy\_Notebooks/Directions.txt /Users/Laurentia/0000\_Github/2016\_Tauxe-et-al\_PmagPy\_Notebooks/Intensities.txt /Users/Laurentia/0000\_Github/2016\_Tauxe-et-al\_PmagPy\_Notebooks/SiteNfo.txt
    \end{Verbatim}

    \subsection{Combining and plotting the Halls (1974) and Swanson-Hysell
et al. (2014)
data}\label{combining-and-plotting-the-halls-1974-and-swanson-hysell-et-al.-2014-data}

    First, let's read in the data from the pmag\_results.txt table into a
Pandas dataframe.

    \begin{Verbatim}[commandchars=\\\{\}]
{\color{incolor}In [{\color{incolor}13}]:} \PY{n}{SH2014\PYZus{}sites} \PY{o}{=} \PY{n}{pd}\PY{o}{.}\PY{n}{read\PYZus{}csv}\PY{p}{(}\PY{l+s}{\PYZsq{}}\PY{l+s}{./Example\PYZus{}Data/Swanson\PYZhy{}Hysell2014/pmag\PYZus{}results.txt}\PY{l+s}{\PYZsq{}}\PY{p}{,}
                                    \PY{n}{sep}\PY{o}{=}\PY{l+s}{\PYZsq{}}\PY{l+s+se}{\PYZbs{}t}\PY{l+s}{\PYZsq{}}\PY{p}{,}\PY{n}{skiprows}\PY{o}{=}\PY{l+m+mi}{1}\PY{p}{)}
         \PY{n}{SH2014\PYZus{}sites}\PY{o}{.}\PY{n}{head}\PY{p}{(}\PY{l+m+mi}{1}\PY{p}{)}
\end{Verbatim}

            \begin{Verbatim}[commandchars=\\\{\}]
{\color{outcolor}Out[{\color{outcolor}13}]:}    average\_age  average\_age\_high  average\_age\_low average\_age\_unit  \textbackslash{}
         0         1106              1110             1103               Ma   
         
            average\_alpha95  average\_dec  average\_height  average\_inc  average\_lat  \textbackslash{}
         0              2.7         79.7            11.8        -70.5      48.8122   
         
            average\_lon   {\ldots}    er\_specimen\_names      magic\_method\_codes  \textbackslash{}
         0      272.338   {\ldots}                  NaN  DE-FM:LP-DC4:LP-DIR-AF   
         
             pmag\_result\_name tilt\_dec\_corr tilt\_dec\_uncorr  tilt\_inc\_corr  \textbackslash{}
         0  SI1(11.8 to 26.4)          79.7           120.3          -70.5   
         
           tilt\_inc\_uncorr  vgp\_alpha95 vgp\_lat vgp\_lon  
         0           -77.1          NaN    33.1   229.6  
         
         [1 rows x 27 columns]
\end{Verbatim}
        
    Swanson-Hysell et al. (2014) argued that data from the upper third of
the Simpson Island stratigraphy should be compared with the reverse data
from the Halls (1974) Nipigon Strait region study. The dataframe can be
filtered using the average\_height value from the pmag\_results table.

    \begin{Verbatim}[commandchars=\\\{\}]
{\color{incolor}In [{\color{incolor}14}]:} \PY{n}{SH2014\PYZus{}OslerR\PYZus{}upper} \PY{o}{=} \PY{n}{SH2014\PYZus{}sites}\PY{o}{.}\PY{n}{ix}\PY{p}{[}\PY{n}{SH2014\PYZus{}sites}\PY{o}{.}\PY{n}{average\PYZus{}height}\PY{o}{\PYZgt{}}\PY{l+m+mi}{2082}\PY{p}{]}
         \PY{n}{SH2014\PYZus{}OslerR\PYZus{}upper}\PY{o}{.}\PY{n}{reset\PYZus{}index}\PY{p}{(}\PY{n}{inplace}\PY{o}{=}\PY{n+nb+bp}{True}\PY{p}{)}
         \PY{n}{SH2014\PYZus{}OslerR\PYZus{}upper}\PY{o}{.}\PY{n}{head}\PY{p}{(}\PY{p}{)}
\end{Verbatim}

            \begin{Verbatim}[commandchars=\\\{\}]
{\color{outcolor}Out[{\color{outcolor}14}]:}    index  average\_age  average\_age\_high  average\_age\_low average\_age\_unit  \textbackslash{}
         0     50         1106              1110             1103               Ma   
         1     51         1106              1110             1103               Ma   
         2     52         1106              1110             1103               Ma   
         3     53         1106              1110             1103               Ma   
         4     54         1106              1110             1103               Ma   
         
            average\_alpha95  average\_dec  average\_height  average\_inc  average\_lat  \textbackslash{}
         0              4.3        120.7          2089.0        -56.8      48.7499   
         1              4.1        111.5          2104.4        -50.9      48.7494   
         2              4.7        109.6          2116.8        -60.6      48.7490   
         3              4.0        151.1          2143.4        -62.4      48.7485   
         4              2.2        126.1          2336.0        -68.9      48.7466   
         
             {\ldots}    er\_specimen\_names      magic\_method\_codes     pmag\_result\_name  \textbackslash{}
         0   {\ldots}                  NaN  DE-FM:LP-DC4:LP-DIR-AF  SI4(106.0 to 121.4)   
         1   {\ldots}                  NaN  DE-FM:LP-DC4:LP-DIR-AF  SI4(121.4 to 127.3)   
         2   {\ldots}                  NaN  DE-FM:LP-DC4:LP-DIR-AF  SI4(133.8 to 143.1)   
         3   {\ldots}                  NaN  DE-FM:LP-DC4:LP-DIR-AF  SI4(160.4 to 171.1)   
         4   {\ldots}                  NaN  DE-FM:LP-DC4:LP-DIR-AF      SI8(0.0 to 3.9)   
         
            tilt\_dec\_corr tilt\_dec\_uncorr tilt\_inc\_corr  tilt\_inc\_uncorr vgp\_alpha95  \textbackslash{}
         0          120.7           140.0         -56.8            -46.6         NaN   
         1          111.5           129.6         -50.9            -43.6         NaN   
         2          109.6           134.8         -60.6            -52.7         NaN   
         3          151.1           163.7         -62.4            -46.6         NaN   
         4          126.1           152.4         -68.9            -56.4         NaN   
         
            vgp\_lat vgp\_lon  
         0     46.4   190.3  
         1     36.9   190.4  
         2     41.7   201.8  
         3     69.5   179.1  
         4     56.4   209.4  
         
         [5 rows x 28 columns]
\end{Verbatim}
        
    Let's fish out the declinations and inclinations in geographic (\emph{in
situ}) coordinates (is) and those in tilt-corrected coordinates (tc)
from the pmag\_results table from the Swanson-Hysell2014 dataset and
convert them from a dataframe object to a python list object. We can see
what happened with a print command.

    \begin{Verbatim}[commandchars=\\\{\}]
{\color{incolor}In [{\color{incolor}15}]:} \PY{n}{SH2014\PYZus{}upperR\PYZus{}dec\PYZus{}is} \PY{o}{=} \PY{n}{SH2014\PYZus{}OslerR\PYZus{}upper}\PY{p}{[}\PY{l+s}{\PYZsq{}}\PY{l+s}{tilt\PYZus{}dec\PYZus{}uncorr}\PY{l+s}{\PYZsq{}}\PY{p}{]}\PY{o}{.}\PY{n}{tolist}\PY{p}{(}\PY{p}{)}
         \PY{n}{SH2014\PYZus{}upperR\PYZus{}inc\PYZus{}is} \PY{o}{=} \PY{n}{SH2014\PYZus{}OslerR\PYZus{}upper}\PY{p}{[}\PY{l+s}{\PYZsq{}}\PY{l+s}{tilt\PYZus{}inc\PYZus{}uncorr}\PY{l+s}{\PYZsq{}}\PY{p}{]}\PY{o}{.}\PY{n}{tolist}\PY{p}{(}\PY{p}{)}
         \PY{n}{SH2014\PYZus{}upperR\PYZus{}dec\PYZus{}tc} \PY{o}{=} \PY{n}{SH2014\PYZus{}OslerR\PYZus{}upper}\PY{p}{[}\PY{l+s}{\PYZsq{}}\PY{l+s}{tilt\PYZus{}dec\PYZus{}corr}\PY{l+s}{\PYZsq{}}\PY{p}{]}\PY{o}{.}\PY{n}{tolist}\PY{p}{(}\PY{p}{)}
         \PY{n}{SH2014\PYZus{}upperR\PYZus{}inc\PYZus{}tc} \PY{o}{=} \PY{n}{SH2014\PYZus{}OslerR\PYZus{}upper}\PY{p}{[}\PY{l+s}{\PYZsq{}}\PY{l+s}{tilt\PYZus{}inc\PYZus{}corr}\PY{l+s}{\PYZsq{}}\PY{p}{]}\PY{o}{.}\PY{n}{tolist}\PY{p}{(}\PY{p}{)}
         \PY{k}{print} \PY{n}{SH2014\PYZus{}upperR\PYZus{}inc\PYZus{}tc}
\end{Verbatim}

    \begin{Verbatim}[commandchars=\\\{\}]
[-56.799999999999997, -50.899999999999999, -60.600000000000001, -62.399999999999999, -68.900000000000006, -71.599999999999994, -59.100000000000001, -64.400000000000006, -69.599999999999994, -66.0, -65.200000000000003, -63.299999999999997, -63.0, -65.299999999999997, -72.200000000000003, -78.599999999999994, -79.5, -53.299999999999997, -52.0, -55.299999999999997, -66.299999999999997, -61.0, -48.399999999999999, -53.600000000000001, -55.600000000000001, -57.100000000000001, -42.5, -50.399999999999999, -49.399999999999999, -56.0, -60.899999999999999, -61.0, -70.0, -52.5]
    \end{Verbatim}

    And now the same for the Halls1974 data table.

    \begin{Verbatim}[commandchars=\\\{\}]
{\color{incolor}In [{\color{incolor}16}]:} \PY{n}{Halls1974\PYZus{}upperR\PYZus{}dec\PYZus{}is} \PY{o}{=} \PY{n}{Halls1974\PYZus{}sites\PYZus{}r\PYZus{}is}\PY{p}{[}\PY{l+s}{\PYZsq{}}\PY{l+s}{site\PYZus{}dec}\PY{l+s}{\PYZsq{}}\PY{p}{]}\PY{o}{.}\PY{n}{tolist}\PY{p}{(}\PY{p}{)}
         \PY{n}{Halls1974\PYZus{}upperR\PYZus{}inc\PYZus{}is} \PY{o}{=} \PY{n}{Halls1974\PYZus{}sites\PYZus{}r\PYZus{}is}\PY{p}{[}\PY{l+s}{\PYZsq{}}\PY{l+s}{site\PYZus{}inc}\PY{l+s}{\PYZsq{}}\PY{p}{]}\PY{o}{.}\PY{n}{tolist}\PY{p}{(}\PY{p}{)}
         \PY{n}{Halls1974\PYZus{}upperR\PYZus{}dec\PYZus{}tc} \PY{o}{=} \PY{n}{Halls1974\PYZus{}sites\PYZus{}r\PYZus{}tc}\PY{p}{[}\PY{l+s}{\PYZsq{}}\PY{l+s}{site\PYZus{}dec}\PY{l+s}{\PYZsq{}}\PY{p}{]}\PY{o}{.}\PY{n}{tolist}\PY{p}{(}\PY{p}{)}
         \PY{n}{Halls1974\PYZus{}upperR\PYZus{}inc\PYZus{}tc} \PY{o}{=} \PY{n}{Halls1974\PYZus{}sites\PYZus{}r\PYZus{}tc}\PY{p}{[}\PY{l+s}{\PYZsq{}}\PY{l+s}{site\PYZus{}inc}\PY{l+s}{\PYZsq{}}\PY{p}{]}\PY{o}{.}\PY{n}{tolist}\PY{p}{(}\PY{p}{)}
\end{Verbatim}

    We can combine the data from the two papers using the \textbf{numpy}
(np) concatenate function:

    \begin{Verbatim}[commandchars=\\\{\}]
{\color{incolor}In [{\color{incolor}17}]:} \PY{n}{combined\PYZus{}upperR\PYZus{}dec\PYZus{}is} \PY{o}{=} \PY{n}{np}\PY{o}{.}\PY{n}{concatenate}\PY{p}{(}\PY{p}{(}\PY{n}{SH2014\PYZus{}upperR\PYZus{}dec\PYZus{}is}\PY{p}{,}
                                                 \PY{n}{Halls1974\PYZus{}upperR\PYZus{}dec\PYZus{}is}\PY{p}{)}\PY{p}{,} \PY{n}{axis}\PY{o}{=}\PY{l+m+mi}{0}\PY{p}{)}
         \PY{n}{combined\PYZus{}upperR\PYZus{}inc\PYZus{}is} \PY{o}{=} \PY{n}{np}\PY{o}{.}\PY{n}{concatenate}\PY{p}{(}\PY{p}{(}\PY{n}{SH2014\PYZus{}upperR\PYZus{}inc\PYZus{}is}\PY{p}{,}
                                                 \PY{n}{Halls1974\PYZus{}upperR\PYZus{}inc\PYZus{}is}\PY{p}{)}\PY{p}{,} \PY{n}{axis}\PY{o}{=}\PY{l+m+mi}{0}\PY{p}{)}
         
         \PY{n}{combined\PYZus{}upperR\PYZus{}dec\PYZus{}tc} \PY{o}{=} \PY{n}{np}\PY{o}{.}\PY{n}{concatenate}\PY{p}{(}\PY{p}{(}\PY{n}{SH2014\PYZus{}upperR\PYZus{}dec\PYZus{}tc}\PY{p}{,}
                                                 \PY{n}{Halls1974\PYZus{}upperR\PYZus{}dec\PYZus{}tc}\PY{p}{)}\PY{p}{,} \PY{n}{axis}\PY{o}{=}\PY{l+m+mi}{0}\PY{p}{)}
         \PY{n}{combined\PYZus{}upperR\PYZus{}inc\PYZus{}tc} \PY{o}{=} \PY{n}{np}\PY{o}{.}\PY{n}{concatenate}\PY{p}{(}\PY{p}{(}\PY{n}{SH2014\PYZus{}upperR\PYZus{}inc\PYZus{}tc}\PY{p}{,}
                                                 \PY{n}{Halls1974\PYZus{}upperR\PYZus{}inc\PYZus{}tc}\PY{p}{)}\PY{p}{,} \PY{n}{axis}\PY{o}{=}\PY{l+m+mi}{0}\PY{p}{)}
\end{Verbatim}

    Now we can plot the data!

    \begin{Verbatim}[commandchars=\\\{\}]
{\color{incolor}In [{\color{incolor}18}]:} \PY{n}{plt}\PY{o}{.}\PY{n}{figure}\PY{p}{(}\PY{n}{num}\PY{o}{=}\PY{l+m+mi}{1}\PY{p}{,}\PY{n}{figsize}\PY{o}{=}\PY{p}{(}\PY{l+m+mi}{5}\PY{p}{,}\PY{l+m+mi}{5}\PY{p}{)}\PY{p}{)}
         \PY{n}{ipmag}\PY{o}{.}\PY{n}{plot\PYZus{}net}\PY{p}{(}\PY{n}{fignum}\PY{o}{=}\PY{l+m+mi}{1}\PY{p}{)}
         \PY{n}{ipmag}\PY{o}{.}\PY{n}{plot\PYZus{}di}\PY{p}{(}\PY{n}{combined\PYZus{}upperR\PYZus{}dec\PYZus{}is}\PY{p}{,}
                       \PY{n}{combined\PYZus{}upperR\PYZus{}inc\PYZus{}is}\PY{p}{,}\PY{n}{color}\PY{o}{=}\PY{l+s}{\PYZsq{}}\PY{l+s}{r}\PY{l+s}{\PYZsq{}}\PY{p}{,} \PY{n}{label}\PY{o}{=}\PY{l+s}{\PYZsq{}}\PY{l+s}{insitu directions}\PY{l+s}{\PYZsq{}}\PY{p}{)}
         \PY{n}{ipmag}\PY{o}{.}\PY{n}{plot\PYZus{}di}\PY{p}{(}\PY{n}{combined\PYZus{}upperR\PYZus{}dec\PYZus{}tc}\PY{p}{,}
                       \PY{n}{combined\PYZus{}upperR\PYZus{}inc\PYZus{}tc}\PY{p}{,}\PY{n}{color}\PY{o}{=}\PY{l+s}{\PYZsq{}}\PY{l+s}{b}\PY{l+s}{\PYZsq{}}\PY{p}{,} \PY{n}{label}\PY{o}{=}\PY{l+s}{\PYZsq{}}\PY{l+s}{tilt\PYZhy{}corrected directions}\PY{l+s}{\PYZsq{}}\PY{p}{)}
         \PY{n}{plt}\PY{o}{.}\PY{n}{legend}\PY{p}{(}\PY{p}{)}
         \PY{n}{plt}\PY{o}{.}\PY{n}{show}\PY{p}{(}\PY{p}{)}
\end{Verbatim}

    \begin{center}
    \adjustimage{max size={0.9\linewidth}{0.9\paperheight}}{Example_PmagPy_Notebook_files/Example_PmagPy_Notebook_39_0.pdf}
    \end{center}
    { \hspace*{\fill} \\}
    
    And print out some useful summary information.

    \begin{Verbatim}[commandchars=\\\{\}]
{\color{incolor}In [{\color{incolor}19}]:} \PY{n}{OslerUpper\PYZus{}is\PYZus{}mean} \PY{o}{=} \PY{n}{ipmag}\PY{o}{.}\PY{n}{fisher\PYZus{}mean}\PY{p}{(}\PY{n}{combined\PYZus{}upperR\PYZus{}dec\PYZus{}is}\PY{p}{,}
                                                \PY{n}{combined\PYZus{}upperR\PYZus{}inc\PYZus{}is}\PY{p}{)}
         \PY{k}{print} \PY{l+s}{\PYZdq{}}\PY{l+s}{The Fisher mean of the insitu upper Osler R directions:}\PY{l+s}{\PYZdq{}}
         \PY{n}{ipmag}\PY{o}{.}\PY{n}{print\PYZus{}direction\PYZus{}mean}\PY{p}{(}\PY{n}{OslerUpper\PYZus{}is\PYZus{}mean}\PY{p}{)}
         \PY{k}{print} \PY{l+s}{\PYZsq{}}\PY{l+s}{\PYZsq{}}
         \PY{n}{OslerUpper\PYZus{}tc\PYZus{}mean} \PY{o}{=} \PY{n}{ipmag}\PY{o}{.}\PY{n}{fisher\PYZus{}mean}\PY{p}{(}\PY{n}{combined\PYZus{}upperR\PYZus{}dec\PYZus{}tc}\PY{p}{,}
                                               \PY{n}{combined\PYZus{}upperR\PYZus{}inc\PYZus{}tc}\PY{p}{)}
         \PY{k}{print} \PY{l+s}{\PYZdq{}}\PY{l+s}{The Fisher mean of the tilt\PYZhy{}corrected upper Osler R directions:}\PY{l+s}{\PYZdq{}}
         \PY{n}{ipmag}\PY{o}{.}\PY{n}{print\PYZus{}direction\PYZus{}mean}\PY{p}{(}\PY{n}{OslerUpper\PYZus{}tc\PYZus{}mean}\PY{p}{)}
         \PY{k}{print} \PY{l+s}{\PYZsq{}}\PY{l+s}{\PYZsq{}}
         \PY{k}{print} \PY{l+s}{\PYZsq{}}\PY{l+s}{The k\PYZus{}2/k\PYZus{}1 ratio is:}\PY{l+s}{\PYZsq{}}
         \PY{k}{print} \PY{n}{OslerUpper\PYZus{}tc\PYZus{}mean}\PY{p}{[}\PY{l+s}{\PYZsq{}}\PY{l+s}{k}\PY{l+s}{\PYZsq{}}\PY{p}{]}\PY{o}{/}\PY{n}{OslerUpper\PYZus{}is\PYZus{}mean}\PY{p}{[}\PY{l+s}{\PYZsq{}}\PY{l+s}{k}\PY{l+s}{\PYZsq{}}\PY{p}{]}
\end{Verbatim}

    \begin{Verbatim}[commandchars=\\\{\}]
The Fisher mean of the insitu upper Osler R directions:
Dec: 128.3  Inc: -47.6
Number of directions in mean (n): 59
Angular radius of 95\% confidence (a\_95): 4.2
Precision parameter (k) estimate: 19.9

The Fisher mean of the tilt-corrected upper Osler R directions:
Dec: 110.9  Inc: -59.9
Number of directions in mean (n): 59
Angular radius of 95\% confidence (a\_95): 2.8
Precision parameter (k) estimate: 43.9

The k\_2/k\_1 ratio is:
2.20159915009
    \end{Verbatim}

    \subsection{Fold test on the site mean
directions}\label{fold-test-on-the-site-mean-directions}

In the above plot, the blue directions that have been corrected for
tilting have a higher precision than the red values that are uncorrected
for tilting. The ratio of the precision parameter from before and after
tilt-correction (\(k_2\)/\(k_1\)) is 2.2 (see output of code above).
Calculating this ratio provides a way to qualitatively assess whether
there is improvement in the precision of the data such that the
magnetization was likely acquired prior to tilting. This ratio was at
the heart of the McElhinny (1964;
doi:10.1111/j.1365-246X.1964.tb06300.x) fold test (in which this would
constitute of a positive test). However, that test has been shown to not
be meaningful as summarized by McFadden (1990;
doi:10.1111/j.1365-246X.1990.tb01761.x).

Halls (1974) noted that the precision increases with structural
correction in the Osler dataset, but did not report the values of a
statistical fold test. Swanson-Hysell et al. (2014) included all of
these data, but did not report the results of the fold test. Here we
conduct a Tauxe and Watson (1994; doi:10.1016/0012-821X(94)90006-X)
bootstrap fold test on these data that reveals that the tightest
grouping of vectors is acheived upon correction for bedding tilt thereby
constituting a positive fold test.

Before we can do that, we must wrangle the data into the format expected
by the foldtest program. We can start with the in situ directions from
the pmag\_results.txt files and pair them with the bedding orientations
in the er\_sites.txt files for each of the studies.

The \emph{in situ} (geographic coordinates) for the reverse sites from
Halls (1974) were read in from the pmag\_sites table into
Halls1974\_sites\_r\_is. We read the data from the Swanson-Hysell 2014
data from pmag\_results table, so the column headers are different. Both
of these data sets must be paired with the bedding information in the
er\_sites tables for each study and put into the format expected by the
function \textbf{ipmag.bootstrap\_fold\_test} which expects an array of
declination, inclination, dip direction and dip for all the sites where
the directional data are in geographic coordinations.

The first step is to make a container for the directions and
orientations. In this case, we make an empty list OslerR\_upper\_diddd.

    \begin{Verbatim}[commandchars=\\\{\}]
{\color{incolor}In [{\color{incolor}20}]:} \PY{n}{OslerR\PYZus{}upper\PYZus{}diddd}\PY{o}{=}\PY{p}{[}\PY{p}{]}
\end{Verbatim}

    Let's start with the Halls (1974) data set in Halls1974\_sites\_r\_is.
It is handier for the MagIC data tables in this exercise to have a list
of dictionaries instead of a Pandas data frame. So let's convert the
Halls 1974 filtered dataframe to a list of dictionaries called
Halls1974.

    \begin{Verbatim}[commandchars=\\\{\}]
{\color{incolor}In [{\color{incolor}21}]:} \PY{n}{Halls1974}\PY{o}{=}\PY{n}{Halls1974\PYZus{}sites\PYZus{}r\PYZus{}is}\PY{o}{.}\PY{n}{T}\PY{o}{.}\PY{n}{to\PYZus{}dict}\PY{p}{(}\PY{p}{)}\PY{o}{.}\PY{n}{values}\PY{p}{(}\PY{p}{)}
\end{Verbatim}

    Now we can read in the data from the er\_sites.txt table (with the
bedding attitudes) using the function \textbf{pmag.magic\_read} which
reads in magic tables into a list of dictionaries. Then we will step
through the records and pair each site with its bedding orientations
using a handy function \textbf{get\_dict\_item} from the \textbf{pmag}
library that will find correct site record. The function expects a list
of dictionaries (Halls1974\_sites), a key to filter on
(`er\_site\_name'), the value of the key (\textbf{pmag.magic\_read}
reads things in as strings but because the names are numbers, pandas
converted them to integers, hence the str(site{[}`er\_site\_name'{]}),
and whether the two must be equal (`T'), not equal (`F'), contain
(`has') or not contain (`not'). After finding the correct orientation
record for the site, we can put the directions and bedding orientations
together into the list \textbf{OslerR\_upper\_diddd}.

    \begin{Verbatim}[commandchars=\\\{\}]
{\color{incolor}In [{\color{incolor}22}]:} \PY{n}{Halls1974\PYZus{}sites}\PY{p}{,}\PY{n}{filetype}\PY{o}{=}\PY{n}{pmag}\PY{o}{.}\PY{n}{magic\PYZus{}read}\PY{p}{(}\PY{l+s}{\PYZsq{}}\PY{l+s}{./Example\PYZus{}Data/Halls1974/er\PYZus{}sites.txt}\PY{l+s}{\PYZsq{}}\PY{p}{)} \PY{c}{\PYZsh{} reads in the data}
         \PY{k}{for} \PY{n}{site} \PY{o+ow}{in} \PY{n}{Halls1974}\PY{p}{:}
             \PY{n}{orientations}\PY{o}{=}\PY{n}{pmag}\PY{o}{.}\PY{n}{get\PYZus{}dictitem}\PY{p}{(}\PY{n}{Halls1974\PYZus{}sites}\PY{p}{,}\PY{l+s}{\PYZsq{}}\PY{l+s}{er\PYZus{}site\PYZus{}name}\PY{l+s}{\PYZsq{}}\PY{p}{,}
                                                \PY{n+nb}{str}\PY{p}{(}\PY{n}{site}\PY{p}{[}\PY{l+s}{\PYZsq{}}\PY{l+s}{er\PYZus{}site\PYZus{}name}\PY{l+s}{\PYZsq{}}\PY{p}{]}\PY{p}{)}\PY{p}{,}\PY{l+s}{\PYZsq{}}\PY{l+s}{T}\PY{l+s}{\PYZsq{}}\PY{p}{)}
             \PY{k}{if} \PY{n+nb}{len}\PY{p}{(}\PY{n}{orientations}\PY{p}{)}\PY{o}{\PYZgt{}}\PY{l+m+mi}{0}\PY{p}{:} \PY{c}{\PYZsh{} record found}
                 \PY{n}{OslerR\PYZus{}upper\PYZus{}diddd}\PY{o}{.}\PY{n}{append}\PY{p}{(}\PY{p}{[}\PY{n}{site}\PY{p}{[}\PY{l+s}{\PYZsq{}}\PY{l+s}{site\PYZus{}dec}\PY{l+s}{\PYZsq{}}\PY{p}{]}\PY{p}{,}\PY{n}{site}\PY{p}{[}\PY{l+s}{\PYZsq{}}\PY{l+s}{site\PYZus{}inc}\PY{l+s}{\PYZsq{}}\PY{p}{]}\PY{p}{,} 
                                 \PY{n+nb}{float}\PY{p}{(}\PY{n}{orientations}\PY{p}{[}\PY{l+m+mi}{0}\PY{p}{]}\PY{p}{[}\PY{l+s}{\PYZsq{}}\PY{l+s}{site\PYZus{}bed\PYZus{}dip\PYZus{}direction}\PY{l+s}{\PYZsq{}}\PY{p}{]}\PY{p}{)}\PY{p}{,}
                                            \PY{n+nb}{float}\PY{p}{(}\PY{n}{orientations}\PY{p}{[}\PY{l+m+mi}{0}\PY{p}{]}\PY{p}{[}\PY{l+s}{\PYZsq{}}\PY{l+s}{site\PYZus{}bed\PYZus{}dip}\PY{l+s}{\PYZsq{}}\PY{p}{]}\PY{p}{)}\PY{p}{]}\PY{p}{)}
             \PY{k}{else}\PY{p}{:}
                 \PY{k}{print} \PY{l+s}{\PYZsq{}}\PY{l+s}{no orientations found for site, }\PY{l+s}{\PYZsq{}}\PY{p}{,}\PY{n}{site}\PY{p}{[}\PY{l+s}{\PYZsq{}}\PY{l+s}{er\PYZus{}site\PYZus{}name}\PY{l+s}{\PYZsq{}}\PY{p}{]}
\end{Verbatim}

    We can do the same for the filtered upper Osler sequence of
Swanson-Hysell et al. (2014). These have slightly different keys, but
the general idea is the same. In the cell below we convert the dataframe
to a list of dictionaries, fish out the bedding orientations and attach
them to the same list as before (OslerR\_upper\_diddd).

    \begin{Verbatim}[commandchars=\\\{\}]
{\color{incolor}In [{\color{incolor}23}]:} \PY{n}{SH2014}\PY{o}{=}\PY{n}{SH2014\PYZus{}OslerR\PYZus{}upper}\PY{o}{.}\PY{n}{T}\PY{o}{.}\PY{n}{to\PYZus{}dict}\PY{p}{(}\PY{p}{)}\PY{o}{.}\PY{n}{values}\PY{p}{(}\PY{p}{)}
         \PY{n}{SH2014\PYZus{}sites}\PY{p}{,}\PY{n}{filetype}\PY{o}{=}\PY{n}{pmag}\PY{o}{.}\PY{n}{magic\PYZus{}read}\PY{p}{(}\PY{l+s}{\PYZsq{}}\PY{l+s}{./Example\PYZus{}Data/Swanson\PYZhy{}Hysell2014/er\PYZus{}sites.txt}\PY{l+s}{\PYZsq{}}\PY{p}{)}
         \PY{k}{for} \PY{n}{site} \PY{o+ow}{in} \PY{n}{SH2014}\PY{p}{:}
             \PY{n}{orientations}\PY{o}{=}\PY{n}{pmag}\PY{o}{.}\PY{n}{get\PYZus{}dictitem}\PY{p}{(}\PY{n}{SH2014\PYZus{}sites}\PY{p}{,}\PY{l+s}{\PYZsq{}}\PY{l+s}{er\PYZus{}site\PYZus{}name}\PY{l+s}{\PYZsq{}}\PY{p}{,} 
                                            \PY{n+nb}{str}\PY{p}{(}\PY{n}{site}\PY{p}{[}\PY{l+s}{\PYZsq{}}\PY{l+s}{er\PYZus{}site\PYZus{}names}\PY{l+s}{\PYZsq{}}\PY{p}{]}\PY{p}{)}\PY{p}{,}\PY{l+s}{\PYZsq{}}\PY{l+s}{T}\PY{l+s}{\PYZsq{}}\PY{p}{)}
             \PY{k}{if} \PY{n+nb}{len}\PY{p}{(}\PY{n}{orientations}\PY{p}{)}\PY{o}{\PYZgt{}}\PY{l+m+mi}{0}\PY{p}{:} \PY{c}{\PYZsh{} record found}
                 \PY{n}{OslerR\PYZus{}upper\PYZus{}diddd}\PY{o}{.}\PY{n}{append}\PY{p}{(}\PY{p}{[}\PY{n}{site}\PY{p}{[}\PY{l+s}{\PYZsq{}}\PY{l+s}{tilt\PYZus{}dec\PYZus{}uncorr}\PY{l+s}{\PYZsq{}}\PY{p}{]}\PY{p}{,}\PY{n}{site}\PY{p}{[}\PY{l+s}{\PYZsq{}}\PY{l+s}{tilt\PYZus{}inc\PYZus{}uncorr}\PY{l+s}{\PYZsq{}}\PY{p}{]}\PY{p}{,} 
                                 \PY{n+nb}{float}\PY{p}{(}\PY{n}{orientations}\PY{p}{[}\PY{l+m+mi}{0}\PY{p}{]}\PY{p}{[}\PY{l+s}{\PYZsq{}}\PY{l+s}{site\PYZus{}bed\PYZus{}dip\PYZus{}direction}\PY{l+s}{\PYZsq{}}\PY{p}{]}\PY{p}{)}\PY{p}{,}
                                            \PY{n+nb}{float}\PY{p}{(}\PY{n}{orientations}\PY{p}{[}\PY{l+m+mi}{0}\PY{p}{]}\PY{p}{[}\PY{l+s}{\PYZsq{}}\PY{l+s}{site\PYZus{}bed\PYZus{}dip}\PY{l+s}{\PYZsq{}}\PY{p}{]}\PY{p}{)}\PY{p}{]}\PY{p}{)}
             \PY{k}{else}\PY{p}{:}
                 \PY{k}{print} \PY{l+s}{\PYZsq{}}\PY{l+s}{no orientations found for site, }\PY{l+s}{\PYZsq{}}\PY{p}{,}\PY{n}{site}\PY{p}{[}\PY{l+s}{\PYZsq{}}\PY{l+s}{er\PYZus{}site\PYZus{}names}\PY{l+s}{\PYZsq{}}\PY{p}{]}
\end{Verbatim}

    Now all we have to do is make a numpy array out of the
OslerR\_upper\_diddd and send it to the
\textbf{ipmag.bootstrap\_fold\_test} function.

    \begin{Verbatim}[commandchars=\\\{\}]
{\color{incolor}In [{\color{incolor}24}]:} \PY{n}{diddd}\PY{o}{=}\PY{n}{np}\PY{o}{.}\PY{n}{array}\PY{p}{(}\PY{n}{OslerR\PYZus{}upper\PYZus{}diddd}\PY{p}{)}
         \PY{n}{ipmag}\PY{o}{.}\PY{n}{bootstrap\PYZus{}fold\PYZus{}test}\PY{p}{(}\PY{n}{diddd}\PY{p}{,}\PY{n}{num\PYZus{}sims}\PY{o}{=}\PY{l+m+mi}{100}\PY{p}{,} \PY{n}{min\PYZus{}untilt}\PY{o}{=}\PY{l+m+mi}{0}\PY{p}{,} \PY{n}{max\PYZus{}untilt}\PY{o}{=}\PY{l+m+mi}{140}\PY{p}{)}
\end{Verbatim}

    \begin{Verbatim}[commandchars=\\\{\}]
doing  100  iterations{\ldots}please be patient{\ldots}
    \end{Verbatim}

    \begin{center}
    \adjustimage{max size={0.9\linewidth}{0.9\paperheight}}{Example_PmagPy_Notebook_files/Example_PmagPy_Notebook_51_1.pdf}
    \end{center}
    { \hspace*{\fill} \\}
    
    \begin{center}
    \adjustimage{max size={0.9\linewidth}{0.9\paperheight}}{Example_PmagPy_Notebook_files/Example_PmagPy_Notebook_51_2.pdf}
    \end{center}
    { \hspace*{\fill} \\}
    
    \begin{Verbatim}[commandchars=\\\{\}]
tightest grouping of vectors obtained at (95\% confidence bounds):
98 - 135 percent unfolding
range of all bootstrap samples: 
94  -  137 percent unfolding
    \end{Verbatim}

    \begin{center}
    \adjustimage{max size={0.9\linewidth}{0.9\paperheight}}{Example_PmagPy_Notebook_files/Example_PmagPy_Notebook_51_4.pdf}
    \end{center}
    { \hspace*{\fill} \\}
    
    \section{Developing a mean paleomagnetic
pole}\label{developing-a-mean-paleomagnetic-pole}

The virtual geomagnetic poles (VGPs) calculated from the Halls (1974)
and Swanson-Hysell (2014) site means can be combined into a single
paleomagnetic pole for the upper portion of the reversed polarity Osler
Volcanic Group flows. Developing such a combined pole makes sense from a
stratigraphic perspective given that the data come from a similar
portion of the Osler Volcanic Group stratigraphy. We can test whether or
not this combination makes sense from the data themselves by posing the
question: are the data sets consistent with being drawn from a common
mean?

First, the VGP longitudes and latitudes for the Halls (1974) data can be
accessed from the pmag\_results table. The results need to be filtered
so that individual VGPs are being used (rather than mean poles) and that
the reversed poles are being used (rather than the normal ones).

    \begin{Verbatim}[commandchars=\\\{\}]
{\color{incolor}In [{\color{incolor}25}]:} \PY{n}{Halls1974\PYZus{}results} \PY{o}{=} \PY{n}{pd}\PY{o}{.}\PY{n}{read\PYZus{}csv}\PY{p}{(}\PY{l+s}{\PYZsq{}}\PY{l+s}{./Example\PYZus{}Data/Halls1974/pmag\PYZus{}results.txt}\PY{l+s}{\PYZsq{}}\PY{p}{,}\PY{n}{sep}\PY{o}{=}\PY{l+s}{\PYZsq{}}\PY{l+s+se}{\PYZbs{}t}\PY{l+s}{\PYZsq{}}\PY{p}{,}\PY{n}{skiprows}\PY{o}{=}\PY{l+m+mi}{1}\PY{p}{)}
         
         \PY{c}{\PYZsh{}filter so that individual results are shown filtering out mean poles}
         \PY{n}{Halls1974\PYZus{}results\PYZus{}i} \PY{o}{=} \PY{n}{Halls1974\PYZus{}results}\PY{o}{.}\PY{n}{ix}\PY{p}{[}\PY{n}{Halls1974\PYZus{}results}\PY{p}{[}\PY{l+s}{\PYZsq{}}\PY{l+s}{data\PYZus{}type}\PY{l+s}{\PYZsq{}}\PY{p}{]}\PY{o}{==}\PY{l+s}{\PYZsq{}}\PY{l+s}{i}\PY{l+s}{\PYZsq{}}\PY{p}{]}
         
         \PY{c}{\PYZsh{}filter so that reversed poles are included rather than normal poles}
         \PY{n}{Halls1974\PYZus{}results\PYZus{}r} \PY{o}{=} \PY{n}{Halls1974\PYZus{}results\PYZus{}i}\PY{o}{.}\PY{n}{ix}\PY{p}{[}\PY{n}{Halls1974\PYZus{}results}\PY{p}{[}\PY{l+s}{\PYZsq{}}\PY{l+s}{er\PYZus{}location\PYZus{}names}\PY{l+s}{\PYZsq{}}\PY{p}{]}\PY{o}{==}\PY{l+s}{\PYZsq{}}\PY{l+s}{Osler Volcanics, Nipigon Strait, Lower Reversed}\PY{l+s}{\PYZsq{}}\PY{p}{]}
         
         \PY{n}{Halls1974\PYZus{}results\PYZus{}r}\PY{o}{.}\PY{n}{head}\PY{p}{(}\PY{p}{)}
\end{Verbatim}

            \begin{Verbatim}[commandchars=\\\{\}]
{\color{outcolor}Out[{\color{outcolor}25}]:}    average\_alpha95  average\_dec  average\_inc  average\_k  average\_lat  \textbackslash{}
         5              NaN        106.1        -44.2        NaN    48.661051   
         6              NaN        109.8        -62.4        NaN    48.678061   
         7              NaN        140.7        -64.8        NaN    48.680014   
         8              NaN        110.0        -56.6        NaN    48.676652   
         9              NaN        128.2        -42.0        NaN    48.664307   
         
            average\_lat\_sigma  average\_lon  average\_lon\_sigma  average\_n  average\_r  \textbackslash{}
         5                NaN   271.948459                NaN          1        NaN   
         6                NaN   271.975320                NaN          1        NaN   
         7                NaN   271.973994                NaN          1        NaN   
         8                NaN   271.940536                NaN          1        NaN   
         9                NaN   271.930409                NaN          1        NaN   
         
            {\ldots}           magic\_method\_codes pmag\_result\_name  pole\_comp\_name  \textbackslash{}
         5  {\ldots}   DE-DI:FS-LOC-GOOGLE:LP-DC2     VGP : Site 6  Characteristic   
         6  {\ldots}   DE-DI:FS-LOC-GOOGLE:LP-DC2     VGP : Site 7  Characteristic   
         7  {\ldots}   DE-DI:FS-LOC-GOOGLE:LP-DC2     VGP : Site 8  Characteristic   
         8  {\ldots}   DE-DI:FS-LOC-GOOGLE:LP-DC2     VGP : Site 9  Characteristic   
         9  {\ldots}   DE-DI:FS-LOC-GOOGLE:LP-DC2    VGP : Site 10  Characteristic   
         
           reversal\_test rock\_magnetic\_test tilt\_correction vgp\_alpha95    vgp\_lat  \textbackslash{}
         5            ND                 ND             100         NaN  29.540889   
         6            ND                 ND             100         NaN  42.900204   
         7            ND                 ND             100         NaN  63.777608   
         8            ND                 ND             100         NaN  39.324957   
         9            ND                 ND             100         NaN  42.896450   
         
               vgp\_lon vgp\_n  
         5  188.653302     1  
         6  203.816622     1  
         7  192.735742     1  
         8  196.487855     1  
         9  169.933902     1  
         
         [5 rows x 27 columns]
\end{Verbatim}
        
    The data can be made into a list of {[}vgp\_lon, vgp\_lat{]} values
using the \textbf{ipmag.make\_di\_block} function.

    \begin{Verbatim}[commandchars=\\\{\}]
{\color{incolor}In [{\color{incolor}26}]:} \PY{n}{SH\PYZus{}vgps} \PY{o}{=} \PY{n}{ipmag}\PY{o}{.}\PY{n}{make\PYZus{}di\PYZus{}block}\PY{p}{(}\PY{n}{SH2014\PYZus{}OslerR\PYZus{}upper}\PY{p}{[}\PY{l+s}{\PYZsq{}}\PY{l+s}{vgp\PYZus{}lon}\PY{l+s}{\PYZsq{}}\PY{p}{]}\PY{o}{.}\PY{n}{tolist}\PY{p}{(}\PY{p}{)}\PY{p}{,}
                                       \PY{n}{SH2014\PYZus{}OslerR\PYZus{}upper}\PY{p}{[}\PY{l+s}{\PYZsq{}}\PY{l+s}{vgp\PYZus{}lat}\PY{l+s}{\PYZsq{}}\PY{p}{]}\PY{o}{.}\PY{n}{tolist}\PY{p}{(}\PY{p}{)}\PY{p}{)}
         \PY{n}{Halls\PYZus{}vgps} \PY{o}{=} \PY{n}{ipmag}\PY{o}{.}\PY{n}{make\PYZus{}di\PYZus{}block}\PY{p}{(}\PY{n}{Halls1974\PYZus{}results\PYZus{}r}\PY{p}{[}\PY{l+s}{\PYZsq{}}\PY{l+s}{vgp\PYZus{}lon}\PY{l+s}{\PYZsq{}}\PY{p}{]}\PY{o}{.}\PY{n}{tolist}\PY{p}{(}\PY{p}{)}\PY{p}{,}
                                       \PY{n}{Halls1974\PYZus{}results\PYZus{}r}\PY{p}{[}\PY{l+s}{\PYZsq{}}\PY{l+s}{vgp\PYZus{}lat}\PY{l+s}{\PYZsq{}}\PY{p}{]}\PY{o}{.}\PY{n}{tolist}\PY{p}{(}\PY{p}{)}\PY{p}{)}
\end{Verbatim}

    \subsection{Conducting a common mean
test}\label{conducting-a-common-mean-test}

The question ``are the data sets consistent with being drawn from a
common mean?'' can be addressed utilizing the
\textbf{ipmag.watson\_common\_mean} function. This function calculates
Watson's V statistic from input data through Monte Carlo simulation in
order to test whether two populations of directional data could have
been drawn from a common mean. The critical angle between the two sample
mean directions and the corresponding McFadden and McElhinny (1990)
classification is also printed. A plot can be shown of the cumulative
distribution of the Watson V statistic as calculated during the Monte
Carlo simulations, as suggested by Tauxe et al. (2010).

    \begin{Verbatim}[commandchars=\\\{\}]
{\color{incolor}In [{\color{incolor}27}]:} \PY{n}{ipmag}\PY{o}{.}\PY{n}{common\PYZus{}mean\PYZus{}watson}\PY{p}{(}\PY{n}{SH\PYZus{}vgps}\PY{p}{,}\PY{n}{Halls\PYZus{}vgps}\PY{p}{,}\PY{n}{NumSims}\PY{o}{=}\PY{l+m+mi}{1000}\PY{p}{,}\PY{n}{plot}\PY{o}{=}\PY{l+s}{\PYZsq{}}\PY{l+s}{yes}\PY{l+s}{\PYZsq{}}\PY{p}{)}
\end{Verbatim}

    \begin{Verbatim}[commandchars=\\\{\}]
Results of Watson V test: 

Watson's V:           5.4
Critical value of V:  6.1
"Pass": Since V is less than Vcrit, the null hypothesis
that the two populations are drawn from distributions
that share a common mean direction can not be rejected.

M\&M1990 classification:

Angle between data set means: 7.0
Critical angle for M\&M1990:   7.4
The McFadden and McElhinny (1990) classification for
this test is: 'B'
    \end{Verbatim}

    \begin{center}
    \adjustimage{max size={0.9\linewidth}{0.9\paperheight}}{Example_PmagPy_Notebook_files/Example_PmagPy_Notebook_57_1.pdf}
    \end{center}
    { \hspace*{\fill} \\}
    
    \subsection{Calculating a mean paleomagnetic
pole}\label{calculating-a-mean-paleomagnetic-pole}

We can go ahead and calculate a mean paleomagnetic pole combining the
VGPs from both studies. This mean pole was published in Swanson-Hysell
et al. (2014).

    \begin{Verbatim}[commandchars=\\\{\}]
{\color{incolor}In [{\color{incolor}28}]:} \PY{n}{Osler\PYZus{}upperR\PYZus{}pole} \PY{o}{=} \PY{n}{pmag}\PY{o}{.}\PY{n}{fisher\PYZus{}mean}\PY{p}{(}\PY{n}{SH\PYZus{}vgps}\PY{o}{+}\PY{n}{Halls\PYZus{}vgps}\PY{p}{)}
         \PY{n}{ipmag}\PY{o}{.}\PY{n}{print\PYZus{}pole\PYZus{}mean}\PY{p}{(}\PY{n}{Osler\PYZus{}upperR\PYZus{}pole}\PY{p}{)}
\end{Verbatim}

    \begin{Verbatim}[commandchars=\\\{\}]
Plong: 201.6  Plat: 42.5
Number of directions in mean (n): 59
Angular radius of 95\% confidence (A\_95): 3.7
Precision parameter (k) estimate: 25.7
    \end{Verbatim}

    \subsection{Plotting VGPs and the mean
pole}\label{plotting-vgps-and-the-mean-pole}

The code below uses the \textbf{ipmag.plot\_vgp} function to plot the
virtual geomagnetic poles and the \textbf{ipmag.plot\_pole} function to
plot the calculated mean pole along with its \(A_{95}\) confidence
ellipse.

The plot is developed using the \textbf{Basemap} package which enables
the plotting of data on a variety of geographic projections.
\textbf{Basemap} is not a standard part of most scientific python
distributions so you may need to take extra steps to install it. If
using the Anaconda distribution, you can type
\texttt{conda\ install\ basemap} at the command line. The Enthought
Canopy distribution has a GUI package manager that you can use for
installing the package.

The plot is saved to the Example\_Notebook\_Output folder as an .svg
file.

    \begin{Verbatim}[commandchars=\\\{\}]
{\color{incolor}In [{\color{incolor}29}]:} \PY{k+kn}{from} \PY{n+nn}{mpl\PYZus{}toolkits.basemap} \PY{k+kn}{import} \PY{n}{Basemap}
         
         \PY{n}{m} \PY{o}{=} \PY{n}{Basemap}\PY{p}{(}\PY{n}{projection}\PY{o}{=}\PY{l+s}{\PYZsq{}}\PY{l+s}{ortho}\PY{l+s}{\PYZsq{}}\PY{p}{,}\PY{n}{lat\PYZus{}0}\PY{o}{=}\PY{l+m+mi}{35}\PY{p}{,}\PY{n}{lon\PYZus{}0}\PY{o}{=}\PY{l+m+mi}{200}\PY{p}{,}\PY{n}{resolution}\PY{o}{=}\PY{l+s}{\PYZsq{}}\PY{l+s}{c}\PY{l+s}{\PYZsq{}}\PY{p}{,}\PY{n}{area\PYZus{}thresh}\PY{o}{=}\PY{l+m+mi}{50000}\PY{p}{)}
         \PY{n}{plt}\PY{o}{.}\PY{n}{figure}\PY{p}{(}\PY{n}{figsize}\PY{o}{=}\PY{p}{(}\PY{l+m+mi}{7}\PY{p}{,} \PY{l+m+mi}{7}\PY{p}{)}\PY{p}{)}
         \PY{n}{m}\PY{o}{.}\PY{n}{drawcoastlines}\PY{p}{(}\PY{n}{linewidth}\PY{o}{=}\PY{l+m+mf}{0.25}\PY{p}{)}
         \PY{n}{m}\PY{o}{.}\PY{n}{fillcontinents}\PY{p}{(}\PY{n}{color}\PY{o}{=}\PY{l+s}{\PYZsq{}}\PY{l+s}{bisque}\PY{l+s}{\PYZsq{}}\PY{p}{,}\PY{n}{zorder}\PY{o}{=}\PY{l+m+mi}{1}\PY{p}{)}
         \PY{n}{m}\PY{o}{.}\PY{n}{drawmeridians}\PY{p}{(}\PY{n}{np}\PY{o}{.}\PY{n}{arange}\PY{p}{(}\PY{l+m+mi}{0}\PY{p}{,}\PY{l+m+mi}{360}\PY{p}{,}\PY{l+m+mi}{30}\PY{p}{)}\PY{p}{)}
         \PY{n}{m}\PY{o}{.}\PY{n}{drawparallels}\PY{p}{(}\PY{n}{np}\PY{o}{.}\PY{n}{arange}\PY{p}{(}\PY{o}{\PYZhy{}}\PY{l+m+mi}{90}\PY{p}{,}\PY{l+m+mi}{90}\PY{p}{,}\PY{l+m+mi}{30}\PY{p}{)}\PY{p}{)}
         
         \PY{n}{ipmag}\PY{o}{.}\PY{n}{plot\PYZus{}vgp}\PY{p}{(}\PY{n}{m}\PY{p}{,}\PY{n}{SH2014\PYZus{}OslerR\PYZus{}upper}\PY{p}{[}\PY{l+s}{\PYZsq{}}\PY{l+s}{vgp\PYZus{}lon}\PY{l+s}{\PYZsq{}}\PY{p}{]}\PY{o}{.}\PY{n}{tolist}\PY{p}{(}\PY{p}{)}\PY{p}{,}
                        \PY{n}{SH2014\PYZus{}OslerR\PYZus{}upper}\PY{p}{[}\PY{l+s}{\PYZsq{}}\PY{l+s}{vgp\PYZus{}lat}\PY{l+s}{\PYZsq{}}\PY{p}{]}\PY{o}{.}\PY{n}{tolist}\PY{p}{(}\PY{p}{)}\PY{p}{,}
                        \PY{n}{marker}\PY{o}{=}\PY{l+s}{\PYZsq{}}\PY{l+s}{o}\PY{l+s}{\PYZsq{}}\PY{p}{)}
         
         \PY{n}{ipmag}\PY{o}{.}\PY{n}{plot\PYZus{}vgp}\PY{p}{(}\PY{n}{m}\PY{p}{,}\PY{n}{Halls1974\PYZus{}results\PYZus{}r}\PY{p}{[}\PY{l+s}{\PYZsq{}}\PY{l+s}{vgp\PYZus{}lon}\PY{l+s}{\PYZsq{}}\PY{p}{]}\PY{o}{.}\PY{n}{tolist}\PY{p}{(}\PY{p}{)}\PY{p}{,}
                        \PY{n}{Halls1974\PYZus{}results\PYZus{}r}\PY{p}{[}\PY{l+s}{\PYZsq{}}\PY{l+s}{vgp\PYZus{}lat}\PY{l+s}{\PYZsq{}}\PY{p}{]}\PY{o}{.}\PY{n}{tolist}\PY{p}{(}\PY{p}{)}\PY{p}{,}
                        \PY{n}{marker}\PY{o}{=}\PY{l+s}{\PYZsq{}}\PY{l+s}{o}\PY{l+s}{\PYZsq{}}\PY{p}{,}\PY{n}{label}\PY{o}{=}\PY{l+s}{\PYZsq{}}\PY{l+s}{Osler upper reversed VGPs}\PY{l+s}{\PYZsq{}}\PY{p}{)}
                        
         \PY{n}{ipmag}\PY{o}{.}\PY{n}{plot\PYZus{}pole}\PY{p}{(}\PY{n}{m}\PY{p}{,}\PY{n}{Osler\PYZus{}upperR\PYZus{}pole}\PY{p}{[}\PY{l+s}{\PYZsq{}}\PY{l+s}{dec}\PY{l+s}{\PYZsq{}}\PY{p}{]}\PY{p}{,}
                      \PY{n}{Osler\PYZus{}upperR\PYZus{}pole}\PY{p}{[}\PY{l+s}{\PYZsq{}}\PY{l+s}{inc}\PY{l+s}{\PYZsq{}}\PY{p}{]}\PY{p}{,}
                      \PY{n}{Osler\PYZus{}upperR\PYZus{}pole}\PY{p}{[}\PY{l+s}{\PYZsq{}}\PY{l+s}{alpha95}\PY{l+s}{\PYZsq{}}\PY{p}{]}\PY{p}{,}
                      \PY{n}{marker}\PY{o}{=}\PY{l+s}{\PYZsq{}}\PY{l+s}{s}\PY{l+s}{\PYZsq{}}\PY{p}{,}\PY{n}{label}\PY{o}{=}\PY{l+s}{\PYZsq{}}\PY{l+s}{Osler upper reversed pole}\PY{l+s}{\PYZsq{}}\PY{p}{)}
         
         \PY{n}{plt}\PY{o}{.}\PY{n}{legend}\PY{p}{(}\PY{p}{)}
         \PY{n}{plt}\PY{o}{.}\PY{n}{savefig}\PY{p}{(}\PY{l+s}{\PYZsq{}}\PY{l+s}{Example\PYZus{}Notebook\PYZus{}Output/pole\PYZus{}plot.svg}\PY{l+s}{\PYZsq{}}\PY{p}{)}
         \PY{n}{plt}\PY{o}{.}\PY{n}{show}\PY{p}{(}\PY{p}{)}
\end{Verbatim}

    \begin{center}
    \adjustimage{max size={0.9\linewidth}{0.9\paperheight}}{Example_PmagPy_Notebook_files/Example_PmagPy_Notebook_61_0.pdf}
    \end{center}
    { \hspace*{\fill} \\}
    
    \section{Concluding thoughts}\label{concluding-thoughts}

This notebook is intended to be an illustrative case study of the type
of data analysis that can be accomplished using PmagPy within a Jupyter
notebook. All the capabilities of PmagPy can be utilized within
notebooks, although continued work is needed for the functionality
within some of the command line programs to be made into functions that
work well within the environment.

An advantage of this type of workflow is that it is well-documented and
reproducible. The decisions that went into the data analysis and the
implementation of the statistical tests are fully transparent (as is the
underlying code). Additionally, if one were to seek to add more data to
the mean pole, all of the data analysis could be quickly redone by
executing all of the code in the notebook.

Smaller snippets of code that demonstrate additional PmagPy
functionality within the notebook environment can be seen in this
notebook:

http://pmagpy.github.io/Additional\_PmagPy\_Examples.html

\section{Works cited}\label{works-cited}

Davis, D., and J. Green (1997), Geochronology of the North American
Midcontinent rift in western Lake Superior and implications for its
geodynamic evolution, Can. J. Earth Sci., 34, 476--488,
doi:10.1139/e17--039.

Halls, H. (1974), A paleomagnetic reversal in the Osler Volcanic Group,
northern Lake Superior, Can. J. Earth Sci., 11, 1200--1207,
doi:10.1139/e74--113.

McElhinny, M. W. (1964), Statistical Significance of the Fold Test in
Palaeomagnetism. Geophysical Journal of the Royal Astronomical Society,
8: 338--340. doi: 10.1111/j.1365-246X.1964.tb06300.x

McFadden, P. L. and McElhinny, M. W. (1990), Classification of the
reversal test in palaeomagnetism. Geophysical Journal International,
103: 725--729. doi: 10.1111/j.1365-246X.1990.tb05683.x

McFadden, P. L. (1990), A new fold test for palaeomagnetic studies.
Geophysical Journal International, 103: 163--169. doi:
10.1111/j.1365-246X.1990.tb01761.x

Swanson-Hysell, N. L., A. A. Vaughan, M. R. Mustain, and K. E. Asp
(2014), Confirmation of progressive plate motion during the Midcontinent
Rift's early magmatic stage from the Osler Volcanic Group, Ontario,
Canada, Geochemistry Geophysics Geosystems, 15, 2039--2047,
doi:10.1002/2013GC005180.

Tauxe, L., and G. S. Watson (1994), The fold test: an eigen analysis
approach, Earth Planet. Sci. Lett., 122, 331--341,
doi:10.1016/0012-821X(94)90006-X.

Tauxe, L. with contributions from Subir K. Banerjee, Robert F. Butler
and Rob van der Voo, Essentials of Paleomagnetism, Univ. California
Press, 2010. Current online version:
https://earthref.org/MagIC/books/Tauxe/Essentials/

    \begin{Verbatim}[commandchars=\\\{\}]
{\color{incolor}In [{\color{incolor} }]:} 
\end{Verbatim}


    % Add a bibliography block to the postdoc
    
    
    
    \end{document}
