
% Default to the notebook output style

    


% Inherit from the specified cell style.




    
\documentclass{article}

    
    
    \usepackage{graphicx} % Used to insert images
    \usepackage{adjustbox} % Used to constrain images to a maximum size 
    \usepackage{color} % Allow colors to be defined
    \usepackage{enumerate} % Needed for markdown enumerations to work
    \usepackage{geometry} % Used to adjust the document margins
    \usepackage{amsmath} % Equations
    \usepackage{amssymb} % Equations
    \usepackage{eurosym} % defines \euro
    \usepackage[mathletters]{ucs} % Extended unicode (utf-8) support
    \usepackage[utf8x]{inputenc} % Allow utf-8 characters in the tex document
    \usepackage{fancyvrb} % verbatim replacement that allows latex
    \usepackage{grffile} % extends the file name processing of package graphics 
                         % to support a larger range 
    % The hyperref package gives us a pdf with properly built
    % internal navigation ('pdf bookmarks' for the table of contents,
    % internal cross-reference links, web links for URLs, etc.)
    \usepackage{hyperref}
    \usepackage{longtable} % longtable support required by pandoc >1.10
    \usepackage{booktabs}  % table support for pandoc > 1.12.2
    \usepackage{ulem} % ulem is needed to support strikethroughs (\sout)
    

    
    
    \definecolor{orange}{cmyk}{0,0.4,0.8,0.2}
    \definecolor{darkorange}{rgb}{.71,0.21,0.01}
    \definecolor{darkgreen}{rgb}{.12,.54,.11}
    \definecolor{myteal}{rgb}{.26, .44, .56}
    \definecolor{gray}{gray}{0.45}
    \definecolor{lightgray}{gray}{.95}
    \definecolor{mediumgray}{gray}{.8}
    \definecolor{inputbackground}{rgb}{.95, .95, .85}
    \definecolor{outputbackground}{rgb}{.95, .95, .95}
    \definecolor{traceback}{rgb}{1, .95, .95}
    % ansi colors
    \definecolor{red}{rgb}{.6,0,0}
    \definecolor{green}{rgb}{0,.65,0}
    \definecolor{brown}{rgb}{0.6,0.6,0}
    \definecolor{blue}{rgb}{0,.145,.698}
    \definecolor{purple}{rgb}{.698,.145,.698}
    \definecolor{cyan}{rgb}{0,.698,.698}
    \definecolor{lightgray}{gray}{0.5}
    
    % bright ansi colors
    \definecolor{darkgray}{gray}{0.25}
    \definecolor{lightred}{rgb}{1.0,0.39,0.28}
    \definecolor{lightgreen}{rgb}{0.48,0.99,0.0}
    \definecolor{lightblue}{rgb}{0.53,0.81,0.92}
    \definecolor{lightpurple}{rgb}{0.87,0.63,0.87}
    \definecolor{lightcyan}{rgb}{0.5,1.0,0.83}
    
    % commands and environments needed by pandoc snippets
    % extracted from the output of `pandoc -s`
    \providecommand{\tightlist}{%
      \setlength{\itemsep}{0pt}\setlength{\parskip}{0pt}}
    \DefineVerbatimEnvironment{Highlighting}{Verbatim}{commandchars=\\\{\}}
    % Add ',fontsize=\small' for more characters per line
    \newenvironment{Shaded}{}{}
    \newcommand{\KeywordTok}[1]{\textcolor[rgb]{0.00,0.44,0.13}{\textbf{{#1}}}}
    \newcommand{\DataTypeTok}[1]{\textcolor[rgb]{0.56,0.13,0.00}{{#1}}}
    \newcommand{\DecValTok}[1]{\textcolor[rgb]{0.25,0.63,0.44}{{#1}}}
    \newcommand{\BaseNTok}[1]{\textcolor[rgb]{0.25,0.63,0.44}{{#1}}}
    \newcommand{\FloatTok}[1]{\textcolor[rgb]{0.25,0.63,0.44}{{#1}}}
    \newcommand{\CharTok}[1]{\textcolor[rgb]{0.25,0.44,0.63}{{#1}}}
    \newcommand{\StringTok}[1]{\textcolor[rgb]{0.25,0.44,0.63}{{#1}}}
    \newcommand{\CommentTok}[1]{\textcolor[rgb]{0.38,0.63,0.69}{\textit{{#1}}}}
    \newcommand{\OtherTok}[1]{\textcolor[rgb]{0.00,0.44,0.13}{{#1}}}
    \newcommand{\AlertTok}[1]{\textcolor[rgb]{1.00,0.00,0.00}{\textbf{{#1}}}}
    \newcommand{\FunctionTok}[1]{\textcolor[rgb]{0.02,0.16,0.49}{{#1}}}
    \newcommand{\RegionMarkerTok}[1]{{#1}}
    \newcommand{\ErrorTok}[1]{\textcolor[rgb]{1.00,0.00,0.00}{\textbf{{#1}}}}
    \newcommand{\NormalTok}[1]{{#1}}
    
    % Additional commands for more recent versions of Pandoc
    \newcommand{\ConstantTok}[1]{\textcolor[rgb]{0.53,0.00,0.00}{{#1}}}
    \newcommand{\SpecialCharTok}[1]{\textcolor[rgb]{0.25,0.44,0.63}{{#1}}}
    \newcommand{\VerbatimStringTok}[1]{\textcolor[rgb]{0.25,0.44,0.63}{{#1}}}
    \newcommand{\SpecialStringTok}[1]{\textcolor[rgb]{0.73,0.40,0.53}{{#1}}}
    \newcommand{\ImportTok}[1]{{#1}}
    \newcommand{\DocumentationTok}[1]{\textcolor[rgb]{0.73,0.13,0.13}{\textit{{#1}}}}
    \newcommand{\AnnotationTok}[1]{\textcolor[rgb]{0.38,0.63,0.69}{\textbf{\textit{{#1}}}}}
    \newcommand{\CommentVarTok}[1]{\textcolor[rgb]{0.38,0.63,0.69}{\textbf{\textit{{#1}}}}}
    \newcommand{\VariableTok}[1]{\textcolor[rgb]{0.10,0.09,0.49}{{#1}}}
    \newcommand{\ControlFlowTok}[1]{\textcolor[rgb]{0.00,0.44,0.13}{\textbf{{#1}}}}
    \newcommand{\OperatorTok}[1]{\textcolor[rgb]{0.40,0.40,0.40}{{#1}}}
    \newcommand{\BuiltInTok}[1]{{#1}}
    \newcommand{\ExtensionTok}[1]{{#1}}
    \newcommand{\PreprocessorTok}[1]{\textcolor[rgb]{0.74,0.48,0.00}{{#1}}}
    \newcommand{\AttributeTok}[1]{\textcolor[rgb]{0.49,0.56,0.16}{{#1}}}
    \newcommand{\InformationTok}[1]{\textcolor[rgb]{0.38,0.63,0.69}{\textbf{\textit{{#1}}}}}
    \newcommand{\WarningTok}[1]{\textcolor[rgb]{0.38,0.63,0.69}{\textbf{\textit{{#1}}}}}
    
    
    % Define a nice break command that doesn't care if a line doesn't already
    % exist.
    \def\br{\hspace*{\fill} \\* }
    % Math Jax compatability definitions
    \def\gt{>}
    \def\lt{<}
    % Document parameters
    \title{Additional\_PmagPy\_Examples}
    
    
    

    % Pygments definitions
    
\makeatletter
\def\PY@reset{\let\PY@it=\relax \let\PY@bf=\relax%
    \let\PY@ul=\relax \let\PY@tc=\relax%
    \let\PY@bc=\relax \let\PY@ff=\relax}
\def\PY@tok#1{\csname PY@tok@#1\endcsname}
\def\PY@toks#1+{\ifx\relax#1\empty\else%
    \PY@tok{#1}\expandafter\PY@toks\fi}
\def\PY@do#1{\PY@bc{\PY@tc{\PY@ul{%
    \PY@it{\PY@bf{\PY@ff{#1}}}}}}}
\def\PY#1#2{\PY@reset\PY@toks#1+\relax+\PY@do{#2}}

\expandafter\def\csname PY@tok@gd\endcsname{\def\PY@tc##1{\textcolor[rgb]{0.63,0.00,0.00}{##1}}}
\expandafter\def\csname PY@tok@gu\endcsname{\let\PY@bf=\textbf\def\PY@tc##1{\textcolor[rgb]{0.50,0.00,0.50}{##1}}}
\expandafter\def\csname PY@tok@gt\endcsname{\def\PY@tc##1{\textcolor[rgb]{0.00,0.27,0.87}{##1}}}
\expandafter\def\csname PY@tok@gs\endcsname{\let\PY@bf=\textbf}
\expandafter\def\csname PY@tok@gr\endcsname{\def\PY@tc##1{\textcolor[rgb]{1.00,0.00,0.00}{##1}}}
\expandafter\def\csname PY@tok@cm\endcsname{\let\PY@it=\textit\def\PY@tc##1{\textcolor[rgb]{0.25,0.50,0.50}{##1}}}
\expandafter\def\csname PY@tok@vg\endcsname{\def\PY@tc##1{\textcolor[rgb]{0.10,0.09,0.49}{##1}}}
\expandafter\def\csname PY@tok@m\endcsname{\def\PY@tc##1{\textcolor[rgb]{0.40,0.40,0.40}{##1}}}
\expandafter\def\csname PY@tok@mh\endcsname{\def\PY@tc##1{\textcolor[rgb]{0.40,0.40,0.40}{##1}}}
\expandafter\def\csname PY@tok@go\endcsname{\def\PY@tc##1{\textcolor[rgb]{0.53,0.53,0.53}{##1}}}
\expandafter\def\csname PY@tok@ge\endcsname{\let\PY@it=\textit}
\expandafter\def\csname PY@tok@vc\endcsname{\def\PY@tc##1{\textcolor[rgb]{0.10,0.09,0.49}{##1}}}
\expandafter\def\csname PY@tok@il\endcsname{\def\PY@tc##1{\textcolor[rgb]{0.40,0.40,0.40}{##1}}}
\expandafter\def\csname PY@tok@cs\endcsname{\let\PY@it=\textit\def\PY@tc##1{\textcolor[rgb]{0.25,0.50,0.50}{##1}}}
\expandafter\def\csname PY@tok@cp\endcsname{\def\PY@tc##1{\textcolor[rgb]{0.74,0.48,0.00}{##1}}}
\expandafter\def\csname PY@tok@gi\endcsname{\def\PY@tc##1{\textcolor[rgb]{0.00,0.63,0.00}{##1}}}
\expandafter\def\csname PY@tok@gh\endcsname{\let\PY@bf=\textbf\def\PY@tc##1{\textcolor[rgb]{0.00,0.00,0.50}{##1}}}
\expandafter\def\csname PY@tok@ni\endcsname{\let\PY@bf=\textbf\def\PY@tc##1{\textcolor[rgb]{0.60,0.60,0.60}{##1}}}
\expandafter\def\csname PY@tok@nl\endcsname{\def\PY@tc##1{\textcolor[rgb]{0.63,0.63,0.00}{##1}}}
\expandafter\def\csname PY@tok@nn\endcsname{\let\PY@bf=\textbf\def\PY@tc##1{\textcolor[rgb]{0.00,0.00,1.00}{##1}}}
\expandafter\def\csname PY@tok@no\endcsname{\def\PY@tc##1{\textcolor[rgb]{0.53,0.00,0.00}{##1}}}
\expandafter\def\csname PY@tok@na\endcsname{\def\PY@tc##1{\textcolor[rgb]{0.49,0.56,0.16}{##1}}}
\expandafter\def\csname PY@tok@nb\endcsname{\def\PY@tc##1{\textcolor[rgb]{0.00,0.50,0.00}{##1}}}
\expandafter\def\csname PY@tok@nc\endcsname{\let\PY@bf=\textbf\def\PY@tc##1{\textcolor[rgb]{0.00,0.00,1.00}{##1}}}
\expandafter\def\csname PY@tok@nd\endcsname{\def\PY@tc##1{\textcolor[rgb]{0.67,0.13,1.00}{##1}}}
\expandafter\def\csname PY@tok@ne\endcsname{\let\PY@bf=\textbf\def\PY@tc##1{\textcolor[rgb]{0.82,0.25,0.23}{##1}}}
\expandafter\def\csname PY@tok@nf\endcsname{\def\PY@tc##1{\textcolor[rgb]{0.00,0.00,1.00}{##1}}}
\expandafter\def\csname PY@tok@si\endcsname{\let\PY@bf=\textbf\def\PY@tc##1{\textcolor[rgb]{0.73,0.40,0.53}{##1}}}
\expandafter\def\csname PY@tok@s2\endcsname{\def\PY@tc##1{\textcolor[rgb]{0.73,0.13,0.13}{##1}}}
\expandafter\def\csname PY@tok@vi\endcsname{\def\PY@tc##1{\textcolor[rgb]{0.10,0.09,0.49}{##1}}}
\expandafter\def\csname PY@tok@nt\endcsname{\let\PY@bf=\textbf\def\PY@tc##1{\textcolor[rgb]{0.00,0.50,0.00}{##1}}}
\expandafter\def\csname PY@tok@nv\endcsname{\def\PY@tc##1{\textcolor[rgb]{0.10,0.09,0.49}{##1}}}
\expandafter\def\csname PY@tok@s1\endcsname{\def\PY@tc##1{\textcolor[rgb]{0.73,0.13,0.13}{##1}}}
\expandafter\def\csname PY@tok@kd\endcsname{\let\PY@bf=\textbf\def\PY@tc##1{\textcolor[rgb]{0.00,0.50,0.00}{##1}}}
\expandafter\def\csname PY@tok@sh\endcsname{\def\PY@tc##1{\textcolor[rgb]{0.73,0.13,0.13}{##1}}}
\expandafter\def\csname PY@tok@sc\endcsname{\def\PY@tc##1{\textcolor[rgb]{0.73,0.13,0.13}{##1}}}
\expandafter\def\csname PY@tok@sx\endcsname{\def\PY@tc##1{\textcolor[rgb]{0.00,0.50,0.00}{##1}}}
\expandafter\def\csname PY@tok@bp\endcsname{\def\PY@tc##1{\textcolor[rgb]{0.00,0.50,0.00}{##1}}}
\expandafter\def\csname PY@tok@c1\endcsname{\let\PY@it=\textit\def\PY@tc##1{\textcolor[rgb]{0.25,0.50,0.50}{##1}}}
\expandafter\def\csname PY@tok@kc\endcsname{\let\PY@bf=\textbf\def\PY@tc##1{\textcolor[rgb]{0.00,0.50,0.00}{##1}}}
\expandafter\def\csname PY@tok@c\endcsname{\let\PY@it=\textit\def\PY@tc##1{\textcolor[rgb]{0.25,0.50,0.50}{##1}}}
\expandafter\def\csname PY@tok@mf\endcsname{\def\PY@tc##1{\textcolor[rgb]{0.40,0.40,0.40}{##1}}}
\expandafter\def\csname PY@tok@err\endcsname{\def\PY@bc##1{\setlength{\fboxsep}{0pt}\fcolorbox[rgb]{1.00,0.00,0.00}{1,1,1}{\strut ##1}}}
\expandafter\def\csname PY@tok@mb\endcsname{\def\PY@tc##1{\textcolor[rgb]{0.40,0.40,0.40}{##1}}}
\expandafter\def\csname PY@tok@ss\endcsname{\def\PY@tc##1{\textcolor[rgb]{0.10,0.09,0.49}{##1}}}
\expandafter\def\csname PY@tok@sr\endcsname{\def\PY@tc##1{\textcolor[rgb]{0.73,0.40,0.53}{##1}}}
\expandafter\def\csname PY@tok@mo\endcsname{\def\PY@tc##1{\textcolor[rgb]{0.40,0.40,0.40}{##1}}}
\expandafter\def\csname PY@tok@kn\endcsname{\let\PY@bf=\textbf\def\PY@tc##1{\textcolor[rgb]{0.00,0.50,0.00}{##1}}}
\expandafter\def\csname PY@tok@mi\endcsname{\def\PY@tc##1{\textcolor[rgb]{0.40,0.40,0.40}{##1}}}
\expandafter\def\csname PY@tok@gp\endcsname{\let\PY@bf=\textbf\def\PY@tc##1{\textcolor[rgb]{0.00,0.00,0.50}{##1}}}
\expandafter\def\csname PY@tok@o\endcsname{\def\PY@tc##1{\textcolor[rgb]{0.40,0.40,0.40}{##1}}}
\expandafter\def\csname PY@tok@kr\endcsname{\let\PY@bf=\textbf\def\PY@tc##1{\textcolor[rgb]{0.00,0.50,0.00}{##1}}}
\expandafter\def\csname PY@tok@s\endcsname{\def\PY@tc##1{\textcolor[rgb]{0.73,0.13,0.13}{##1}}}
\expandafter\def\csname PY@tok@kp\endcsname{\def\PY@tc##1{\textcolor[rgb]{0.00,0.50,0.00}{##1}}}
\expandafter\def\csname PY@tok@w\endcsname{\def\PY@tc##1{\textcolor[rgb]{0.73,0.73,0.73}{##1}}}
\expandafter\def\csname PY@tok@kt\endcsname{\def\PY@tc##1{\textcolor[rgb]{0.69,0.00,0.25}{##1}}}
\expandafter\def\csname PY@tok@ow\endcsname{\let\PY@bf=\textbf\def\PY@tc##1{\textcolor[rgb]{0.67,0.13,1.00}{##1}}}
\expandafter\def\csname PY@tok@sb\endcsname{\def\PY@tc##1{\textcolor[rgb]{0.73,0.13,0.13}{##1}}}
\expandafter\def\csname PY@tok@k\endcsname{\let\PY@bf=\textbf\def\PY@tc##1{\textcolor[rgb]{0.00,0.50,0.00}{##1}}}
\expandafter\def\csname PY@tok@se\endcsname{\let\PY@bf=\textbf\def\PY@tc##1{\textcolor[rgb]{0.73,0.40,0.13}{##1}}}
\expandafter\def\csname PY@tok@sd\endcsname{\let\PY@it=\textit\def\PY@tc##1{\textcolor[rgb]{0.73,0.13,0.13}{##1}}}

\def\PYZbs{\char`\\}
\def\PYZus{\char`\_}
\def\PYZob{\char`\{}
\def\PYZcb{\char`\}}
\def\PYZca{\char`\^}
\def\PYZam{\char`\&}
\def\PYZlt{\char`\<}
\def\PYZgt{\char`\>}
\def\PYZsh{\char`\#}
\def\PYZpc{\char`\%}
\def\PYZdl{\char`\$}
\def\PYZhy{\char`\-}
\def\PYZsq{\char`\'}
\def\PYZdq{\char`\"}
\def\PYZti{\char`\~}
% for compatibility with earlier versions
\def\PYZat{@}
\def\PYZlb{[}
\def\PYZrb{]}
\makeatother


    % Exact colors from NB
    \definecolor{incolor}{rgb}{0.0, 0.0, 0.5}
    \definecolor{outcolor}{rgb}{0.545, 0.0, 0.0}



    
    % Prevent overflowing lines due to hard-to-break entities
    \sloppy 
    % Setup hyperref package
    \hypersetup{
      breaklinks=true,  % so long urls are correctly broken across lines
      colorlinks=true,
      urlcolor=blue,
      linkcolor=darkorange,
      citecolor=darkgreen,
      }
    % Slightly bigger margins than the latex defaults
    
    \geometry{verbose,tmargin=1in,bmargin=1in,lmargin=1in,rmargin=1in}
    
     \parindent = 0.0 in
    \parskip = 0.1 in

    \begin{document}

    
    {\textbf{\LARGE{Jupyter notebook demonstrating the use of additional PmagPy
functions}}\label{jupyter-notebook-demonstrating-the-use-of-additional-pmagpy-functions}

    This Jupyter notebook demonstrates a number of PmagPy functions within a
notebook environment running a Python 2.7 kernel. The benefits of
working within these notebooks include: reproducibility, interactive
code development, convenient workspace for projects, version control
(when integrated with GitHub or other version control software) and ease
of sharing.

The notebook can be viewed as html at the following link where the code and tables are better rendered than in this PDF: \url{http://pmagpy.github.io/Additional_PmagPy_Examples.html}

\section{Contents of the notebook}
\subsection{Paleomagnetic Data Analysis
Walkthrough}\label{paleomagnetic-data-analysis-walkthrough}

\textbf{Basic Functions} * \hyperref[The-dipole-equation]{The Dipole
Equation} *
\hyperref[Get-local-geomagnetic-field-estimate-from-IGRF]{Get local
geomagnetic field estimate from IGRF} *
\hyperref[Plotting-Directions]{Plotting Directional Data} *
\hyperref[Calculate-the-Angle-Between-Directions]{Calculating the Angle
Between Two Directions} *
\hyperref[Generate-and-plot-Fisher-distributed-unit-vectors-from-a-specified-distribution]{Fisher-Distributed
Directions} * \hyperref[Flip-polarity-of-directional-data]{Flip
Directional Data}

\textbf{Data Analysis} *
\hyperref[Test-directional-data-for-Fisher-distribution]{Test if
Directions Are Fisher-Distributed} *
\hyperref[Squish-directional-data]{Simulating Inclination Error in
Paleomagnetic Data} * \hyperref[Unsquish-directional-data]{Correcting
for Inclination Error in Paleomagnetic Data} *
\hyperref[Bootstrap-Reversal-Test]{Bootstrap Reversal Test} *
\hyperref[MM1990]{McFadden and McElhinny (1990) Reversal Test}

\textbf{Plotting Paleomagnetic Poles} *
\hyperref[Working-with-Poles]{Working with Poles} *
\hyperref[Calculate-and-Plot-VGPs]{Calculate and Plot VGPs} *
\hyperref[Plotting-APWPs]{Plotting APWPs}

\subsection{Rock Magnetism Data
Analysis}\label{rock-magnetism-data-analysis}

\begin{itemize}
\tightlist
\item
  \hyperref[Working-with-anisotropy-data]{Working with Anisotropy Data}
\item
  \hyperref[Curie-temperature-data]{Working with Curie Temperature Data}
\item
  \hyperref[Day-plots]{Day Plots}
\item
  \hyperref[Hysteresis-Loops]{Hysteresis Loops}
\item
  \hyperref[Demagnetization-Curves]{Demagnetization Curves}
\end{itemize}

\subsection{Additional Features of the Jupyter
Notebook}\label{additional-features-of-the-jupyter-notebook}

\begin{itemize}
\tightlist
\item
  \hyperref[Interactive-plotting]{Interactive Plotting}
\end{itemize}

\textit{Note: This notebook makes use of pandas for reading, displaying,
and using data with a dataframe structure. More information about the
pandas module and its use within PmagPy can be found
\href{http://earthref.org/PmagPy/cookbook/\#x1-1850007.4}{here} within
the documentation of the
\href{http://earthref.org/PmagPy/cookbook/}{PmagPy Cookbook}.}

    \begin{Verbatim}[commandchars=\\\{\}]
{\color{incolor}In [{\color{incolor}1}]:} \PY{c}{\PYZsh{} With the PmagPy folder in the PYTHONPATH, }
        \PY{c}{\PYZsh{} the function modules from PmagPy can be imported}
        \PY{k+kn}{import} \PY{n+nn}{pmagpy.ipmag} \PY{k+kn}{as} \PY{n+nn}{ipmag}
        \PY{k+kn}{import} \PY{n+nn}{pmagpy.pmagplotlib} \PY{k+kn}{as} \PY{n+nn}{pmagplotlib}
        \PY{k+kn}{import} \PY{n+nn}{pmagpy.pmag} \PY{k+kn}{as} \PY{n+nn}{pmag}
        
        \PY{k+kn}{from} \PY{n+nn}{mpl\PYZus{}toolkits.basemap} \PY{k+kn}{import} \PY{n}{Basemap}
        \PY{k+kn}{import} \PY{n+nn}{numpy} \PY{k+kn}{as} \PY{n+nn}{np}
        \PY{k+kn}{import} \PY{n+nn}{pandas} \PY{k+kn}{as} \PY{n+nn}{pd}
        \PY{k+kn}{import} \PY{n+nn}{matplotlib.pyplot} \PY{k+kn}{as} \PY{n+nn}{plt}
        \PY{k+kn}{import} \PY{n+nn}{os}
        \PY{o}{\PYZpc{}}\PY{k}{matplotlib} inline
        \PY{o}{\PYZpc{}}\PY{k}{config} InlineBackend.figure\PYZus{}formats = \PYZob{}\PYZsq{}svg\PYZsq{},\PYZcb{}
\end{Verbatim}

    \section{The dipole equation}\label{the-dipole-equation}

    The following demonstrates the use of a simple function
(\textbf{ipmag.lat\_from\_inc}) which uses the dipole equation to return
expected latitude from inclination data as predicted by a pure
geocentric axial dipole. The expected inclination for the geomagnetic
field can be calculated from a specified latitude using
\textbf{ipmag.inc\_from\_lat}.

    \begin{Verbatim}[commandchars=\\\{\}]
{\color{incolor}In [{\color{incolor}2}]:} \PY{n}{inclination} \PY{o}{=} \PY{n+nb}{range}\PY{p}{(}\PY{l+m+mi}{0}\PY{p}{,}\PY{l+m+mi}{90}\PY{p}{,}\PY{l+m+mi}{1}\PY{p}{)}
        \PY{n}{latitude} \PY{o}{=} \PY{p}{[}\PY{p}{]}
        \PY{k}{for} \PY{n}{inc} \PY{o+ow}{in} \PY{n}{inclination}\PY{p}{:}
            \PY{n}{lat} \PY{o}{=} \PY{n}{ipmag}\PY{o}{.}\PY{n}{lat\PYZus{}from\PYZus{}inc}\PY{p}{(}\PY{n}{inc}\PY{p}{)}
            \PY{n}{latitude}\PY{o}{.}\PY{n}{append}\PY{p}{(}\PY{n}{lat}\PY{p}{)}
\end{Verbatim}

    \begin{Verbatim}[commandchars=\\\{\}]
{\color{incolor}In [{\color{incolor}3}]:} \PY{n}{plt}\PY{o}{.}\PY{n}{plot}\PY{p}{(}\PY{n}{inclination}\PY{p}{,}\PY{n}{latitude}\PY{p}{)}
        \PY{n}{plt}\PY{o}{.}\PY{n}{ylabel}\PY{p}{(}\PY{l+s}{\PYZsq{}}\PY{l+s}{latitude}\PY{l+s}{\PYZsq{}}\PY{p}{)}
        \PY{n}{plt}\PY{o}{.}\PY{n}{xlabel}\PY{p}{(}\PY{l+s}{\PYZsq{}}\PY{l+s}{inclination}\PY{l+s}{\PYZsq{}}\PY{p}{)}
        \PY{n}{plt}\PY{o}{.}\PY{n}{show}\PY{p}{(}\PY{p}{)}
\end{Verbatim}

    \begin{center}
    \adjustimage{max size={0.9\linewidth}{0.9\paperheight}}{Additional_PmagPy_Examples_files/Additional_PmagPy_Examples_6_0.pdf}
    \end{center}
    { \hspace*{\fill} \\}
    
    \hyperref[Jupyter-notebook-demonstrating-the-use-of-additional-PmagPy-functions]{Go
to Top}

    \section{Get local geomagnetic field estimate from
IGRF}\label{get-local-geomagnetic-field-estimate-from-igrf}

    The function \textbf{ipmag.igrf} uses the International Geomagnetic
Reference Field (IGRF) model to estimate the geomagnetic field direction
at a particular location and time. Let's find the direction of the
geomagnetic field in Berkeley, California (37.87° N, 122.27° W,
elevation of 52 m) on August 27, 2013 (in decimal format, 2013.6544).

    \begin{Verbatim}[commandchars=\\\{\}]
{\color{incolor}In [{\color{incolor}4}]:} \PY{n}{berk\PYZus{}igrf} \PY{o}{=} \PY{n}{ipmag}\PY{o}{.}\PY{n}{igrf}\PY{p}{(}\PY{p}{[}\PY{l+m+mf}{2013.6544}\PY{p}{,} \PY{o}{.}\PY{l+m+mo}{052}\PY{p}{,} \PY{l+m+mf}{37.871667}\PY{p}{,} \PY{o}{\PYZhy{}}\PY{l+m+mf}{122.272778}\PY{p}{]}\PY{p}{)}
        \PY{n}{ipmag}\PY{o}{.}\PY{n}{igrf\PYZus{}print}\PY{p}{(}\PY{n}{berk\PYZus{}igrf}\PY{p}{)}
\end{Verbatim}

    \begin{Verbatim}[commandchars=\\\{\}]
Declination: 13.950
Inclination: 61.354
Intensity: 13.950 nT
    \end{Verbatim}

    \hyperref[Jupyter-notebook-demonstrating-the-use-of-additional-PmagPy-functions]{Go
to Top}

    \section{Plotting Directions}\label{plotting-directions}

    We can plot this direction using \textbf{matplotlib} (\textbf{plt}) in
conjunction with a few \textbf{ipmag} functions. To do this, we first
initiate a figure (numbered as Fig. 0, with a size of 6x6) with the
following syntax:

\begin{Shaded}
\begin{Highlighting}[]
\NormalTok{plt.figure(num}\OperatorTok{=}\DecValTok{0}\NormalTok{,figsize}\OperatorTok{=}\NormalTok{(}\DecValTok{6}\NormalTok{,}\DecValTok{6}\NormalTok{))}
\end{Highlighting}
\end{Shaded}

We then draw an equal area stereonet within the figure, specifying the
figure number:

\begin{Shaded}
\begin{Highlighting}[]
\NormalTok{ipmag.plot_net(}\DecValTok{0}\NormalTok{)}
\end{Highlighting}
\end{Shaded}

Now we can plot the direction we just pulled from IGRF using
\textbf{ipmag.plot\_di()}:

\begin{Shaded}
\begin{Highlighting}[]
\NormalTok{ipmag.plot_di(berk_igrf[}\DecValTok{0}\NormalTok{],berk_igrf[}\DecValTok{1}\NormalTok{])}
\end{Highlighting}
\end{Shaded}

To label or color the plotted points, we would pass the same code as
above but with a few extra arguments and one additional line of code:

\begin{Shaded}
\begin{Highlighting}[]
\NormalTok{ipmag.plot_di(berk_igrf[}\DecValTok{0}\NormalTok{],berk_igrf[}\DecValTok{1}\NormalTok{], color}\OperatorTok{=}\StringTok{'r'}\NormalTok{, label}\OperatorTok{=}\StringTok{"Berkeley, CA -- August 27, 2013"}\NormalTok{)}
\NormalTok{plt.legend()}
\end{Highlighting}
\end{Shaded}

We may wish to save the figure we just created. To do so, we would pass
the following \textit{save} function, specifying 1) the relative path to
the folder where we want the figure to be saved and 2) the name of the
file with the desired extension (.pdf in this example):

\begin{Shaded}
\begin{Highlighting}[]
\NormalTok{plt.savefig(}\StringTok{"./Additional_Notebook_Output/Berkeley_IGRF.pdf"}\NormalTok{)}
\end{Highlighting}
\end{Shaded}

To ensure the figure is displayed properly and then cleared from the
namespace, it is good practice to end such a code block with the
following:

\begin{Shaded}
\begin{Highlighting}[]
\NormalTok{plt.show()}
\end{Highlighting}
\end{Shaded}

Now let's run the code we just developed.

    \begin{Verbatim}[commandchars=\\\{\}]
{\color{incolor}In [{\color{incolor}5}]:} \PY{n}{plt}\PY{o}{.}\PY{n}{figure}\PY{p}{(}\PY{n}{num}\PY{o}{=}\PY{l+m+mi}{0}\PY{p}{,}\PY{n}{figsize}\PY{o}{=}\PY{p}{(}\PY{l+m+mi}{5}\PY{p}{,}\PY{l+m+mi}{5}\PY{p}{)}\PY{p}{)}
        \PY{n}{ipmag}\PY{o}{.}\PY{n}{plot\PYZus{}net}\PY{p}{(}\PY{l+m+mi}{0}\PY{p}{)}
        \PY{n}{ipmag}\PY{o}{.}\PY{n}{plot\PYZus{}di}\PY{p}{(}\PY{n}{berk\PYZus{}igrf}\PY{p}{[}\PY{l+m+mi}{0}\PY{p}{]}\PY{p}{,}\PY{n}{berk\PYZus{}igrf}\PY{p}{[}\PY{l+m+mi}{1}\PY{p}{]}\PY{p}{,} \PY{n}{color}\PY{o}{=}\PY{l+s}{\PYZsq{}}\PY{l+s}{r}\PY{l+s}{\PYZsq{}}\PY{p}{,} \PY{n}{label}\PY{o}{=}\PY{l+s}{\PYZdq{}}\PY{l+s}{Berkeley, CA \PYZhy{}\PYZhy{} August 27, 2013}\PY{l+s}{\PYZdq{}}\PY{p}{)}
        \PY{n}{plt}\PY{o}{.}\PY{n}{legend}\PY{p}{(}\PY{p}{)}
        \PY{n}{plt}\PY{o}{.}\PY{n}{savefig}\PY{p}{(}\PY{l+s}{\PYZdq{}}\PY{l+s}{./Additional\PYZus{}Notebook\PYZus{}Output/Berkeley\PYZus{}IGRF.pdf}\PY{l+s}{\PYZdq{}}\PY{p}{)}
        \PY{n}{plt}\PY{o}{.}\PY{n}{show}\PY{p}{(}\PY{p}{)}
\end{Verbatim}

    \begin{center}
    \adjustimage{max size={0.9\linewidth}{0.9\paperheight}}{Additional_PmagPy_Examples_files/Additional_PmagPy_Examples_14_0.pdf}
    \end{center}
    { \hspace*{\fill} \\}
    
    Let's see how this magnetic direction compares to the Geocentric Axial
Dipole (GAD) model of the geomagnetic field. We can estimate the
expected GAD inclination by passing Berkeley's latitude to the function
\textbf{ipmag.inc\_from\_lat}.

We also demonstrate below how to manipulate the placement of the figure
legend to ensure no data points are obscured. \textbf{plt.legend} uses
the ``best'' location by default, but this can be changed with the
following:

\begin{Shaded}
\begin{Highlighting}[]
\NormalTok{plt.legend(loc}\OperatorTok{=}\StringTok{"upper right"}\NormalTok{)}
\end{Highlighting}
\end{Shaded}

or

\begin{Shaded}
\begin{Highlighting}[]
\NormalTok{plt.legend(loc}\OperatorTok{=}\StringTok{"lower center"}\NormalTok{)}
\end{Highlighting}
\end{Shaded}

See the \textbf{plt.legend} documentation for the complete list of
placement options. Alternatively, you can give \texttt{(x,y)}
coordinates to the \texttt{loc=} keyword argument (with the origin
\texttt{(0,0)} at the lower left of the figure). To manipulate placement
even more precisely, use the keyword \texttt{bbox\_to\_anchor} in
conjunction with \texttt{loc}. If this is done, \texttt{loc} becomes the
anchor point on the legend, and \texttt{bbox\_to\_anchor} places this
anchor point at the specified coordinates. The latter method is
demonstrated below. Play around with the \textbf{plt.legend} arguments
to see how this changes things.

    \begin{Verbatim}[commandchars=\\\{\}]
{\color{incolor}In [{\color{incolor}6}]:} \PY{n}{GAD\PYZus{}inc} \PY{o}{=} \PY{n}{ipmag}\PY{o}{.}\PY{n}{inc\PYZus{}from\PYZus{}lat}\PY{p}{(}\PY{l+m+mf}{37.87}\PY{p}{)}
        \PY{n}{plt}\PY{o}{.}\PY{n}{figure}\PY{p}{(}\PY{n}{num}\PY{o}{=}\PY{l+m+mi}{0}\PY{p}{,}\PY{n}{figsize}\PY{o}{=}\PY{p}{(}\PY{l+m+mi}{5}\PY{p}{,}\PY{l+m+mi}{5}\PY{p}{)}\PY{p}{)}
        \PY{n}{ipmag}\PY{o}{.}\PY{n}{plot\PYZus{}net}\PY{p}{(}\PY{l+m+mi}{0}\PY{p}{)}
        \PY{n}{ipmag}\PY{o}{.}\PY{n}{plot\PYZus{}di}\PY{p}{(}\PY{n}{berk\PYZus{}igrf}\PY{p}{[}\PY{l+m+mi}{0}\PY{p}{]}\PY{p}{,}\PY{n}{berk\PYZus{}igrf}\PY{p}{[}\PY{l+m+mi}{1}\PY{p}{]}\PY{p}{,} \PY{n}{color}\PY{o}{=}\PY{l+s}{\PYZsq{}}\PY{l+s}{r}\PY{l+s}{\PYZsq{}}\PY{p}{,} \PY{n}{label}\PY{o}{=}\PY{l+s}{\PYZdq{}}\PY{l+s}{Berkeley, CA \PYZhy{}\PYZhy{} August 27, 2013 (IGRF)}\PY{l+s}{\PYZdq{}}\PY{p}{)}
        \PY{n}{ipmag}\PY{o}{.}\PY{n}{plot\PYZus{}di}\PY{p}{(}\PY{l+m+mi}{0}\PY{p}{,}\PY{n}{GAD\PYZus{}inc}\PY{p}{,} \PY{n}{color}\PY{o}{=}\PY{l+s}{\PYZsq{}}\PY{l+s}{b}\PY{l+s}{\PYZsq{}}\PY{p}{,} \PY{n}{label}\PY{o}{=}\PY{l+s}{\PYZdq{}}\PY{l+s}{Berkeley, CA (GAD)}\PY{l+s}{\PYZdq{}}\PY{p}{)}
        \PY{n}{plt}\PY{o}{.}\PY{n}{legend}\PY{p}{(}\PY{n}{loc}\PY{o}{=}\PY{l+s}{\PYZsq{}}\PY{l+s}{center left}\PY{l+s}{\PYZsq{}}\PY{p}{,} \PY{n}{bbox\PYZus{}to\PYZus{}anchor}\PY{o}{=}\PY{p}{(}\PY{l+m+mf}{1.0}\PY{p}{,} \PY{l+m+mf}{0.5}\PY{p}{)}\PY{p}{)}
        \PY{n}{plt}\PY{o}{.}\PY{n}{show}\PY{p}{(}\PY{p}{)}
\end{Verbatim}

    \begin{center}
    \adjustimage{max size={0.9\linewidth}{0.9\paperheight}}{Additional_PmagPy_Examples_files/Additional_PmagPy_Examples_16_0.pdf}
    \end{center}
    { \hspace*{\fill} \\}
    
    Below, we calculate the angular difference between these two directions.

    \hyperref[Jupyter-notebook-demonstrating-the-use-of-additional-PmagPy-functions]{Go
to Top}

    \section{Calculate the Angle Between
Directions}\label{calculate-the-angle-between-directions}

    While \textbf{ipmag} functions have been optimized to preform tasks
within an interactive computing environment such as the Jupyter
notebook, the \textbf{pmag} functions which are used extensively within
\textbf{ipmag} can also be directly called. Here is a demonstration of
the function \textbf{pmag.angle}, which calculates the angle between two
directions and outputs a \textbf{numpy} array. Continuing our comparison
from the last section, let's calculate the angle between the IGRF and
GAD-estimated magnetic directions calculated and plotted above.

    \begin{Verbatim}[commandchars=\\\{\}]
{\color{incolor}In [{\color{incolor}7}]:} \PY{n}{direction1} \PY{o}{=} \PY{p}{[}\PY{n}{berk\PYZus{}igrf}\PY{p}{[}\PY{l+m+mi}{0}\PY{p}{]}\PY{p}{,}\PY{n}{berk\PYZus{}igrf}\PY{p}{[}\PY{l+m+mi}{1}\PY{p}{]}\PY{p}{]}
        \PY{n}{direction2} \PY{o}{=} \PY{p}{[}\PY{l+m+mi}{0}\PY{p}{,}\PY{n}{GAD\PYZus{}inc}\PY{p}{]}
        \PY{k}{print} \PY{n}{pmag}\PY{o}{.}\PY{n}{angle}\PY{p}{(}\PY{n}{direction1}\PY{p}{,}\PY{n}{direction2}\PY{p}{)}\PY{p}{[}\PY{l+m+mi}{0}\PY{p}{]}
\end{Verbatim}

    \begin{Verbatim}[commandchars=\\\{\}]
8.18973048085
    \end{Verbatim}

    \hyperref[Jupyter-notebook-demonstrating-the-use-of-additional-PmagPy-functions]{Go
to Top}

    \section{Generate and plot Fisher distributed unit vectors from a
specified
distribution}\label{generate-and-plot-fisher-distributed-unit-vectors-from-a-specified-distribution}

    Let's use the function \textbf{ipmag.fishrot} to generate a set of 50
Fisher-distributed directions at a declination of 200° and inclination
of 45°. These directions will serve as an example paleomagnetic dataset
that will be used for the next several examples. The output from
\textbf{ipmag.fishrot} is a nested list of lists of vectors
(declination, inclination, intensity). Generally these vectors are unit
vectors with an intensity of 1.0. We refer to this data structure as a
di\_block. In the code below the first two vectors are shown.

    \begin{Verbatim}[commandchars=\\\{\}]
{\color{incolor}In [{\color{incolor}8}]:} \PY{n}{fisher\PYZus{}directions} \PY{o}{=} \PY{n}{ipmag}\PY{o}{.}\PY{n}{fishrot}\PY{p}{(}\PY{n}{k}\PY{o}{=}\PY{l+m+mi}{40}\PY{p}{,} \PY{n}{n}\PY{o}{=}\PY{l+m+mi}{50}\PY{p}{,} \PY{n}{dec}\PY{o}{=}\PY{l+m+mi}{200}\PY{p}{,} \PY{n}{inc}\PY{o}{=}\PY{l+m+mi}{50}\PY{p}{)}
        \PY{n}{fisher\PYZus{}directions}\PY{p}{[}\PY{l+m+mi}{0}\PY{p}{:}\PY{l+m+mi}{2}\PY{p}{]}
\end{Verbatim}

            \begin{Verbatim}[commandchars=\\\{\}]
{\color{outcolor}Out[{\color{outcolor}8}]:} [[183.99588954660211, 45.029054467016621, 1.0],
         [200.58837258558225, 41.811402826036407, 1.0]]
\end{Verbatim}
        
    This di\_block can be unpacked in separate lists of declination and
inclination using the \textbf{ipmag.unpack\_di\_block} function.

    \begin{Verbatim}[commandchars=\\\{\}]
{\color{incolor}In [{\color{incolor}9}]:} \PY{n}{fisher\PYZus{}decs}\PY{p}{,} \PY{n}{fisher\PYZus{}incs} \PY{o}{=} \PY{n}{ipmag}\PY{o}{.}\PY{n}{unpack\PYZus{}di\PYZus{}block}\PY{p}{(}\PY{n}{fisher\PYZus{}directions}\PY{p}{)}
        \PY{k}{print} \PY{n}{fisher\PYZus{}decs}\PY{p}{[}\PY{l+m+mi}{0}\PY{p}{]}
        \PY{k}{print} \PY{n}{fisher\PYZus{}incs}\PY{p}{[}\PY{l+m+mi}{0}\PY{p}{]}
\end{Verbatim}

    \begin{Verbatim}[commandchars=\\\{\}]
183.995889547
45.029054467
    \end{Verbatim}

    Another way to deal with the di\_block is to make it into a pandas
dataframe which allows for the direction to be nicely displayed and
analyzed. In the code below, a dataframe is made from the
\textit{fisher\_directions} di\_block and then the first 5 rows are
displayed with .head().

    \begin{Verbatim}[commandchars=\\\{\}]
{\color{incolor}In [{\color{incolor}10}]:} \PY{n}{directions} \PY{o}{=} \PY{n}{pd}\PY{o}{.}\PY{n}{DataFrame}\PY{p}{(}\PY{n}{fisher\PYZus{}directions}\PY{p}{,}\PY{n}{columns}\PY{o}{=}\PY{p}{[}\PY{l+s}{\PYZsq{}}\PY{l+s}{dec}\PY{l+s}{\PYZsq{}}\PY{p}{,}\PY{l+s}{\PYZsq{}}\PY{l+s}{inc}\PY{l+s}{\PYZsq{}}\PY{p}{,}\PY{l+s}{\PYZsq{}}\PY{l+s}{length}\PY{l+s}{\PYZsq{}}\PY{p}{]}\PY{p}{)}
         \PY{n}{directions}\PY{o}{.}\PY{n}{head}\PY{p}{(}\PY{p}{)}
\end{Verbatim}

            \begin{Verbatim}[commandchars=\\\{\}]
{\color{outcolor}Out[{\color{outcolor}10}]:}           dec        inc  length
         0  183.995890  45.029054       1
         1  200.588373  41.811403       1
         2  199.546318  36.865786       1
         3  182.021565  57.114399       1
         4  221.438217  38.160275       1
\end{Verbatim}
        
    Now let's calculate the Fisher and Bingham means of these data.

    \begin{Verbatim}[commandchars=\\\{\}]
{\color{incolor}In [{\color{incolor}11}]:} \PY{n}{fisher\PYZus{}mean} \PY{o}{=} \PY{n}{ipmag}\PY{o}{.}\PY{n}{fisher\PYZus{}mean}\PY{p}{(}\PY{n}{directions}\PY{o}{.}\PY{n}{dec}\PY{p}{,}\PY{n}{directions}\PY{o}{.}\PY{n}{inc}\PY{p}{)}
         \PY{n}{bingham\PYZus{}mean} \PY{o}{=} \PY{n}{ipmag}\PY{o}{.}\PY{n}{bingham\PYZus{}mean}\PY{p}{(}\PY{n}{directions}\PY{o}{.}\PY{n}{dec}\PY{p}{,}\PY{n}{directions}\PY{o}{.}\PY{n}{inc}\PY{p}{)}
\end{Verbatim}

    Here's the raw output of the Fisher mean which is a dictionary
containing the mean direction and associated statistics:

    \begin{Verbatim}[commandchars=\\\{\}]
{\color{incolor}In [{\color{incolor}12}]:} \PY{n}{fisher\PYZus{}mean}
\end{Verbatim}

            \begin{Verbatim}[commandchars=\\\{\}]
{\color{outcolor}Out[{\color{outcolor}12}]:} \{'alpha95': 3.159239986865467,
          'csd': 12.553483567656313,
          'dec': 198.86339034639676,
          'inc': 49.196847025953076,
          'k': 41.633365663104868,
          'n': 50,
          'r': 48.823059360693883\}
\end{Verbatim}
        
    The function \textbf{ipmag.print\_direction\_mean} prints formatted
output from this Fisher mean dictionary:

    \begin{Verbatim}[commandchars=\\\{\}]
{\color{incolor}In [{\color{incolor}13}]:} \PY{n}{ipmag}\PY{o}{.}\PY{n}{print\PYZus{}direction\PYZus{}mean}\PY{p}{(}\PY{n}{fisher\PYZus{}mean}\PY{p}{)}
\end{Verbatim}

    \begin{Verbatim}[commandchars=\\\{\}]
Dec: 198.9  Inc: 49.2
Number of directions in mean (n): 50
Angular radius of 95\% confidence (a\_95): 3.2
Precision parameter (k) estimate: 41.6
    \end{Verbatim}

    Now we can plot all of our data using the function
\textbf{ipmag.plot\_di}. We can also plot the Fisher mean with its
angular radius of 95\% confidence ( \(\alpha_{95}\) ) using
\textbf{ipmag.plot\_di\_mean}.

    \begin{Verbatim}[commandchars=\\\{\}]
{\color{incolor}In [{\color{incolor}14}]:} \PY{n}{declinations} \PY{o}{=} \PY{n}{directions}\PY{o}{.}\PY{n}{dec}\PY{o}{.}\PY{n}{tolist}\PY{p}{(}\PY{p}{)}
         \PY{n}{inclinations} \PY{o}{=} \PY{n}{directions}\PY{o}{.}\PY{n}{inc}\PY{o}{.}\PY{n}{tolist}\PY{p}{(}\PY{p}{)}
         
         \PY{n}{plt}\PY{o}{.}\PY{n}{figure}\PY{p}{(}\PY{n}{num}\PY{o}{=}\PY{l+m+mi}{1}\PY{p}{,}\PY{n}{figsize}\PY{o}{=}\PY{p}{(}\PY{l+m+mi}{5}\PY{p}{,}\PY{l+m+mi}{5}\PY{p}{)}\PY{p}{)}
         \PY{n}{ipmag}\PY{o}{.}\PY{n}{plot\PYZus{}net}\PY{p}{(}\PY{l+m+mi}{1}\PY{p}{)}
         \PY{n}{ipmag}\PY{o}{.}\PY{n}{plot\PYZus{}di}\PY{p}{(}\PY{n}{declinations}\PY{p}{,}\PY{n}{inclinations}\PY{p}{)}
         \PY{n}{ipmag}\PY{o}{.}\PY{n}{plot\PYZus{}di\PYZus{}mean}\PY{p}{(}\PY{n}{fisher\PYZus{}mean}\PY{p}{[}\PY{l+s}{\PYZsq{}}\PY{l+s}{dec}\PY{l+s}{\PYZsq{}}\PY{p}{]}\PY{p}{,}\PY{n}{fisher\PYZus{}mean}\PY{p}{[}\PY{l+s}{\PYZsq{}}\PY{l+s}{inc}\PY{l+s}{\PYZsq{}}\PY{p}{]}\PY{p}{,}\PY{n}{fisher\PYZus{}mean}\PY{p}{[}\PY{l+s}{\PYZsq{}}\PY{l+s}{alpha95}\PY{l+s}{\PYZsq{}}\PY{p}{]}\PY{p}{,}\PY{n}{color}\PY{o}{=}\PY{l+s}{\PYZsq{}}\PY{l+s}{r}\PY{l+s}{\PYZsq{}}\PY{p}{)}
\end{Verbatim}

    \begin{center}
    \adjustimage{max size={0.9\linewidth}{0.9\paperheight}}{Additional_PmagPy_Examples_files/Additional_PmagPy_Examples_37_0.pdf}
    \end{center}
    { \hspace*{\fill} \\}
    
    \hyperref[Jupyter-notebook-demonstrating-the-use-of-additional-PmagPy-functions]{Go
to Top}

    \section{Flip polarity of directional
data}\label{flip-polarity-of-directional-data}

    Let's flip all the directions (find their antipodes) of the
Fisher-distributed population using the function
\textbf{ipmag.do\_flip()} function and plot the resulting directions.

    \begin{Verbatim}[commandchars=\\\{\}]
{\color{incolor}In [{\color{incolor}15}]:} \PY{c}{\PYZsh{} get reversed directions}
         \PY{n}{dec\PYZus{}reversed}\PY{p}{,}\PY{n}{inc\PYZus{}reversed} \PY{o}{=} \PY{n}{ipmag}\PY{o}{.}\PY{n}{do\PYZus{}flip}\PY{p}{(}\PY{n}{declinations}\PY{p}{,}\PY{n}{inclinations}\PY{p}{)}
         
         \PY{c}{\PYZsh{} take the Fisher mean of these reversed directions}
         \PY{n}{rev\PYZus{}mean} \PY{o}{=} \PY{n}{ipmag}\PY{o}{.}\PY{n}{fisher\PYZus{}mean}\PY{p}{(}\PY{n}{dec\PYZus{}reversed}\PY{p}{,}\PY{n}{inc\PYZus{}reversed}\PY{p}{)}
         
         \PY{c}{\PYZsh{} plot the flipped directions}
         \PY{n}{plt}\PY{o}{.}\PY{n}{figure}\PY{p}{(}\PY{n}{num}\PY{o}{=}\PY{l+m+mi}{1}\PY{p}{,}\PY{n}{figsize}\PY{o}{=}\PY{p}{(}\PY{l+m+mi}{5}\PY{p}{,}\PY{l+m+mi}{5}\PY{p}{)}\PY{p}{)}
         \PY{n}{ipmag}\PY{o}{.}\PY{n}{plot\PYZus{}net}\PY{p}{(}\PY{l+m+mi}{1}\PY{p}{)}
         \PY{n}{ipmag}\PY{o}{.}\PY{n}{plot\PYZus{}di}\PY{p}{(}\PY{n}{dec\PYZus{}reversed}\PY{p}{,} \PY{n}{inc\PYZus{}reversed}\PY{p}{)}
         \PY{n}{ipmag}\PY{o}{.}\PY{n}{plot\PYZus{}di\PYZus{}mean}\PY{p}{(}\PY{n}{rev\PYZus{}mean}\PY{p}{[}\PY{l+s}{\PYZsq{}}\PY{l+s}{dec}\PY{l+s}{\PYZsq{}}\PY{p}{]}\PY{p}{,}\PY{n}{rev\PYZus{}mean}\PY{p}{[}\PY{l+s}{\PYZsq{}}\PY{l+s}{inc}\PY{l+s}{\PYZsq{}}\PY{p}{]}\PY{p}{,}\PY{n}{rev\PYZus{}mean}\PY{p}{[}\PY{l+s}{\PYZsq{}}\PY{l+s}{alpha95}\PY{l+s}{\PYZsq{}}\PY{p}{]}\PY{p}{,}\PY{n}{color}\PY{o}{=}\PY{l+s}{\PYZsq{}}\PY{l+s}{r}\PY{l+s}{\PYZsq{}}\PY{p}{,}\PY{n}{marker}\PY{o}{=}\PY{l+s}{\PYZsq{}}\PY{l+s}{s}\PY{l+s}{\PYZsq{}}\PY{p}{)}
\end{Verbatim}

    \begin{center}
    \adjustimage{max size={0.9\linewidth}{0.9\paperheight}}{Additional_PmagPy_Examples_files/Additional_PmagPy_Examples_41_0.pdf}
    \end{center}
    { \hspace*{\fill} \\}
    
    \hyperref[Jupyter-notebook-demonstrating-the-use-of-additional-PmagPy-functions]{Go
to Top}

    \section{Test directional data for Fisher
distribution}\label{test-directional-data-for-fisher-distribution}

    The function \textbf{ipmag.fishqq} tests whether directional data are
Fisher-distributed. Let's use this test on the random Fisher-distributed
directions we just created (it should pass!).

    \begin{Verbatim}[commandchars=\\\{\}]
{\color{incolor}In [{\color{incolor}16}]:} \PY{n}{ipmag}\PY{o}{.}\PY{n}{fishqq}\PY{p}{(}\PY{n}{declinations}\PY{p}{,} \PY{n}{inclinations}\PY{p}{)}
\end{Verbatim}

            \begin{Verbatim}[commandchars=\\\{\}]
{\color{outcolor}Out[{\color{outcolor}16}]:} \{'Dec': 198.86305338122148,
          'Inc': 49.174480281318843,
          'Me': 0.83647984581601553,
          'Me\_critical': 1.094,
          'Mode': 'Mode 1',
          'Mu': 0.83924082417308898,
          'Mu\_critical': 1.207,
          'N': 50,
          'Test\_result': 'consistent with Fisherian model'\}
\end{Verbatim}
        
    \begin{center}
    \adjustimage{max size={0.9\linewidth}{0.9\paperheight}}{Additional_PmagPy_Examples_files/Additional_PmagPy_Examples_45_1.pdf}
    \end{center}
    { \hspace*{\fill} \\}
    
    \begin{center}
    \adjustimage{max size={0.9\linewidth}{0.9\paperheight}}{Additional_PmagPy_Examples_files/Additional_PmagPy_Examples_45_2.pdf}
    \end{center}
    { \hspace*{\fill} \\}
    
    \hyperref[Jupyter-notebook-demonstrating-the-use-of-additional-PmagPy-functions]{Go
to Top}

    \section{Squish directional data}\label{squish-directional-data}

    Inclination flattening can occur for magnetizations in sedimentary
rocks. We can simulate inclination error of a specified ``flattening
factor'' with the function \textbf{ipmag.squish}. Flattening factors
range from 0 (completely flattened) to 1 (no flattening). Let's squish
our directions with a 0.4 flattening factor.

    \begin{Verbatim}[commandchars=\\\{\}]
{\color{incolor}In [{\color{incolor}17}]:} \PY{c}{\PYZsh{} squish all inclinations}
         \PY{n}{squished\PYZus{}incs} \PY{o}{=} \PY{p}{[}\PY{p}{]}
         \PY{k}{for} \PY{n}{inclination} \PY{o+ow}{in} \PY{n}{inclinations}\PY{p}{:}
             \PY{n}{squished\PYZus{}incs}\PY{o}{.}\PY{n}{append}\PY{p}{(}\PY{n}{ipmag}\PY{o}{.}\PY{n}{squish}\PY{p}{(}\PY{n}{inclination}\PY{p}{,} \PY{l+m+mf}{0.4}\PY{p}{)}\PY{p}{)}
         
         \PY{c}{\PYZsh{} plot the squished directional data}
         \PY{n}{plt}\PY{o}{.}\PY{n}{figure}\PY{p}{(}\PY{n}{num}\PY{o}{=}\PY{l+m+mi}{1}\PY{p}{,}\PY{n}{figsize}\PY{o}{=}\PY{p}{(}\PY{l+m+mi}{5}\PY{p}{,}\PY{l+m+mi}{5}\PY{p}{)}\PY{p}{)}
         \PY{n}{ipmag}\PY{o}{.}\PY{n}{plot\PYZus{}net}\PY{p}{(}\PY{l+m+mi}{1}\PY{p}{)}
         \PY{n}{ipmag}\PY{o}{.}\PY{n}{plot\PYZus{}di}\PY{p}{(}\PY{n}{declinations}\PY{p}{,}\PY{n}{squished\PYZus{}incs}\PY{p}{)}
         \PY{n}{squished\PYZus{}DIs} \PY{o}{=} \PY{n}{np}\PY{o}{.}\PY{n}{array}\PY{p}{(}\PY{n+nb}{zip}\PY{p}{(}\PY{n}{declinations}\PY{p}{,}\PY{n}{squished\PYZus{}incs}\PY{p}{)}\PY{p}{)}
\end{Verbatim}

    \begin{center}
    \adjustimage{max size={0.9\linewidth}{0.9\paperheight}}{Additional_PmagPy_Examples_files/Additional_PmagPy_Examples_49_0.pdf}
    \end{center}
    { \hspace*{\fill} \\}
    
    \begin{Verbatim}[commandchars=\\\{\}]
{\color{incolor}In [{\color{incolor}18}]:} \PY{n}{ipmag}\PY{o}{.}\PY{n}{fisher\PYZus{}mean}\PY{p}{(}\PY{n}{di\PYZus{}block}\PY{o}{=}\PY{n}{squished\PYZus{}DIs}\PY{p}{)}
\end{Verbatim}

            \begin{Verbatim}[commandchars=\\\{\}]
{\color{outcolor}Out[{\color{outcolor}18}]:} \{'alpha95': 3.7940228548357475,
          'csd': 14.997170953500827,
          'dec': 198.72106646496212,
          'inc': 25.567588455989114,
          'k': 29.171002445333873,
          'n': 50,
          'r': 48.320249703731456\}
\end{Verbatim}
        
    \hyperref[Jupyter-notebook-demonstrating-the-use-of-additional-PmagPy-functions]{Go
to Top}

    \section{Unsquish directional data}\label{unsquish-directional-data}

    We can also ``unsquish'' data by a specified flattening factor. Let's
unsquish the data we squished above with the function
\textbf{ipmag.unsquish}. Using a flattening factor of 0.4 will restore
the data to its original state.

    \begin{Verbatim}[commandchars=\\\{\}]
{\color{incolor}In [{\color{incolor}19}]:} \PY{n}{unsquished\PYZus{}incs} \PY{o}{=} \PY{p}{[}\PY{p}{]}
         \PY{k}{for} \PY{n}{squished\PYZus{}inc} \PY{o+ow}{in} \PY{n}{squished\PYZus{}incs}\PY{p}{:}
             \PY{n}{unsquished\PYZus{}incs}\PY{o}{.}\PY{n}{append}\PY{p}{(}\PY{n}{ipmag}\PY{o}{.}\PY{n}{unsquish}\PY{p}{(}\PY{n}{squished\PYZus{}inc}\PY{p}{,} \PY{l+m+mf}{0.4}\PY{p}{)}\PY{p}{)}
         
         \PY{c}{\PYZsh{} plot the squished directional data}
         \PY{n}{plt}\PY{o}{.}\PY{n}{figure}\PY{p}{(}\PY{n}{num}\PY{o}{=}\PY{l+m+mi}{1}\PY{p}{,}\PY{n}{figsize}\PY{o}{=}\PY{p}{(}\PY{l+m+mi}{5}\PY{p}{,}\PY{l+m+mi}{5}\PY{p}{)}\PY{p}{)}
         \PY{n}{ipmag}\PY{o}{.}\PY{n}{plot\PYZus{}net}\PY{p}{(}\PY{l+m+mi}{1}\PY{p}{)}
         \PY{n}{ipmag}\PY{o}{.}\PY{n}{plot\PYZus{}di}\PY{p}{(}\PY{n}{declinations}\PY{p}{,}\PY{n}{unsquished\PYZus{}incs}\PY{p}{)}
\end{Verbatim}

    \begin{center}
    \adjustimage{max size={0.9\linewidth}{0.9\paperheight}}{Additional_PmagPy_Examples_files/Additional_PmagPy_Examples_54_0.pdf}
    \end{center}
    { \hspace*{\fill} \\}
    
    \hyperref[Jupyter-notebook-demonstrating-the-use-of-additional-PmagPy-functions]{Go
to Top}

    \section{Bootstrap Reversal Test}\label{bootstrap-reversal-test}

    Here we carry out two types of reversal tests with \textbf{ipmag} to
test if two populations are antipodal to one another: the bootstrap
reversal test (Tauxe, 2010; \textbf{ipmag.reversal\_test\_bootstrap})
and the McFadden and McElhinny (1990) reversal test, which is an
adaptation of the Watson V test for a common mean
(\textbf{ipmag.reversal\_test\_MM1990}). The code below uses
\textbf{ipmag.fishrot} to simulate normal directions and reversed
directions from antipodal Fisher distributions.

    \begin{Verbatim}[commandchars=\\\{\}]
{\color{incolor}In [{\color{incolor}20}]:} \PY{n}{normal\PYZus{}directions} \PY{o}{=} \PY{n}{ipmag}\PY{o}{.}\PY{n}{fishrot}\PY{p}{(}\PY{n}{k}\PY{o}{=}\PY{l+m+mi}{20}\PY{p}{,}\PY{n}{n}\PY{o}{=}\PY{l+m+mi}{40}\PY{p}{,}\PY{n}{dec}\PY{o}{=}\PY{l+m+mi}{30}\PY{p}{,}\PY{n}{inc}\PY{o}{=}\PY{l+m+mi}{45}\PY{p}{)}
         \PY{n}{reversed\PYZus{}directions} \PY{o}{=} \PY{n}{ipmag}\PY{o}{.}\PY{n}{fishrot}\PY{p}{(}\PY{n}{k}\PY{o}{=}\PY{l+m+mi}{20}\PY{p}{,}\PY{n}{n}\PY{o}{=}\PY{l+m+mi}{30}\PY{p}{,}\PY{n}{dec}\PY{o}{=}\PY{l+m+mi}{210}\PY{p}{,}\PY{n}{inc}\PY{o}{=}\PY{o}{\PYZhy{}}\PY{l+m+mi}{45}\PY{p}{)}
         \PY{n}{combined\PYZus{}directions} \PY{o}{=} \PY{n}{normal\PYZus{}directions} \PY{o}{+} \PY{n}{reversed\PYZus{}directions}
         
         \PY{n}{plt}\PY{o}{.}\PY{n}{figure}\PY{p}{(}\PY{n}{num}\PY{o}{=}\PY{l+m+mi}{1}\PY{p}{,}\PY{n}{figsize}\PY{o}{=}\PY{p}{(}\PY{l+m+mi}{5}\PY{p}{,}\PY{l+m+mi}{5}\PY{p}{)}\PY{p}{)}
         \PY{n}{ipmag}\PY{o}{.}\PY{n}{plot\PYZus{}net}\PY{p}{(}\PY{l+m+mi}{1}\PY{p}{)}
         \PY{n}{ipmag}\PY{o}{.}\PY{n}{plot\PYZus{}di}\PY{p}{(}\PY{n}{di\PYZus{}block}\PY{o}{=}\PY{n}{combined\PYZus{}directions}\PY{p}{)}
\end{Verbatim}

    \begin{center}
    \adjustimage{max size={0.9\linewidth}{0.9\paperheight}}{Additional_PmagPy_Examples_files/Additional_PmagPy_Examples_58_0.pdf}
    \end{center}
    { \hspace*{\fill} \\}
    
    \begin{Verbatim}[commandchars=\\\{\}]
{\color{incolor}In [{\color{incolor}21}]:} \PY{n}{ipmag}\PY{o}{.}\PY{n}{reversal\PYZus{}test\PYZus{}bootstrap}\PY{p}{(}\PY{n}{di\PYZus{}block}\PY{o}{=}\PY{n}{combined\PYZus{}directions}\PY{p}{,} 
                                       \PY{n}{plot\PYZus{}stereo}\PY{o}{=}\PY{n+nb+bp}{True}\PY{p}{,} \PY{n}{save}\PY{o}{=}\PY{n+nb+bp}{True}\PY{p}{,} 
                                       \PY{n}{save\PYZus{}folder}\PY{o}{=}\PY{l+s}{\PYZsq{}}\PY{l+s}{./Additional\PYZus{}Notebook\PYZus{}Output/}\PY{l+s}{\PYZsq{}}\PY{p}{)}
\end{Verbatim}

    \begin{Verbatim}[commandchars=\\\{\}]
Here are the results of the bootstrap test for a common mean:
    \end{Verbatim}

    \begin{center}
    \adjustimage{max size={0.9\linewidth}{0.9\paperheight}}{Additional_PmagPy_Examples_files/Additional_PmagPy_Examples_59_1.pdf}
    \end{center}
    { \hspace*{\fill} \\}
    
    \begin{center}
    \adjustimage{max size={0.9\linewidth}{0.9\paperheight}}{Additional_PmagPy_Examples_files/Additional_PmagPy_Examples_59_2.pdf}
    \end{center}
    { \hspace*{\fill} \\}
    
    \hyperref[Jupyter-notebook-demonstrating-the-use-of-additional-PmagPy-functions]{Go
to Top}

     \#\# McFadden and McElhinny (1990) Reversal Test

    \begin{Verbatim}[commandchars=\\\{\}]
{\color{incolor}In [{\color{incolor}22}]:} \PY{n}{ipmag}\PY{o}{.}\PY{n}{reversal\PYZus{}test\PYZus{}MM1990}\PY{p}{(}\PY{n}{di\PYZus{}block}\PY{o}{=}\PY{n}{combined\PYZus{}directions}\PY{p}{,} 
                                    \PY{n}{plot\PYZus{}CDF}\PY{o}{=}\PY{n+nb+bp}{True}\PY{p}{,} \PY{n}{plot\PYZus{}stereo}\PY{o}{=}\PY{n+nb+bp}{True}\PY{p}{,} 
                                    \PY{n}{save}\PY{o}{=}\PY{n+nb+bp}{True}\PY{p}{,} \PY{n}{save\PYZus{}folder}\PY{o}{=} \PY{l+s}{\PYZsq{}}\PY{l+s}{./Additional\PYZus{}Notebook\PYZus{}Output/}\PY{l+s}{\PYZsq{}}\PY{p}{)}
\end{Verbatim}

    \begin{Verbatim}[commandchars=\\\{\}]
Results of Watson V test: 

Watson's V:           2.3
Critical value of V:  6.1
"Pass": Since V is less than Vcrit, the null hypothesis
that the two populations are drawn from distributions
that share a common mean direction can not be rejected.

M\&M1990 classification:

Angle between data set means: 4.6
Critical angle for M\&M1990:   7.4
The McFadden and McElhinny (1990) classification for
this test is: 'B'
    \end{Verbatim}

    \begin{center}
    \adjustimage{max size={0.9\linewidth}{0.9\paperheight}}{Additional_PmagPy_Examples_files/Additional_PmagPy_Examples_62_1.pdf}
    \end{center}
    { \hspace*{\fill} \\}
    
    \begin{center}
    \adjustimage{max size={0.9\linewidth}{0.9\paperheight}}{Additional_PmagPy_Examples_files/Additional_PmagPy_Examples_62_2.pdf}
    \end{center}
    { \hspace*{\fill} \\}
    
    \hyperref[Jupyter-notebook-demonstrating-the-use-of-additional-PmagPy-functions]{Go
to Top}

    \section{Working with Poles}\label{working-with-poles}

    A variety of plotting functions within PmagPy, together with the Basemap
package of matplotlib, provide a great way to work with paleomagnetic
poles, virtual geomagnetic poles, and polar wander paths.

    \begin{Verbatim}[commandchars=\\\{\}]
{\color{incolor}In [{\color{incolor}23}]:} \PY{c}{\PYZsh{} initiate figure and specify figure size}
         \PY{n}{plt}\PY{o}{.}\PY{n}{figure}\PY{p}{(}\PY{n}{figsize}\PY{o}{=}\PY{p}{(}\PY{l+m+mi}{5}\PY{p}{,} \PY{l+m+mi}{5}\PY{p}{)}\PY{p}{)}
         
         \PY{c}{\PYZsh{} initiate a Basemap projection, specifying the latitude and}
         \PY{c}{\PYZsh{} longitude (lat\PYZus{}0 and lon\PYZus{}0) at which our figure is centered.}
         \PY{n}{pmap} \PY{o}{=} \PY{n}{Basemap}\PY{p}{(}\PY{n}{projection}\PY{o}{=}\PY{l+s}{\PYZsq{}}\PY{l+s}{ortho}\PY{l+s}{\PYZsq{}}\PY{p}{,}\PY{n}{lat\PYZus{}0}\PY{o}{=}\PY{l+m+mi}{30}\PY{p}{,}\PY{n}{lon\PYZus{}0}\PY{o}{=}\PY{l+m+mi}{320}\PY{p}{,}
                        \PY{n}{resolution}\PY{o}{=}\PY{l+s}{\PYZsq{}}\PY{l+s}{c}\PY{l+s}{\PYZsq{}}\PY{p}{,}\PY{n}{area\PYZus{}thresh}\PY{o}{=}\PY{l+m+mi}{50000}\PY{p}{)}
         \PY{c}{\PYZsh{} other optional modifications to the globe figure}
         \PY{n}{pmap}\PY{o}{.}\PY{n}{drawcoastlines}\PY{p}{(}\PY{n}{linewidth}\PY{o}{=}\PY{l+m+mf}{0.25}\PY{p}{)}
         \PY{n}{pmap}\PY{o}{.}\PY{n}{fillcontinents}\PY{p}{(}\PY{n}{color}\PY{o}{=}\PY{l+s}{\PYZsq{}}\PY{l+s}{bisque}\PY{l+s}{\PYZsq{}}\PY{p}{,}\PY{n}{lake\PYZus{}color}\PY{o}{=}\PY{l+s}{\PYZsq{}}\PY{l+s}{white}\PY{l+s}{\PYZsq{}}\PY{p}{,}\PY{n}{zorder}\PY{o}{=}\PY{l+m+mi}{1}\PY{p}{)}
         \PY{n}{pmap}\PY{o}{.}\PY{n}{drawmapboundary}\PY{p}{(}\PY{n}{fill\PYZus{}color}\PY{o}{=}\PY{l+s}{\PYZsq{}}\PY{l+s}{white}\PY{l+s}{\PYZsq{}}\PY{p}{)}
         \PY{n}{pmap}\PY{o}{.}\PY{n}{drawmeridians}\PY{p}{(}\PY{n}{np}\PY{o}{.}\PY{n}{arange}\PY{p}{(}\PY{l+m+mi}{0}\PY{p}{,}\PY{l+m+mi}{360}\PY{p}{,}\PY{l+m+mi}{30}\PY{p}{)}\PY{p}{)}
         \PY{n}{pmap}\PY{o}{.}\PY{n}{drawparallels}\PY{p}{(}\PY{n}{np}\PY{o}{.}\PY{n}{arange}\PY{p}{(}\PY{o}{\PYZhy{}}\PY{l+m+mi}{90}\PY{p}{,}\PY{l+m+mi}{90}\PY{p}{,}\PY{l+m+mi}{30}\PY{p}{)}\PY{p}{)}
         
         
         \PY{c}{\PYZsh{} Here we plot a pole at 340 E longitude, 30 N latitude with an}
         \PY{c}{\PYZsh{} alpha 95 error angle of 5 degrees. Keyword arguments allow us}
         \PY{c}{\PYZsh{} to specify the label, shape, and color of this data.}
         \PY{n}{ipmag}\PY{o}{.}\PY{n}{plot\PYZus{}pole}\PY{p}{(}\PY{n}{pmap}\PY{p}{,}\PY{l+m+mi}{340}\PY{p}{,}\PY{l+m+mi}{30}\PY{p}{,}\PY{l+m+mi}{5}\PY{p}{,}\PY{n}{label}\PY{o}{=}\PY{l+s}{\PYZsq{}}\PY{l+s}{VGP examples}\PY{l+s}{\PYZsq{}}\PY{p}{,}
                        \PY{n}{marker}\PY{o}{=}\PY{l+s}{\PYZsq{}}\PY{l+s}{s}\PY{l+s}{\PYZsq{}}\PY{p}{,}\PY{n}{color}\PY{o}{=}\PY{l+s}{\PYZsq{}}\PY{l+s}{Blue}\PY{l+s}{\PYZsq{}}\PY{p}{)}
         
         \PY{c}{\PYZsh{} We can plot multiple poles sequentially on the same globe using}
         \PY{c}{\PYZsh{} the same plot\PYZus{}pole function.}
         \PY{n}{ipmag}\PY{o}{.}\PY{n}{plot\PYZus{}pole}\PY{p}{(}\PY{n}{pmap}\PY{p}{,}\PY{l+m+mi}{290}\PY{p}{,}\PY{o}{\PYZhy{}}\PY{l+m+mi}{3}\PY{p}{,}\PY{l+m+mi}{9}\PY{p}{,}\PY{n}{marker}\PY{o}{=}\PY{l+s}{\PYZsq{}}\PY{l+s}{s}\PY{l+s}{\PYZsq{}}\PY{p}{,}\PY{n}{color}\PY{o}{=}\PY{l+s}{\PYZsq{}}\PY{l+s}{Blue}\PY{l+s}{\PYZsq{}}\PY{p}{)}
         
         \PY{n}{plt}\PY{o}{.}\PY{n}{legend}\PY{p}{(}\PY{p}{)}
         \PY{c}{\PYZsh{} Optional save (uncomment to save the figure)}
         \PY{c}{\PYZsh{}plt.savefig(\PYZsq{}Code\PYZus{}output/VGP\PYZus{}example.pdf\PYZsq{})}
         \PY{n}{plt}\PY{o}{.}\PY{n}{show}\PY{p}{(}\PY{p}{)}
\end{Verbatim}

    \begin{center}
    \adjustimage{max size={0.9\linewidth}{0.9\paperheight}}{Additional_PmagPy_Examples_files/Additional_PmagPy_Examples_66_0.pdf}
    \end{center}
    { \hspace*{\fill} \\}
    
    \hyperref[Jupyter-notebook-demonstrating-the-use-of-additional-PmagPy-functions]{Go
to Top}

    \section{Calculate and Plot VGPs}\label{calculate-and-plot-vgps}

    Using the function \textbf{ipmag.vgp\_calc}, we can calculate virtual
geomagnetic poles (VGPs) of our fFisher-distributed directions. We'll
need to first assign a location to these magnetic directions - let's
assume they are from Berkeley, CA (37.87° N, 122.27° W).

    \begin{Verbatim}[commandchars=\\\{\}]
{\color{incolor}In [{\color{incolor}24}]:} \PY{c}{\PYZsh{} plug in site latitude and longitude to the \PYZdq{}directions\PYZdq{} dataframe}
         \PY{n}{directions}\PY{p}{[}\PY{l+s}{\PYZsq{}}\PY{l+s}{site\PYZus{}lat}\PY{l+s}{\PYZsq{}}\PY{p}{]} \PY{o}{=} \PY{l+m+mf}{37.97}
         \PY{n}{directions}\PY{p}{[}\PY{l+s}{\PYZsq{}}\PY{l+s}{site\PYZus{}lon}\PY{l+s}{\PYZsq{}}\PY{p}{]} \PY{o}{=} \PY{o}{\PYZhy{}}\PY{l+m+mf}{122.27}
         
         \PY{c}{\PYZsh{} calculate VGPs (this automatically adds VGP data to the dataframe)}
         \PY{n}{ipmag}\PY{o}{.}\PY{n}{vgp\PYZus{}calc}\PY{p}{(}\PY{n}{directions}\PY{p}{,} \PY{n}{dec\PYZus{}tc} \PY{o}{=} \PY{l+s}{\PYZsq{}}\PY{l+s}{dec}\PY{l+s}{\PYZsq{}}\PY{p}{,} \PY{n}{inc\PYZus{}tc} \PY{o}{=} \PY{l+s}{\PYZsq{}}\PY{l+s}{inc}\PY{l+s}{\PYZsq{}}\PY{p}{)}
         \PY{n}{directions}\PY{o}{.}\PY{n}{head}\PY{p}{(}\PY{p}{)}
\end{Verbatim}

            \begin{Verbatim}[commandchars=\\\{\}]
{\color{outcolor}Out[{\color{outcolor}24}]:}           dec        inc  length  site\_lat  site\_lon  paleolatitude  \textbackslash{}
         0  183.995890  45.029054       1     37.97   -122.27      26.588302   
         1  200.588373  41.811403       1     37.97   -122.27      24.095659   
         2  199.546318  36.865786       1     37.97   -122.27      20.553229   
         3  182.021565  57.114399       1     37.97   -122.27      37.715082   
         4  221.438217  38.160275       1     37.97   -122.27      21.449872   
         
              vgp\_lat     vgp\_lon  vgp\_lat\_rev  vgp\_lon\_rev  
         0 -25.333013  233.776578    25.333013    53.776578  
         1 -24.992308  216.987184    24.992308    36.987184  
         2 -28.660199  216.813085    28.660199    36.813085  
         3 -14.291968  236.079855    14.291968    56.079855  
         4 -18.969679  197.086706    18.969679    17.086706  
\end{Verbatim}
        
    We have already calculated the Fisher mean of this data, so let's
translate it to a VGP too. For a one-line dataset, we plug the Fisher
mean data into a \textbf{pandas} \textit{Series} instead of a
\textit{DataFrame} (a \textit{DataFrame} can be considered a sequence of
concatenated \textit{Series}).

    \begin{Verbatim}[commandchars=\\\{\}]
{\color{incolor}In [{\color{incolor}25}]:} \PY{n}{mean\PYZus{}pole} \PY{o}{=} \PY{n}{pd}\PY{o}{.}\PY{n}{Series}\PY{p}{(}\PY{n}{fisher\PYZus{}mean}\PY{p}{)}
         \PY{n}{mean\PYZus{}pole}\PY{p}{[}\PY{l+s}{\PYZsq{}}\PY{l+s}{site\PYZus{}lat}\PY{l+s}{\PYZsq{}}\PY{p}{]} \PY{o}{=} \PY{l+m+mf}{37.97}
         \PY{n}{mean\PYZus{}pole}\PY{p}{[}\PY{l+s}{\PYZsq{}}\PY{l+s}{site\PYZus{}lon}\PY{l+s}{\PYZsq{}}\PY{p}{]} \PY{o}{=} \PY{o}{\PYZhy{}}\PY{l+m+mf}{122.27}
         \PY{n}{ipmag}\PY{o}{.}\PY{n}{vgp\PYZus{}calc}\PY{p}{(}\PY{n}{mean\PYZus{}pole}\PY{p}{,} \PY{n}{dec\PYZus{}tc} \PY{o}{=} \PY{l+s}{\PYZsq{}}\PY{l+s}{dec}\PY{l+s}{\PYZsq{}}\PY{p}{,} \PY{n}{inc\PYZus{}tc} \PY{o}{=} \PY{l+s}{\PYZsq{}}\PY{l+s}{inc}\PY{l+s}{\PYZsq{}}\PY{p}{)}
         \PY{n}{mean\PYZus{}pole}
\end{Verbatim}

            \begin{Verbatim}[commandchars=\\\{\}]
{\color{outcolor}Out[{\color{outcolor}25}]:} alpha95               3.15924
         csd                   12.5535
         dec                   198.863
         inc                   49.1968
         k                     41.6334
         n                          50
         r                     48.8231
         site\_lat                37.97
         site\_lon              -122.27
         paleolatitude          30.079
         vgp\_lat              -19.7048
         vgp\_lon          220.44194226
         vgp\_lat\_rev           19.7048
         vgp\_lon\_rev           40.4419
         dtype: object
\end{Verbatim}
        
    \begin{Verbatim}[commandchars=\\\{\}]
{\color{incolor}In [{\color{incolor}26}]:} \PY{n}{plt}\PY{o}{.}\PY{n}{figure}\PY{p}{(}\PY{n}{figsize}\PY{o}{=}\PY{p}{(}\PY{l+m+mi}{6}\PY{p}{,} \PY{l+m+mi}{6}\PY{p}{)}\PY{p}{)}
         \PY{n}{pmap} \PY{o}{=} \PY{n}{Basemap}\PY{p}{(}\PY{n}{projection}\PY{o}{=}\PY{l+s}{\PYZsq{}}\PY{l+s}{ortho}\PY{l+s}{\PYZsq{}}\PY{p}{,}\PY{n}{lat\PYZus{}0}\PY{o}{=}\PY{o}{\PYZhy{}}\PY{l+m+mi}{30}\PY{p}{,}\PY{n}{lon\PYZus{}0}\PY{o}{=}\PY{o}{\PYZhy{}}\PY{l+m+mi}{130}\PY{p}{,}
                        \PY{n}{resolution}\PY{o}{=}\PY{l+s}{\PYZsq{}}\PY{l+s}{c}\PY{l+s}{\PYZsq{}}\PY{p}{,}\PY{n}{area\PYZus{}thresh}\PY{o}{=}\PY{l+m+mi}{50000}\PY{p}{)}
         \PY{n}{pmap}\PY{o}{.}\PY{n}{drawcoastlines}\PY{p}{(}\PY{n}{linewidth}\PY{o}{=}\PY{l+m+mf}{0.25}\PY{p}{)}
         \PY{n}{pmap}\PY{o}{.}\PY{n}{fillcontinents}\PY{p}{(}\PY{n}{color}\PY{o}{=}\PY{l+s}{\PYZsq{}}\PY{l+s}{bisque}\PY{l+s}{\PYZsq{}}\PY{p}{,}\PY{n}{lake\PYZus{}color}\PY{o}{=}\PY{l+s}{\PYZsq{}}\PY{l+s}{white}\PY{l+s}{\PYZsq{}}\PY{p}{,}\PY{n}{zorder}\PY{o}{=}\PY{l+m+mi}{1}\PY{p}{)}
         \PY{n}{pmap}\PY{o}{.}\PY{n}{drawmapboundary}\PY{p}{(}\PY{n}{fill\PYZus{}color}\PY{o}{=}\PY{l+s}{\PYZsq{}}\PY{l+s}{white}\PY{l+s}{\PYZsq{}}\PY{p}{)}
         \PY{n}{pmap}\PY{o}{.}\PY{n}{drawmeridians}\PY{p}{(}\PY{n}{np}\PY{o}{.}\PY{n}{arange}\PY{p}{(}\PY{l+m+mi}{0}\PY{p}{,}\PY{l+m+mi}{360}\PY{p}{,}\PY{l+m+mi}{30}\PY{p}{)}\PY{p}{)}
         \PY{n}{pmap}\PY{o}{.}\PY{n}{drawparallels}\PY{p}{(}\PY{n}{np}\PY{o}{.}\PY{n}{arange}\PY{p}{(}\PY{o}{\PYZhy{}}\PY{l+m+mi}{90}\PY{p}{,}\PY{l+m+mi}{90}\PY{p}{,}\PY{l+m+mi}{30}\PY{p}{)}\PY{p}{)}
         
         \PY{c}{\PYZsh{} use the print\PYZus{}pole\PYZus{}mean function to print the mean data above the globe}
         \PY{n}{ipmag}\PY{o}{.}\PY{n}{print\PYZus{}pole\PYZus{}mean}\PY{p}{(}\PY{n}{mean\PYZus{}pole}\PY{p}{)}
         \PY{k}{for} \PY{n}{n} \PY{o+ow}{in} \PY{n+nb}{range}\PY{p}{(}\PY{n+nb}{len}\PY{p}{(}\PY{n}{directions}\PY{p}{)}\PY{p}{)}\PY{p}{:}
             \PY{n}{ipmag}\PY{o}{.}\PY{n}{plot\PYZus{}vgp}\PY{p}{(}\PY{n}{pmap}\PY{p}{,} \PY{n}{directions}\PY{p}{[}\PY{l+s}{\PYZsq{}}\PY{l+s}{vgp\PYZus{}lon}\PY{l+s}{\PYZsq{}}\PY{p}{]}\PY{p}{[}\PY{n}{n}\PY{p}{]}\PY{p}{,} 
                            \PY{n}{directions}\PY{p}{[}\PY{l+s}{\PYZsq{}}\PY{l+s}{vgp\PYZus{}lat}\PY{l+s}{\PYZsq{}}\PY{p}{]}\PY{p}{[}\PY{n}{n}\PY{p}{]}\PY{p}{)}
         \PY{n}{ipmag}\PY{o}{.}\PY{n}{plot\PYZus{}pole}\PY{p}{(}\PY{n}{pmap}\PY{p}{,} \PY{n}{mean\PYZus{}pole}\PY{p}{[}\PY{l+s}{\PYZsq{}}\PY{l+s}{vgp\PYZus{}lon}\PY{l+s}{\PYZsq{}}\PY{p}{]}\PY{p}{,} \PY{n}{mean\PYZus{}pole}\PY{p}{[}\PY{l+s}{\PYZsq{}}\PY{l+s}{vgp\PYZus{}lat}\PY{l+s}{\PYZsq{}}\PY{p}{]}\PY{p}{,} 
                         \PY{n}{mean\PYZus{}pole}\PY{p}{[}\PY{l+s}{\PYZsq{}}\PY{l+s}{alpha95}\PY{l+s}{\PYZsq{}}\PY{p}{]}\PY{p}{,} \PY{n}{color}\PY{o}{=}\PY{l+s}{\PYZsq{}}\PY{l+s}{r}\PY{l+s}{\PYZsq{}}\PY{p}{)}
\end{Verbatim}

    \begin{Verbatim}[commandchars=\\\{\}]
Plong: 198.9  Plat: 49.2
Number of directions in mean (n): 50.0
Angular radius of 95\% confidence (A\_95): 3.2
Precision parameter (k) estimate: 41.6
    \end{Verbatim}

    \begin{center}
    \adjustimage{max size={0.9\linewidth}{0.9\paperheight}}{Additional_PmagPy_Examples_files/Additional_PmagPy_Examples_73_1.pdf}
    \end{center}
    { \hspace*{\fill} \\}
    
    \hyperref[Jupyter-notebook-demonstrating-the-use-of-additional-PmagPy-functions]{Go
to Top}

    \section{Plotting APWPs}\label{plotting-apwps}

    The capability to plot multiple poles in sequence provides a good way to
visualize polar wander paths. Here we use the Phanerozoic APWP of
Laurentia \textit{(Torsvik, 2012)} to demonstrate the plot\_pole\_colorbar
function.

We first upload the Torsvik (2012) data using the pandas function
\textit{read\_csv}.

    \begin{Verbatim}[commandchars=\\\{\}]
{\color{incolor}In [{\color{incolor}27}]:} \PY{n}{Laurentia\PYZus{}Pole\PYZus{}Compilation} \PY{o}{=} \PY{n}{pd}\PY{o}{.}\PY{n}{read\PYZus{}csv}\PY{p}{(}\PY{l+s}{\PYZsq{}}\PY{l+s}{./Additional\PYZus{}Data/Torsvik2012/Laurentia\PYZus{}Pole\PYZus{}Compilation.csv}\PY{l+s}{\PYZsq{}}\PY{p}{)}
         \PY{n}{Laurentia\PYZus{}Pole\PYZus{}Compilation}\PY{o}{.}\PY{n}{head}\PY{p}{(}\PY{p}{)}
\end{Verbatim}

            \begin{Verbatim}[commandchars=\\\{\}]
{\color{outcolor}Out[{\color{outcolor}27}]:}    Q  A95  Com                 Formation   Lat    Lon  CLat   CLon  RLat  \textbackslash{}
         0  5  3.9  NaN         Dunkard Formation -44.1  301.5 -41.5  300.4 -38.0   
         1  5  2.1  NaN      Laborcita  Formation -42.1  312.1 -43.0  313.4 -32.7   
         2  5  3.4    \#       Wescogame Formation -44.1  303.9 -46.3  306.8 -38.2   
         3  6  3.1    I        Glenshaw Formation -28.6  299.9 -28.6  299.9 -28.6   
         4  5  1.8  NaN  Lower  Casper  Formation -45.7  308.6 -50.5  314.6 -37.6   
         
            RLon               EULER  Age GPDB RefNo/Reference  
         0  43.0  (63.2/\_ 13.9/79.9)  300               302, T  
         1  52.9  (63.2/\_ 13.9/79.9)  301             1311,  T  
         2  51.4  (63.2/\_ 13.9/79.9)  301             1311,  T  
         3  32.4  (63.2/\_ 13.9/79.9)  303        Kodama (2009)  
         4  59.8  (63.2/\_ 13.9/79.9)  303             1455,  T  
\end{Verbatim}
        
    \begin{Verbatim}[commandchars=\\\{\}]
{\color{incolor}In [{\color{incolor}28}]:} \PY{c}{\PYZsh{} initiate the figure as in the plot\PYZus{}pole example}
         \PY{n}{plt}\PY{o}{.}\PY{n}{figure}\PY{p}{(}\PY{n}{figsize}\PY{o}{=}\PY{p}{(}\PY{l+m+mi}{6}\PY{p}{,} \PY{l+m+mi}{6}\PY{p}{)}\PY{p}{)}
         \PY{n}{pmap} \PY{o}{=} \PY{n}{Basemap}\PY{p}{(}\PY{n}{projection}\PY{o}{=}\PY{l+s}{\PYZsq{}}\PY{l+s}{ortho}\PY{l+s}{\PYZsq{}}\PY{p}{,}\PY{n}{lat\PYZus{}0}\PY{o}{=}\PY{l+m+mi}{10}\PY{p}{,}\PY{n}{lon\PYZus{}0}\PY{o}{=}\PY{l+m+mi}{320}\PY{p}{,}
                        \PY{n}{resolution}\PY{o}{=}\PY{l+s}{\PYZsq{}}\PY{l+s}{c}\PY{l+s}{\PYZsq{}}\PY{p}{,}\PY{n}{area\PYZus{}thresh}\PY{o}{=}\PY{l+m+mi}{50000}\PY{p}{)}
         \PY{n}{pmap}\PY{o}{.}\PY{n}{drawcoastlines}\PY{p}{(}\PY{n}{linewidth}\PY{o}{=}\PY{l+m+mf}{0.25}\PY{p}{)}
         \PY{n}{pmap}\PY{o}{.}\PY{n}{fillcontinents}\PY{p}{(}\PY{n}{color}\PY{o}{=}\PY{l+s}{\PYZsq{}}\PY{l+s}{bisque}\PY{l+s}{\PYZsq{}}\PY{p}{,}\PY{n}{lake\PYZus{}color}\PY{o}{=}\PY{l+s}{\PYZsq{}}\PY{l+s}{white}\PY{l+s}{\PYZsq{}}\PY{p}{,}\PY{n}{zorder}\PY{o}{=}\PY{l+m+mi}{1}\PY{p}{)}
         \PY{n}{pmap}\PY{o}{.}\PY{n}{drawmapboundary}\PY{p}{(}\PY{n}{fill\PYZus{}color}\PY{o}{=}\PY{l+s}{\PYZsq{}}\PY{l+s}{white}\PY{l+s}{\PYZsq{}}\PY{p}{)}
         \PY{n}{pmap}\PY{o}{.}\PY{n}{drawmeridians}\PY{p}{(}\PY{n}{np}\PY{o}{.}\PY{n}{arange}\PY{p}{(}\PY{l+m+mi}{0}\PY{p}{,}\PY{l+m+mi}{360}\PY{p}{,}\PY{l+m+mi}{30}\PY{p}{)}\PY{p}{)}
         \PY{n}{pmap}\PY{o}{.}\PY{n}{drawparallels}\PY{p}{(}\PY{n}{np}\PY{o}{.}\PY{n}{arange}\PY{p}{(}\PY{o}{\PYZhy{}}\PY{l+m+mi}{90}\PY{p}{,}\PY{l+m+mi}{90}\PY{p}{,}\PY{l+m+mi}{30}\PY{p}{)}\PY{p}{)}
         
         \PY{c}{\PYZsh{} Loop through the uploaded data and use the plot\PYZus{}pole\PYZus{}colorbar function}
         \PY{c}{\PYZsh{} (instead of plot\PYZus{}pole) to plot the individual poles. The input of this}
         \PY{c}{\PYZsh{} function is very similar to that of plot\PYZus{}pole but has the additional}
         \PY{c}{\PYZsh{} arguments of (1)AGE, (2)MINIMUM AND (3)MAXIMUM AGES OF PLOTTED POLES.}
         \PY{c}{\PYZsh{} Note that the ages are treated as negative numbers \PYZhy{}\PYZhy{} this just determines}
         \PY{c}{\PYZsh{} the direction of the colorbar.}
         \PY{k}{for} \PY{n}{n} \PY{o+ow}{in} \PY{n+nb}{xrange} \PY{p}{(}\PY{l+m+mi}{0}\PY{p}{,} \PY{n+nb}{len}\PY{p}{(}\PY{n}{Laurentia\PYZus{}Pole\PYZus{}Compilation}\PY{p}{)}\PY{p}{)}\PY{p}{:}
              \PY{n}{m} \PY{o}{=} \PY{n}{ipmag}\PY{o}{.}\PY{n}{plot\PYZus{}pole\PYZus{}colorbar}\PY{p}{(}\PY{n}{pmap}\PY{p}{,} \PY{n}{Laurentia\PYZus{}Pole\PYZus{}Compilation}\PY{p}{[}\PY{l+s}{\PYZsq{}}\PY{l+s}{CLon}\PY{l+s}{\PYZsq{}}\PY{p}{]}\PY{p}{[}\PY{n}{n}\PY{p}{]}\PY{p}{,}
                                           \PY{n}{Laurentia\PYZus{}Pole\PYZus{}Compilation}\PY{p}{[}\PY{l+s}{\PYZsq{}}\PY{l+s}{CLat}\PY{l+s}{\PYZsq{}}\PY{p}{]}\PY{p}{[}\PY{n}{n}\PY{p}{]}\PY{p}{,}
                                           \PY{n}{Laurentia\PYZus{}Pole\PYZus{}Compilation}\PY{p}{[}\PY{l+s}{\PYZsq{}}\PY{l+s}{A95}\PY{l+s}{\PYZsq{}}\PY{p}{]}\PY{p}{[}\PY{n}{n}\PY{p}{]}\PY{p}{,}
                                           \PY{o}{\PYZhy{}}\PY{n}{Laurentia\PYZus{}Pole\PYZus{}Compilation}\PY{p}{[}\PY{l+s}{\PYZsq{}}\PY{l+s}{Age}\PY{l+s}{\PYZsq{}}\PY{p}{]}\PY{p}{[}\PY{n}{n}\PY{p}{]}\PY{p}{,}
                                           \PY{o}{\PYZhy{}}\PY{l+m+mi}{532}\PY{p}{,}
                                           \PY{o}{\PYZhy{}}\PY{l+m+mi}{300}\PY{p}{,}
                                           \PY{n}{markersize}\PY{o}{=}\PY{l+m+mi}{80}\PY{p}{,} \PY{n}{color}\PY{o}{=}\PY{l+s}{\PYZdq{}}\PY{l+s}{k}\PY{l+s}{\PYZdq{}}\PY{p}{,} \PY{n}{alpha}\PY{o}{=}\PY{l+m+mi}{1}\PY{p}{)}
         
         \PY{n}{pmap}\PY{o}{.}\PY{n}{colorbar}\PY{p}{(}\PY{n}{m}\PY{p}{,}\PY{n}{location}\PY{o}{=}\PY{l+s}{\PYZsq{}}\PY{l+s}{bottom}\PY{l+s}{\PYZsq{}}\PY{p}{,}\PY{n}{pad}\PY{o}{=}\PY{l+s}{\PYZdq{}}\PY{l+s}{5}\PY{l+s}{\PYZpc{}}\PY{l+s}{\PYZdq{}}\PY{p}{,}\PY{n}{label}\PY{o}{=}\PY{l+s}{\PYZsq{}}\PY{l+s}{Age of magnetization (Ma)}\PY{l+s}{\PYZsq{}}\PY{p}{)}
         
         \PY{c}{\PYZsh{} Optional save (uncomment to save the figure)}
         \PY{c}{\PYZsh{}plt.savefig(\PYZsq{}Additional\PYZus{}Notebook\PYZus{}Output/plot\PYZus{}pole\PYZus{}colorbar\PYZus{}example.pdf\PYZsq{})}
         
         \PY{n}{plt}\PY{o}{.}\PY{n}{show}\PY{p}{(}\PY{p}{)}
\end{Verbatim}

    \begin{center}
    \adjustimage{max size={0.9\linewidth}{0.9\paperheight}}{Additional_PmagPy_Examples_files/Additional_PmagPy_Examples_78_0.pdf}
    \end{center}
    { \hspace*{\fill} \\}
    
    \hyperref[Jupyter-notebook-demonstrating-the-use-of-additional-PmagPy-functions]{Go
to Top}

    \section{Working with anisotropy
data}\label{working-with-anisotropy-data}

    The following code demonstrates reading magnetic anisotropy data into a
pandas DataFrame.

    \begin{Verbatim}[commandchars=\\\{\}]
{\color{incolor}In [{\color{incolor}29}]:} \PY{n}{aniso\PYZus{}data} \PY{o}{=} \PY{n}{pd}\PY{o}{.}\PY{n}{read\PYZus{}csv}\PY{p}{(}\PY{l+s}{\PYZsq{}}\PY{l+s}{./Additional\PYZus{}Data/ani\PYZus{}depthplot/rmag\PYZus{}anisotropy.txt}\PY{l+s}{\PYZsq{}}\PY{p}{,}
                                  \PY{n}{delimiter}\PY{o}{=}\PY{l+s}{\PYZsq{}}\PY{l+s+se}{\PYZbs{}t}\PY{l+s}{\PYZsq{}}\PY{p}{,}\PY{n}{skiprows}\PY{o}{=}\PY{l+m+mi}{1}\PY{p}{)}
         \PY{n}{aniso\PYZus{}data}\PY{o}{.}\PY{n}{head}\PY{p}{(}\PY{p}{)}
\end{Verbatim}

            \begin{Verbatim}[commandchars=\\\{\}]
{\color{outcolor}Out[{\color{outcolor}29}]:}    anisotropy\_n  anisotropy\_s1  anisotropy\_s2  anisotropy\_s3  anisotropy\_s4  \textbackslash{}
         0           192       0.332294       0.332862       0.334844      -0.000048   
         1           192       0.333086       0.332999       0.333916      -0.000262   
         2           192       0.333750       0.332208       0.334041      -0.000699   
         3           192       0.330565       0.333928       0.335507       0.000603   
         4           192       0.332747       0.332939       0.334314      -0.001516   
         
            anisotropy\_s5  anisotropy\_s6  anisotropy\_sigma  anisotropy\_tilt\_correction  \textbackslash{}
         0       0.000027      -0.000263          0.000122                          -1   
         1      -0.000322       0.000440          0.000259                          -1   
         2       0.000663       0.002888          0.000093                          -1   
         3       0.000212      -0.000932          0.000198                          -1   
         4      -0.000311      -0.000099          0.000162                          -1   
         
           anisotropy\_type      anisotropy\_unit  er\_analyst\_mail\_names  \textbackslash{}
         0             AMS  Normalized by trace                    NaN   
         1             AMS  Normalized by trace                    NaN   
         2             AMS  Normalized by trace                    NaN   
         3             AMS  Normalized by trace                    NaN   
         4             AMS  Normalized by trace                    NaN   
         
           er\_citation\_names er\_location\_name          er\_sample\_name  \textbackslash{}
         0        This study           U1361A  318-U1361A-001H-2-W-35   
         1        This study           U1361A  318-U1361A-001H-3-W-90   
         2        This study           U1361A  318-U1361A-001H-4-W-50   
         3        This study           U1361A  318-U1361A-001H-5-W-59   
         4        This study           U1361A  318-U1361A-001H-6-W-60   
         
                      er\_site\_name        er\_specimen\_name       magic\_method\_codes  
         0  318-U1361A-001H-2-W-35  318-U1361A-001H-2-W-35  LP-X:AE-H:LP-AN-MS:SO-V  
         1  318-U1361A-001H-3-W-90  318-U1361A-001H-3-W-90  LP-X:AE-H:LP-AN-MS:SO-V  
         2  318-U1361A-001H-4-W-50  318-U1361A-001H-4-W-50  LP-X:AE-H:LP-AN-MS:SO-V  
         3  318-U1361A-001H-5-W-59  318-U1361A-001H-5-W-59  LP-X:AE-H:LP-AN-MS:SO-V  
         4  318-U1361A-001H-6-W-60  318-U1361A-001H-6-W-60  LP-X:AE-H:LP-AN-MS:SO-V  
\end{Verbatim}
        
    The function \textbf{ipmag.aniso\_depthplot} is one example of how
PmagPy works with such data to generate plots.

    \begin{Verbatim}[commandchars=\\\{\}]
{\color{incolor}In [{\color{incolor}30}]:} \PY{n}{ipmag}\PY{o}{.}\PY{n}{aniso\PYZus{}depthplot}\PY{p}{(}\PY{n}{dir\PYZus{}path}\PY{o}{=}\PY{l+s}{\PYZsq{}}\PY{l+s}{./Additional\PYZus{}Data/ani\PYZus{}depthplot/}\PY{l+s}{\PYZsq{}}\PY{p}{)}\PY{p}{;}
\end{Verbatim}

    \begin{center}
    \adjustimage{max size={0.9\linewidth}{0.9\paperheight}}{Additional_PmagPy_Examples_files/Additional_PmagPy_Examples_84_0.pdf}
    \end{center}
    { \hspace*{\fill} \\}
    
    \hyperref[Jupyter-notebook-demonstrating-the-use-of-additional-PmagPy-functions]{Go
to Top}

    \section{Curie temperature data}\label{curie-temperature-data}

    \begin{Verbatim}[commandchars=\\\{\}]
{\color{incolor}In [{\color{incolor}31}]:} \PY{n}{ipmag}\PY{o}{.}\PY{n}{curie}\PY{p}{(}\PY{n}{path\PYZus{}to\PYZus{}file}\PY{o}{=}\PY{l+s}{\PYZsq{}}\PY{l+s}{./Additional\PYZus{}Data/curie/}\PY{l+s}{\PYZsq{}}\PY{p}{,}
                     \PY{n}{file\PYZus{}name}\PY{o}{=}\PY{l+s}{\PYZsq{}}\PY{l+s}{curie\PYZus{}example.dat}\PY{l+s}{\PYZsq{}}\PY{p}{,} \PY{n}{save}\PY{o}{=}\PY{n+nb+bp}{True}\PY{p}{,} 
                     \PY{n}{save\PYZus{}folder}\PY{o}{=}\PY{l+s}{\PYZsq{}}\PY{l+s}{Additional\PYZus{}Notebook\PYZus{}Output/curie/}\PY{l+s}{\PYZsq{}}\PY{p}{)}
\end{Verbatim}

    \begin{Verbatim}[commandchars=\\\{\}]
second deriative maximum is at T=205
    \end{Verbatim}

    \begin{center}
    \adjustimage{max size={0.9\linewidth}{0.9\paperheight}}{Additional_PmagPy_Examples_files/Additional_PmagPy_Examples_87_1.pdf}
    \end{center}
    { \hspace*{\fill} \\}
    
    \begin{center}
    \adjustimage{max size={0.9\linewidth}{0.9\paperheight}}{Additional_PmagPy_Examples_files/Additional_PmagPy_Examples_87_2.pdf}
    \end{center}
    { \hspace*{\fill} \\}
    
    \begin{center}
    \adjustimage{max size={0.9\linewidth}{0.9\paperheight}}{Additional_PmagPy_Examples_files/Additional_PmagPy_Examples_87_3.pdf}
    \end{center}
    { \hspace*{\fill} \\}
    
    \begin{center}
    \adjustimage{max size={0.9\linewidth}{0.9\paperheight}}{Additional_PmagPy_Examples_files/Additional_PmagPy_Examples_87_4.pdf}
    \end{center}
    { \hspace*{\fill} \\}
    
    \hyperref[Jupyter-notebook-demonstrating-the-use-of-additional-PmagPy-functions]{Go
to Top}

    \section{Day plots}\label{day-plots}

    Here we demonstrate the function \textbf{ipmag.dayplot}, which creates
Day plots, squareness/coercivity and squareness/coercivity of remanence
diagrams using hysteresis data.

    \begin{Verbatim}[commandchars=\\\{\}]
{\color{incolor}In [{\color{incolor}32}]:} \PY{n}{ipmag}\PY{o}{.}\PY{n}{dayplot}\PY{p}{(}\PY{n}{path\PYZus{}to\PYZus{}file}\PY{o}{=}\PY{l+s}{\PYZsq{}}\PY{l+s}{./Additional\PYZus{}Data/dayplot\PYZus{}magic/}\PY{l+s}{\PYZsq{}}\PY{p}{,} 
                       \PY{n}{hyst\PYZus{}file}\PY{o}{=}\PY{l+s}{\PYZsq{}}\PY{l+s}{dayplot\PYZus{}magic\PYZus{}example.dat}\PY{l+s}{\PYZsq{}}\PY{p}{,}
                       \PY{n}{save}\PY{o}{=}\PY{n+nb+bp}{True}\PY{p}{,}\PY{n}{save\PYZus{}folder}\PY{o}{=}\PY{l+s}{\PYZsq{}}\PY{l+s}{Additional\PYZus{}Notebook\PYZus{}Output/day\PYZus{}plots/}\PY{l+s}{\PYZsq{}}\PY{p}{)}\PY{p}{;}
\end{Verbatim}

    \begin{center}
    \adjustimage{max size={0.9\linewidth}{0.9\paperheight}}{Additional_PmagPy_Examples_files/Additional_PmagPy_Examples_91_0.pdf}
    \end{center}
    { \hspace*{\fill} \\}
    
    \begin{center}
    \adjustimage{max size={0.9\linewidth}{0.9\paperheight}}{Additional_PmagPy_Examples_files/Additional_PmagPy_Examples_91_1.pdf}
    \end{center}
    { \hspace*{\fill} \\}
    
    \begin{center}
    \adjustimage{max size={0.9\linewidth}{0.9\paperheight}}{Additional_PmagPy_Examples_files/Additional_PmagPy_Examples_91_2.pdf}
    \end{center}
    { \hspace*{\fill} \\}
    
    
    \begin{verbatim}
<matplotlib.figure.Figure at 0x114da9290>
    \end{verbatim}

    
    \hyperref[Jupyter-notebook-demonstrating-the-use-of-additional-PmagPy-functions]{Go
to Top}

    \section{Hysteresis Loops}\label{hysteresis-loops}

    The function \textbf{ipmag.hysteresis\_magic} also generates a set of
hysteresis plots with data from a \textit{magic\_measurements} file.

    \begin{Verbatim}[commandchars=\\\{\}]
{\color{incolor}In [{\color{incolor}33}]:} \PY{n}{ipmag}\PY{o}{.}\PY{n}{hysteresis\PYZus{}magic}\PY{p}{(}\PY{n}{path\PYZus{}to\PYZus{}file}\PY{o}{=}\PY{l+s}{\PYZsq{}}\PY{l+s}{./Additional\PYZus{}Data/hysteresis\PYZus{}magic/}\PY{l+s}{\PYZsq{}}\PY{p}{,}
                          \PY{n}{hyst\PYZus{}file}\PY{o}{=}\PY{l+s}{\PYZsq{}}\PY{l+s}{hysteresis\PYZus{}magic\PYZus{}example.dat}\PY{l+s}{\PYZsq{}}\PY{p}{,} \PY{n}{save}\PY{o}{=}\PY{n+nb+bp}{True}\PY{p}{,}
                         \PY{n}{save\PYZus{}folder}\PY{o}{=}\PY{l+s}{\PYZsq{}}\PY{l+s}{./Additional\PYZus{}Notebook\PYZus{}Output/hysteresis}\PY{l+s}{\PYZsq{}}\PY{p}{)}
\end{Verbatim}

    \begin{Verbatim}[commandchars=\\\{\}]
IS06a-1 1 out of  8
    \end{Verbatim}

    \begin{center}
    \adjustimage{max size={0.9\linewidth}{0.9\paperheight}}{Additional_PmagPy_Examples_files/Additional_PmagPy_Examples_95_1.pdf}
    \end{center}
    { \hspace*{\fill} \\}
    
    \begin{Verbatim}[commandchars=\\\{\}]
IS06a-2 2 out of  8
    \end{Verbatim}

    \begin{center}
    \adjustimage{max size={0.9\linewidth}{0.9\paperheight}}{Additional_PmagPy_Examples_files/Additional_PmagPy_Examples_95_3.pdf}
    \end{center}
    { \hspace*{\fill} \\}
    
    \begin{Verbatim}[commandchars=\\\{\}]
IS06a-3 3 out of  8
    \end{Verbatim}

    \begin{center}
    \adjustimage{max size={0.9\linewidth}{0.9\paperheight}}{Additional_PmagPy_Examples_files/Additional_PmagPy_Examples_95_5.pdf}
    \end{center}
    { \hspace*{\fill} \\}
    
    \begin{Verbatim}[commandchars=\\\{\}]
IS06a-4 4 out of  8
    \end{Verbatim}

    \begin{center}
    \adjustimage{max size={0.9\linewidth}{0.9\paperheight}}{Additional_PmagPy_Examples_files/Additional_PmagPy_Examples_95_7.pdf}
    \end{center}
    { \hspace*{\fill} \\}
    
    \begin{Verbatim}[commandchars=\\\{\}]
IS06a-5 5 out of  8
    \end{Verbatim}

    \begin{center}
    \adjustimage{max size={0.9\linewidth}{0.9\paperheight}}{Additional_PmagPy_Examples_files/Additional_PmagPy_Examples_95_9.pdf}
    \end{center}
    { \hspace*{\fill} \\}
    
    \begin{Verbatim}[commandchars=\\\{\}]
IS06a-6 6 out of  8
    \end{Verbatim}

    \begin{center}
    \adjustimage{max size={0.9\linewidth}{0.9\paperheight}}{Additional_PmagPy_Examples_files/Additional_PmagPy_Examples_95_11.pdf}
    \end{center}
    { \hspace*{\fill} \\}
    
    \begin{Verbatim}[commandchars=\\\{\}]
IS06a-8 7 out of  8
    \end{Verbatim}

    \begin{center}
    \adjustimage{max size={0.9\linewidth}{0.9\paperheight}}{Additional_PmagPy_Examples_files/Additional_PmagPy_Examples_95_13.pdf}
    \end{center}
    { \hspace*{\fill} \\}
    
    \begin{Verbatim}[commandchars=\\\{\}]
IS06a-9 8 out of  8
    \end{Verbatim}

    \begin{center}
    \adjustimage{max size={0.9\linewidth}{0.9\paperheight}}{Additional_PmagPy_Examples_files/Additional_PmagPy_Examples_95_15.pdf}
    \end{center}
    { \hspace*{\fill} \\}
    
    \hyperref[Jupyter-notebook-demonstrating-the-use-of-additional-PmagPy-functions]{Go
to Top}

    \section{Demagnetization Curves}\label{demagnetization-curves}

    The function \textbf{ipmag.demag\_magic} filters and plots
demagnetization data. These data will be read and combined by
expedition, location, site or sample according to the \textit{plot\_by}
keyword argument. Alternatively, you can choose to plot each specimen
measurement individually. By default, all plots generated by this
function will be shown. If you only wish to plot a single subset of
data, you can use the keyword argument \textit{individual} to specify the
name of the one site, location, sample, etc. that you would like to see.

Below, we use the \textit{magic\_measurements.txt} file of Swanson-Hysell
et al., 2014 to plot demagnetization data by site. We then specify an
individual site (`SI1(58.1 to 64.1)') that will plot alone. Like other
functions, these plots can be optionally saved out of the notebook.

    \begin{Verbatim}[commandchars=\\\{\}]
{\color{incolor}In [{\color{incolor}34}]:} \PY{n}{ipmag}\PY{o}{.}\PY{n}{demag\PYZus{}magic}\PY{p}{(}\PY{n}{path\PYZus{}to\PYZus{}file}\PY{o}{=}\PY{l+s}{\PYZsq{}}\PY{l+s}{./Example\PYZus{}Data/Swanson\PYZhy{}Hysell2014/}\PY{l+s}{\PYZsq{}}\PY{p}{,} 
                          \PY{n}{plot\PYZus{}by}\PY{o}{=}\PY{l+s}{\PYZsq{}}\PY{l+s}{site}\PY{l+s}{\PYZsq{}}\PY{p}{,} \PY{n}{treat}\PY{o}{=}\PY{l+s}{\PYZsq{}}\PY{l+s}{AF}\PY{l+s}{\PYZsq{}}\PY{p}{,} \PY{n}{individual}\PY{o}{=} \PY{l+s}{\PYZsq{}}\PY{l+s}{SI1(58.1 to 64.1)}\PY{l+s}{\PYZsq{}}\PY{p}{)}
\end{Verbatim}

    \begin{Verbatim}[commandchars=\\\{\}]
13395  records read from  ./Example\_Data/Swanson-Hysell2014/magic\_measurements.txt
SI1(58.1 to 64.1) plotting by:  er\_site\_name
    \end{Verbatim}

    \begin{center}
    \adjustimage{max size={0.9\linewidth}{0.9\paperheight}}{Additional_PmagPy_Examples_files/Additional_PmagPy_Examples_99_1.pdf}
    \end{center}
    { \hspace*{\fill} \\}
    
    \hyperref[Jupyter-notebook-demonstrating-the-use-of-additional-PmagPy-functions]{Go
to Top}

    \section{Interactive plotting}\label{interactive-plotting}

    IPy Widgets are part of what makes the Jupyter notebook environment so
powerful -- these widgets allow user interaction with figures. We first
demonstrate the use of the \textbf{interact} widget, imported below.

\textit{Note: If you do not have the ipywidgets package installed, you may
choose to either install it through Anaconda or Enthought (depending on
your Python distribution), manually install it (a bit more difficult),
or simply skip the next few blocks of code. Below are quick installation
instructions for those with either an Anaconda or Enthought Canopy
distribution.}

\textbf{\textit{Installation on Anaconda}}

On the command line, enter

\begin{Shaded}
\begin{Highlighting}[]
\KeywordTok{conda} \NormalTok{install ipywidgets}
\end{Highlighting}
\end{Shaded}

Make sure this installs within the Python 2 environment (if you have
Python 3 as your default environment).

\textbf{\textit{Installation on Enthought Canopy}}

Open the Canopy application and navigate to the Package Manager. Search
for and install ipywidgets.

    \begin{Verbatim}[commandchars=\\\{\}]
{\color{incolor}In [{\color{incolor}35}]:} \PY{k+kn}{from} \PY{n+nn}{ipywidgets} \PY{k+kn}{import} \PY{n}{interact}
\end{Verbatim}

    The \textbf{interact} widget allows adjustable values (within specified
bounds) to all keyword arguments of a function. It can be used as a
wrapper function, as seen below. Here we create a new function,
\textbf{squish\_interactive}, which streamlines the
\textbf{ipmag.squish} function and automatically inputs the
fisher-distributed directions created at the beginning of the notebook.
This new function also allows us to reduce the keyword arguments to the
\textit{factor} variable, which is the only value we want to be actively
adjustable. Finally, to make the \textbf{squish\_interactive} function
interactive in the notebook, we ``wrap'' this function with
\textbf{@interact} placed directly above our new function.

    \begin{Verbatim}[commandchars=\\\{\}]
{\color{incolor}In [{\color{incolor}36}]:} \PY{n+nd}{@interact}
         \PY{k}{def} \PY{n+nf}{squish\PYZus{}interactive}\PY{p}{(}\PY{n}{flattening\PYZus{}factor}\PY{o}{=}\PY{p}{(}\PY{l+m+mf}{0.}\PY{p}{,}\PY{l+m+mf}{1.}\PY{p}{,}\PY{o}{.}\PY{l+m+mi}{1}\PY{p}{)}\PY{p}{)}\PY{p}{:}
             \PY{n}{squished\PYZus{}incs} \PY{o}{=} \PY{p}{[}\PY{p}{]}
             \PY{k}{for} \PY{n}{inclination} \PY{o+ow}{in} \PY{n}{inclinations}\PY{p}{:}
                 \PY{n}{squished\PYZus{}incs}\PY{o}{.}\PY{n}{append}\PY{p}{(}\PY{n}{ipmag}\PY{o}{.}\PY{n}{squish}\PY{p}{(}\PY{n}{inclination}\PY{p}{,} \PY{n}{flattening\PYZus{}factor}\PY{p}{)}\PY{p}{)}
         
             \PY{c}{\PYZsh{} plot the squished directional data}
             \PY{n}{plt}\PY{o}{.}\PY{n}{figure}\PY{p}{(}\PY{n}{num}\PY{o}{=}\PY{l+m+mi}{1}\PY{p}{,}\PY{n}{figsize}\PY{o}{=}\PY{p}{(}\PY{l+m+mi}{6}\PY{p}{,}\PY{l+m+mi}{6}\PY{p}{)}\PY{p}{)}
             \PY{n}{ipmag}\PY{o}{.}\PY{n}{plot\PYZus{}net}\PY{p}{(}\PY{l+m+mi}{1}\PY{p}{)}
             \PY{n}{ipmag}\PY{o}{.}\PY{n}{plot\PYZus{}di}\PY{p}{(}\PY{n}{declinations}\PY{p}{,}\PY{n}{squished\PYZus{}incs}\PY{p}{)}
\end{Verbatim}

    \begin{center}
    \adjustimage{max size={0.9\linewidth}{0.9\paperheight}}{Additional_PmagPy_Examples_files/Additional_PmagPy_Examples_105_0.pdf}
    \end{center}
    { \hspace*{\fill} \\}
    
    \textbf{interact} can also be used as a regular function call -- the
name of the interactive function is passed as the first argument,
followed by the adjustable keyword arguments. Below, we demonstrate
passing the \textit{curie} function's parameters to \textbf{interact}.

    \begin{Verbatim}[commandchars=\\\{\}]
{\color{incolor}In [{\color{incolor}37}]:} \PY{n}{interact}\PY{p}{(}\PY{n}{ipmag}\PY{o}{.}\PY{n}{curie}\PY{p}{,} \PY{n}{path\PYZus{}to\PYZus{}file}\PY{o}{=}\PY{l+s}{\PYZsq{}}\PY{l+s}{./Additional\PYZus{}Data/curie/}\PY{l+s}{\PYZsq{}}\PY{p}{,}\PY{n}{file\PYZus{}name}\PY{o}{=}\PY{l+s}{\PYZsq{}}\PY{l+s}{curie\PYZus{}example.dat}\PY{l+s}{\PYZsq{}}\PY{p}{,}\PY{n}{window\PYZus{}length} \PY{o}{=} \PY{p}{(}\PY{l+m+mi}{1}\PY{p}{,}\PY{l+m+mi}{60}\PY{p}{)}\PY{p}{)}\PY{p}{;}
\end{Verbatim}

    \begin{Verbatim}[commandchars=\\\{\}]
second deriative maximum is at T=205
    \end{Verbatim}

    \begin{center}
    \adjustimage{max size={0.9\linewidth}{0.9\paperheight}}{Additional_PmagPy_Examples_files/Additional_PmagPy_Examples_107_1.pdf}
    \end{center}
    { \hspace*{\fill} \\}
    
    \begin{center}
    \adjustimage{max size={0.9\linewidth}{0.9\paperheight}}{Additional_PmagPy_Examples_files/Additional_PmagPy_Examples_107_2.pdf}
    \end{center}
    { \hspace*{\fill} \\}
    
    \begin{center}
    \adjustimage{max size={0.9\linewidth}{0.9\paperheight}}{Additional_PmagPy_Examples_files/Additional_PmagPy_Examples_107_3.pdf}
    \end{center}
    { \hspace*{\fill} \\}
    
    \begin{center}
    \adjustimage{max size={0.9\linewidth}{0.9\paperheight}}{Additional_PmagPy_Examples_files/Additional_PmagPy_Examples_107_4.pdf}
    \end{center}
    { \hspace*{\fill} \\}
    
    \hyperref[Jupyter-notebook-demonstrating-the-use-of-additional-PmagPy-functions]{Go
to Top}

    \begin{Verbatim}[commandchars=\\\{\}]
{\color{incolor}In [{\color{incolor} }]:} 
\end{Verbatim}


    % Add a bibliography block to the postdoc
    
    
    
    \end{document}
